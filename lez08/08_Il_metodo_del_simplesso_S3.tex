\documentclass{beamer}
\usepackage{booktabs}
\def\presentationtype{3}
\input{../template/savoia_benincasa}
\setbeamersize{description width=10pt}

\setbeamertemplate{enumerate item}{Es. \nolezione.\presentationtype.\arabic{enumi})}
\begin{document}

\generatitolo

\begin{frame}[allowframebreaks]{Esercizi}
    \begin{enumerate}    
     \item Si effettuino le due operazioni di pivot sulla matrice tableau per l'esempio del problema di mix ottimo di produzione.\\~\\
     1) Si riscriva la matrice tableau;\\~\\
     2) si effettui l’operazione di pivot in posizione (2, 1) e si scriva il secondo tableau;\\~\\
     3) si effettui, sul secondo tableau, l’operazione di pivot in posizione (1, 2) e si scriva il terzo tableau.\\~\\
     In corrispondenza di ogni tableau si legga la soluzione di base corrispondente e la si metta in corrispondenza con la soluzione grafica già effettuata in una precedente lezione.\\~\\
     %Si produca una relazione e la si carichi sulla piattaforma di e-learning
    \end{enumerate}
\end{frame}

\begin{frame}[fragile]{Il tableau dell'esercizio}
\centering
\begin{tikzpicture}
{\tiny
\node[%
  align=center,
  text width=3.5em,
  text height=5ex,
  row 1/.style={gray, text height=1em},
  row 2/.style={blue!50!gray, text height=1em},
  row 3/.style={blue},
  column 1/.style={gray,text width=1em},
  column 2/.style={red!50!gray,text width=1.5em},
  column 3/.style={red},
  row 3 column 1/.style={gray},
  row 3 column 2/.style={red!50!gray},
  row 3 column 3/.style={red},
  matrix of math nodes] (M)
{%
% Indice delle righe (M-1)
~&~&0&1&2&3&4&5\\
% Intestazione delle colonne (M-2)
~ &~&  \mbox{rhs}& x_1 & x_2 & x_3 & x_4 & x_5\\
% Riga 0, fo e costi ridotti (M-3)
0 &  z  &    0 & -120 & -40 & 0 & 0 & 0 \\
1 & x_1 & 2200 &   40 &  20 & 1 & 0 & 0 \\
2 & x_2 &  320 &    8 &   2 & 0 & 1 & 0 \\
3 & x_3 &  100 &    1 &   1 & 0 & 0 & 1 \\
};

% riquadro
\draw(M-3-3.north west) -- (M-3-8.north east) -- (M-6-8.south east) -- (M-6-3.south west) -- cycle;

% separatore orizzontale
\draw[dashed] (M-4-3.north west) -- (M-4-8.north east);

% separatore verticale
\draw[dashed] (M-3-4.north west) -- (M-6-4.south west);

}
\end{tikzpicture}
%sono fragile: lasciami uno spazio vuoto

\end{frame}

\end{document}
