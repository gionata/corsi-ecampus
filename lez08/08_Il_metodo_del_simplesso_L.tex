\documentclass{beamer}
\def\presentationtype{0}
\input{../template/savoia_benincasa}

\begin{document}

\generatitolo

\section{Risolvere un Programma Lineare}

\begin{frame}[allowframebreaks]{Il metodo del Simplesso}

\begin{itemize}
\item Il metodo del simplesso primale \`e un algoritmo per la risoluzione algebrica di problemi di P.L. espressi in forma canonica forte.

\item \`E il metodo proposto da George Dantzig nel 1947
per la risoluzione dei problemi generali di P.L.
Per anni \`e stato anche l'unico algoritmo disponibile.

\item Rimane un metodo tra i pi\`u utilizzati per la soluzione di problemi di P.L. nonostante attualmente siano disponibili altri algoritmi.

\item Pu\`o essere utilizzato, insieme ad altre tecniche, nella risoluzione dei problemi di P.L.I.

\item Permette di effettuare agevolmente l'analisi di post-ottimalit\`a.

\item  Esistono molti pacchetti software che lo implementano, sia gratuiti che commerciali.
\end{itemize}
\end{frame}

\begin{frame}[allowframebreaks]{Le possibili soluzioni}
Un problema di P.L. soddisfa sempre uno dei tre casi seguenti:

\begin{enumerate}
\item il problema \`e inammissibile: la regione ammissibile \`e vuota;

\item il problema \`e illimitato: \`e possibile trovare delle soluzioni ammissibili facendo crescere a piacere il valore di una o pi\`u variabili decisionali in modo che la funzione obiettivo diminuisca (o aumenti per problemi di massimo) a piacere;

\item il problema ammette soluzione ottima: esiste almeno una soluzione ammissibile che minimizza (massimizza) il valore della funzione obiettivo. Tale valore \`e limitato.
\end{enumerate}	

\begin{itemize}
\item   Risolvere un problema di P.L. significa 
  riconoscere il verificarsi di uno dei tre casi

\item   se \`e verificato il caso 3, fornire
\begin{itemize}
  \item il valore assunto all'ottimo dalle variabili decisionali (il vettore $\vecX^\star$)
  \item il valore assunto all'ottimo dalla funzione obiettivo ($z^\star$)
\end{itemize}
\end{itemize}
\end{frame}

\section{Matrice tableau}

\begin{frame}{Matrice tableau}
\begin{columns}
\begin{column}{0.7\textwidth}
{\small %
\begin{itemize}
\item Il problema di P.L. in forma canonica pu\`o essere
	rappresentato in una tabella di $m+1$ righe ed $n+1$
	colonne
	
\item La tabella $\matr{M}$ \`e una matrice, detta 
	\emph{matrice tableau}.

\item Rappresenta le equazioni del problema di
   programmazione lineare in modo compatto.

\item  Si noti che in alto a sinistra si \`e posto $-d$
che e i termini noti sono stati spostati sulla sinistra.
\end{itemize} %
}
\end{column}
\begin{column}{0.3\textwidth}
{\tiny
\centering\fbox{
\begin{minipage}{0.5\textwidth}
$\min z = \vecC^T x + d$
$$\begin{cases}
\matrA \vecX = \vecB\\
\vecX \geq \vec{0}
\end{cases}$$
\end{minipage}
}

\vfill

%$$\bordermatrix{
%        &  -z   & x_{1}    &  x_{2}	&\cdots&x_{n} \cr
%   f.o.\   &  -d      &  c_1       & c_2 & \cdots&c_{n} \cr
%	1    &  b_{1}& a_{11} & a_{12} &\cdots&a_{1n} \cr
%    2    & b_{2} & a_{21} & a_{22} &\cdots &a_{2n} \cr
%\vdots & \vdots &\vdots &\vdots &\ddots &\vdots \cr
%	m    &  b_{m}& a_{m1} & a_{m2}    &\cdots&a_{mn} \cr
%}$$
%
%\vfill

\begin{center}
\begin{tabular}{|c|p{0.2\textwidth}|}
\hline
$-d$ & \multicolumn{1}{|c|}{\phantom{\LARGE{X}}$\vecC^T$\phantom{\LARGE{X}}}\\
\hline
$\vecB$ & \multicolumn{1}{|c|}{\phantom{\LARGE{X}}$\matrA$\phantom{\LARGE{X}}}\\
\hline
\end{tabular}
\end{center}

}
\end{column}
\end{columns}
\end{frame}

\begin{frame}[fragile]{La matrice tableau di un PL in forma canonica}
\centering
\begin{tikzpicture}
{\scriptsize
\node[
  row 1/.style=gray,
  row 2/.style=blue,
  column 1/.style=gray,
  column 2/.style=red,
  row 2 column 1/.style={gray},
  cell/.style=rectangle,
  text height=2ex,
  text width=1.5em,
  align=center,
  column  9/.style={text width=2em},
  column 10/.style={text width=2em},
  column 12/.style={text width=2em},
  column 14/.style={text width=2em},
  matrix of math nodes] (M)
{
&0&1&2&\cdots&k&\cdots&p&s_1&s_2&\cdots&s_h&\cdots&s_m\\
0 & -d & c_1 & c_2 & \cdots & c_k & \cdots & c_p & \alt<1-2>{c_{s_1}}{0} &  \alt<1-2>{c_{s_2}}{0} & \cdots &  \alt<1-2>{c_{s_h}}{0} & \cdots &  \alt<1-2>{c_{s_m}}{0}\\
1& b_1 &  a_{11} &  a_{12} &  \cdots &  a_{1k} &  \cdots &  a_{1{p}}&  1 &  0 & \cdots&0 & \cdots&0\\
2& b_2 &  a_{21} &  a_{22} &  \cdots &  a_{2k} &  \cdots &  a_{2{p}}  &  0 & 1 &  \cdots & 0 & \cdots&0\\
\vdots & \vdots &  \vdots &  \vdots &  \vdots &  \ddots &  \vdots &  \ddots &  \vdots &  \vdots &\ddots  & \vdots & \cdots&0\\
h& b_h &  a_{h1} &  a_{h2} &  \cdots & a_{hk} &  \cdots &  a_{h{p}} &  0 & 0 & \cdots & 1 & \cdots&0\\
\vdots& \vdots &  \vdots &  \vdots &  \vdots &  \ddots &  \vdots &  \ddots &\vdots & \vdots  & \vdots & \vdots & \ddots&0\\
m& b_m &  a_{m1} &  a_{m2} &  \cdots &  a_{mk} &  \cdots &  a_{m{p}}&  0 &  0 & \cdots&0& \cdots & 1\\
};
% separatore verticale
\draw[dashed] (M-2-3.north west) -- (M-8-3.south west);
% separatore orizzontale
\draw[dashed] (M-3-1.north east) -- (M-3-14.north east);
% riquadro
\draw(M-2-2.north west) -- (M-2-14.north east) -- (M-8-14.south east) -- (M-8-2.south west) -- cycle;
% A_s: 3--8: 9--14
\uncover<2->{
\fill[red!50,fill opacity=0.5,thick,draw=red!60!black]  (M-3-9.north west) rectangle (M-8-14.south east);
\node[red!40!black, below = of M-8-11.north east,draw=red!40,thick] (as) {$\alpha)\ \mathbf{A}_S = \mathbf{I}_m$};
\draw[->,red!60,thick] (as) to (M-5-11.east);
}
% c_s: 2: 9--12
\onslide<3->{
\fill[green!50,fill opacity=0.5,thick,draw=green!60!black]  (M-2-9.north west) rectangle (M-2-14.south east);
\node[green!40!black, above = of M-2-11.south east,draw=green!40,thick] (cs) {$\beta)\ \mathbf{c}_S = \mathbf{0}$};
\draw[->,green!60,thick] (cs) to (M-2-11.east);
}
% b: 3--8: 2
\uncover<4->{
\fill[blue!50,fill opacity=0.5,thick,draw=blue!60!black]  (M-3-2.north west) rectangle (M-8-2.south east);
\node[blue!40!black, below = of M-8-2.north,draw=blue!40,thick] (b) {$\gamma\ \mathbf{b} \geq \mathbf{0}$};
\draw[->,blue!60,thick] (b) to [out=180,in=180] (M-5-2);
}
}
\end{tikzpicture}
%sono fragile: lasciami uno spazio vuoto

\end{frame}

\section{Pivot}

\begin{frame}[allowframebreaks]{Operazione ``pivot''}

L'operazione di ``pivot'' in posizione $(h, k )$,
con $h$ e $k$ diversi da 0,
\`e un'operazione sulla matrice tableau definita
quando $a_{hk} \neq 0$ e consiste nei seguenti passi:

\begin{itemize}
\item si dividono tutti gli elementi della riga $h$ per $a_{hk}$;

\item si sottraggono opportuni multipli della nuova riga $h$
	ottenuta al passo precedente a tutte le altre righe in modo
	tale che tutti gli elementi della colonna $k$, ad eccezione
	di quello in posizione $(h, k )$, diventino nulli.
\end{itemize}

\framebreak

Consideriamo l'operazione solo sulle matrici in forma canonica e un elemento pivot non nullo in una colonna fuori  base.
Il significato dell'operazione di pivot \`e il seguente:

\begin{itemize}
\item si risolve l'equazione $h$ rispetto a $x_k$;

\item si usa l'equazione $h$ per eliminare la variabile
	$x_k$ dalle altre equazioni di vincolo e dalla funzione
	obiettivo.
\end{itemize}

L'operazione di ``pivot'' in posizione $(h, k )$, con $h$ e $k$
diversi da 0, corrisponde a far entrare in base la variabile
fuori base $x_k$ e a far uscire dalla base la variabile $x_{s_h}$
in posizione $h$-esima nella sequenza di base.

\begin{center}
\includegraphics{img/pivot_entra_esce}
\end{center}

\begin{itemize}
\item L'operazione di pivot corrisponde al calcolo algebrico di una
nuova soluzione di base

\item A seguito dell'operazione di pivot in $(h, k)$ la matrice tableau
e la sequenza degli indici di base cambiano

\item L'elemento di pivot $a_{hk}$  \`e indicato da una circonferenza.
La variabile entrante in base $x_k$ \`e indicata da una freccia
verso il basso posta al di sopra della colonna $k$; la variabile
uscente dalla base $x_{s_h}$ \`e indicata da una freccia verso
l'alto sopra la colonna $s_h$
\end{itemize}
\end{frame}

\begin{frame}[fragile]{}

\vspace*{-0.35cm}

\centering
\begin{tikzpicture}
{\tiny
\node[%
  align=center,
  text width=1.5em,
  text height=4ex,
  row 1/.style={gray, text height=1ex},
  row 2/.style={blue!50!gray, text height=1ex},
  row 3/.style={blue},
  column 1/.style={gray,text width=1em},
  column 2/.style={red!50!gray},
  % rhs
  column 3/.style={red,text width=6em},
  % x_1
  column 4/.style={text width=3.5em},
  % x_2
  column 5/.style={text width=3.5em},
  % ...
  column 6/.style={text width=1em},
  % x_k
  column 7/.style={text width=3.5em},  
  % ...
  column 8/.style={text width=1em},
  % x_s_h
  column 9/.style={text width=3.5em},
  % ...
  column 10/.style={text width=1em},
  % x_n
  column 11/.style={text width=6em},
  % spiegazione
  column 12/.style={text width=8em},
  row 3 column 1/.style={gray},
  row 3 column 2/.style={red!50!gray},
  row 3 column 3/.style={red},
  matrix of math nodes] (M)
{%
% Indice delle righe (M-1)
~&~&0&1&2&\cdots&\alt<1-3,5->{k}{~}&\cdots&\alt<1-3,5->{s_h}{~}&\cdots&n\\
% Intestazione delle colonne (M-2)
~ &~&  \mbox{rhs}&  x_1 &  x_2 &  \cdots & x_k &  \cdots & x_{s_h} &  \cdots &  x_n & ~\\
% Riga 0, fo e costi ridotti (M-3)
0 & z & \alt<1-5>{-d}{-d-\frac{c_k b_{h}}{a_{hk}}} & \alt<1-5>{c_1}{\overline{c_1}} & \alt<1-5>{c_2}{\overline{c_2}} & \cdots & \alt<1-5>{c_k}{0} & \cdots & \alt<1-5>{0}{-\frac{c_k}{a_{hk}}} & \cdots &  \alt<1-5>{c_n}{c_n-\frac{c_k a_{hn}}{a_{hk}}} & \alt<1-5>{}{{\mathbf{A}_0} \gets {\mathbf{A}_0} - \frac{c_k}{a_{hk}} {\mathbf{A}_h}}\\
1& \alt<1-5>{x_{s_1}}{x_{{s^\prime}1}} & \alt<1-5>{b_1}{b_1-\frac{a_{1k} b_{h}}{a_{hk}}} & \alt<1-5>{a_{11}}{\overline{a_{11}}} &  \alt<1-5>{a_{12}}{\overline{a_{12}}} &  \cdots &  \alt<1-5>{a_{1k}}{0} &  \cdots &  \alt<1-5>{0}{-\frac{a_{1k}}{a_{hk}}}&  \cdots &  \alt<1-5>{a_{1n}}{a_{1n}-\frac{a_{1k} a_{hn}}{a_{hk}}} & \alt<1-5>{}{{\mathbf{A}_1} \gets {\mathbf{A}_1} - \frac{a_{1k}}{a_{hk}} {\mathbf{A}_h}}\\
2& \alt<1-5>{x_{s_2}}{x_{{s^\prime}2}} & \alt<1-5>{b_2}{b_1-\frac{a_{2k} b_{h}}{a_{hk}}} & \alt<1-5>{a_{21}}{\overline{a_{21}}} &  \alt<1-5>{a_{22}}{\overline{a_{22}}} &  \cdots &  \alt<1-5>{a_{2k}}{0} &  \cdots &  \alt<1-5>{0}{-\frac{a_{2k}}{a_{hk}}} &  \cdots & \alt<1-5>{a_{2n}}{a_{2n}-\frac{a_{2k} a_{hn}}{a_{hk}}} & \alt<1-5>{}{{\mathbf{A}_2} \gets {\mathbf{A}_2} - \frac{a_{2k}}{a_{hk}} {\mathbf{A}_h}}\\
\vdots & \vdots &  \vdots &  \vdots &  \vdots &  \ddots &  \vdots &  \ddots &  \vdots &\ddots  & \vdots & \alt<1-5>{}{\vdots}\\
h& \alt<1-4>{x_{s_h}}{x_{{s^\prime}h}} & \alt<1-4>{b_h}{\frac{b_h}{a_{hk}}} &  \alt<1-4>{a_{h1}}{\frac{a_{h1}}{a_{hk}}} &  \alt<1-4>{a_{h2}}{\frac{a_{h2}}{a_{hk}}} & \cdots & \alt<1-4>{a_{hk}}{1} & \cdots &  \alt<1-4>{1}{\frac{1}{a_{hk}}} & \cdots & \alt<1-4>{a_{hn}}{\frac{a_{hn}}{a_{hk}}} & \alt<1-4>{}{{\mathbf{A}_h} \gets \frac{1}{a_{hk}} {\mathbf{A}_h}} \\
\vdots& \vdots &  \vdots &  \vdots &  \vdots &  \ddots &  \vdots &  \ddots &\vdots & \ddots&\vdots & \alt<1-5>{}{\vdots}\\
m& \alt<1-5>{x_{s_m}}{x_{{s^\prime}m}} & \alt<1-5>{b_m}{b_m-\frac{a_{mk} b_{h}}{a_{hk}}} & \alt<1-5>{a_{m1}}{\overline{a_{m1}}} & \alt<1-5>{a_{m2}}{\overline{a_{m2}}} & \cdots & \alt<1-5>{a_{mk}}{0} &  \cdots &  \alt<1-5>{0}{-\frac{a_{mk}}{a_{hk}}} & \cdots& \alt<1-5>{a_{mn}}{a_{mn}-\frac{a_{mk}{a_{hn}}}{a_{hk}}} & \alt<1-5>{}{{\mathbf{A}_m} \gets {\mathbf{A}_m} - \frac{a_{mk}}{a_{hk}} {\mathbf{A}_h}}\\
};

% riquadro
\draw(M-3-3.north west) -- (M-3-11.north east) -- (M-9-11.south east) -- (M-9-3.south west) -- cycle;

% separatore orizzontale
\draw[dashed] (M-4-3.north west) -- (M-4-11.north east);

% separatore verticale
\draw[dashed] (M-3-4.north west) -- (M-9-4.south west);

\onslide<2-3>{
% evidenzia chi entra
\fill[yellow!50,fill opacity=0.25,thick,draw=green!60!black]  (M-2-7.north west) rectangle (M-9-7.south east);
\node[green!40!black,draw=green!40,thick] at (M-1-7.north) {entra $\mathbf{x}_k$};

% e il pivot
\fill[yellow!50,fill opacity=0.25,thick,draw=green!60!black]  (M-7-2.north west) rectangle (M-7-11.south east);
}

\onslide<3>{
\fill[magenta!50,fill opacity=0.25,thick,draw=green!60!black]  (M-2-9.north west) rectangle (M-9-9.south east);
\node[green!40!black,draw=green!40,thick] at (M-1-9.north) {esce $\mathbf{x}_{s_h}$};
}

% entra x_k
\draw<4>[->,thick,red!50!blue] (M-1-7.north) to (M-2-7.north);
% esce x_{s_h}
\draw<4>[->,thick,red!50!blue] (M-2-9.north) to (M-1-9.north);

\draw<4>[thick,draw=blue,fill=yellow!50,opacity=0.3] (M-7-7.base) circle (1.5em);

\node[below = of M-8-7]
{
\only<1> {Tableau iniziale}
\only<2> {Vengono date la colonna $k$ e la riga $h$}
\only<3> {\`E determinata la colonna uscente dalla base $s_h$}
\only<4> {Indicazione dell'operazione di pivot in $(h, k)$ e delle variabili entranti ed uscenti dalla base}
\only<5> {Si inizia l'operazione dividendo la riga $h$ per il valore dell'elemento $a_{hk}$}
\only<6> {Ad ogni riga $i \in \{0, 1, \ldots m\} \setminus h$ si sottrae la riga $h$ moltiplicata per $\frac{a_{ik}}{a_{hk}}$}
};

}
\end{tikzpicture}
%sono fragile: lasciami uno spazio vuoto

\end{frame}

\begin{frame}[allowframebreaks]{Pivot: dalla base $S$ a quella adiacente $S^\prime$ }

Consideriamo l'intera matrice tableau $\matr{M}$ con gli indici di
riga che partono da 0, riga della funzione obiettivo e gli indici
di colonna che partono da 0, colonna dei termini noti.

Sia $S$ la sequenza degli indici di base prima dell'operazione di
pivot e $S^\prime$ la sequenza a seguito dell'operazione di pivot
in $(h, k)$.

$$
\begin{cases}
\matr{M}^{S^\prime}_i \gets \frac{1}{m_{hk}^S} \matr{M}^{S}_h & i = h\\
\matr{M}^{S^\prime}_i \gets \frac{m_{ik}^S}{m_{hk}^S} \matr{M}^{S}_h & i \neq h\\
\end{cases}
$$

$$
\begin{cases}
m^{S^\prime}_{ij} \gets \frac{m_{ij}^S}{m_{hk}^S} & i = h,\ j\in\{0,1,\ldots,n\}\\
m^{S^\prime}_{ij} \gets m^{S}_{ij}-\frac{m_{ik}^S}{m_{hk}^S} m_{hj}^S & i \in \{0,1,\ldots,m\} \setminus h,\ j\in\{0,1,\ldots,n\}\\
\end{cases}
$$

\begin{itemize}
\item Il valore della funzione obiettivo si modifica secondo
$z^{S^\prime} \gets z^S + \frac{c_k}{a_{hk}} b_h$

\item Il valore delle variabili si modifica secondo
$
x_{j}^{S^\prime} \gets
\begin{cases}
0                  & j \in \overline{S} \cup \{s_h\} \setminus \{k\}\\
\frac{b_h}{a_{hk}} & j \in \{k\}\\
x_{j}^S-a_{ik} x_{s_h}^{S^\prime} & j \in S \setminus \{s_h\}, i = \arg_t \{s_t = j\}
\end{cases}
$

\item La variabile $x_k$ entra in base:
$0 \to \frac{b_h}{a_{hk}}$

\item La variabile $x_{s_h}$ esce dalla base: $b_{s_h} \to 0$
\end{itemize}  
 
\end{frame}

\begin{frame}[allowframebreaks]{Riassunto}

\begin{itemize}
\item La forma canonica permette la lettura immediata di una
soluzione di base.

\item Se la forma canonica \`e forte allora la soluzione \`e una
soluzione di base ammissibile.

\item L'operazione di ``pivot'' applicato alla matrice tableau
permettono di risolvere iterativamente un sistema di
equazioni lineari calcolando al contempo vincoli e
funzione obiettivo.

\item La forma canonica permette di trattare la funzione
obiettivo in funzione delle sole variabili non di base.
In questo modo si ha

\begin{itemize}
\item un facile computo del valore scalare z;

\item un metodo semplice per verificare l'ottimalit\`a
della soluzione o, in caso contrario, le variabili candidate
ad entrare in base in una soluzione non peggiorante.
\end{itemize}

\item La forma canonica permette di applicare il metodo di
Gauss-Jordan e di calcolare in modo iterativo l'inversa
di una matrice.
\end{itemize}
\end{frame}

\end{document}
