\documentclass{beamer}
\def\presentationtype{0}
\input{../template/savoia_benincasa}

\begin{document}

\generatitolo

\section{Il problema della dieta}

\begin{frame}%[allowframebreaks]
{Il problema della dieta -- esempio}

\begin{block}{}
Un'azienda produce mangimi per animali a partire da quattro prodotti grezzi (orzo, avena, sesamo, arachidi). Le proteine e i grassi contenuti per unità nei materiali grezzi, insieme al costo unitario, sono riportati in tabella.%~\ref{tab:istanza}

Si vuole determinare la composizione di una mistura alimentare di minimo costo soddisfacente le esigenze nutritive. 
\end{block}
\end{frame}

\begin{frame}%[allowframebreaks]
{Il problema della dieta -- istanza di esempio}

\begin{table}
\begin{tabular}{c|cccc|c}\toprule
&\bf Orzo&\bf Avena&\bf Sesamo&\bf Arachidi&\parbox{1.5cm}{{\bf \begin{center}Requisiti nutritivi\end{center}}}\\\midrule
\bf Proteine & 12& 12 & 40 & 60 & 20 \\
\bf Grassi   &  2&  6 & 12 &  2 &  5 \\\midrule
\bf Costi    & 24& 30 & 40 & 50 &    \\\bottomrule
\end{tabular}
\caption{Composizione degli alimenti, costo e requisiti nutritivi}
\label{tab:istanza}
\end{table}

\end{frame}

\section{Costruzione del modello}

\begin{frame}{Insiemi e parametri}
	\begin{block}{Insiemi}
		\begin{description}
			\item[Nutrienti] insieme dei nutrienti
			\item[Alimenti] insieme degli alimenti
%			\item[Apporti] insieme $\rm Nutrienti \times \rm Alimenti$
		\end{description}
	\end{block}
	\begin{block}{Parametri}
		\begin{description}
			\item[costo] costo di un unit\`a di un alimento
			\item[req\_nutrizionale] quantit\`a minima di nutrienti necessaria
			\item[apporto] quantit\`a di nutriente apportata da
			 un'unit\`a di alimento
		\end{description}
	\end{block}
\end{frame}

\begin{frame}{Variabili decisionali}
Si deve determinare la composizione della mistura alimentare, per cui definiamo quattro variabili che rappresentano le unità di prodotto grezzo che saranno impiegate per la composizione del mangime.
\begin{block}{Variabili}
	\begin{description}
		\item[$x_1$]		unità di orzo,
		\item[$x_2$]		unità di avena,
		\item[$x_3$]		unità di sesamo,
		\item[$x_4$]		unità di arachidi.
	\end{description}
\end{block}

  Le variabili sono continue e non negative.

\end{frame}

\begin{frame}{Funzione obiettivo}
Si deve determinare la composizione della mistura che minimizzi il costo totale. Il costo, espresso in funzione delle variabili decisionale \`e:

$$z=24 x_1 + 30 x_2 + 40 x_3 + 50 x_4$$

In generale:

$$  \sum_{j \in {\rm \textcolor{col_set}{Alimenti}}} {\rm \textcolor{col_par}{costo}}_j \, x_j$$

Il costo è espresso da una funzione lineare: rispetta le ipotesi di proporzionalit\`a, additivit\`a e divisibilit\`a.
\end{frame}

\begin{frame}{Vincoli}

Il mangime prodotto deve soddisfare le esigenze nutritive apportando un quantitativo minimo di nutrienti

la quantit\`a di nutrienti non pu\`o essere inferiore al requisito nutrizionale corrispondente

$$  \sum_{j \in {\rm \textcolor{col_set}{Alimenti}}} {\rm \textcolor{col_par}{apporto}}_{ij}  \, x_j \geq {\rm \textcolor{col_par}{req\_nutrizionale}} \quad i \in {\rm \textcolor{col_set}{Nutrienti}}$$

Inoltre valgono i vincoli di non negativit\`a delle variabili
\end{frame}

\begin{frame}{Dieta ottima: modello basato su insiemi}
\begin{columns}%{}

\column{.49\textwidth}

\begin{minipage}{\textwidth}
\footnotesize{
\begin{tabular}{lp{0.85\textwidth}}
\color{col_set}$N$		&Insieme dei nutrienti\\
\color{col_set}$A$		&Insieme degli alimenti\\
~\\
$\textcolor{col_par}{a}_{\textcolor{col_idx}{ij}}$	&Apporto del nutriente $i\in N$ da un alimento $j\in A$\\
$\textcolor{col_par}{b}_{\textcolor{col_idx}{i}}$	&Requisito del nutriente $i\in N$\\
$\textcolor{col_par}{c}_{\textcolor{col_idx}{j}}$	&Costo unitario alimento $j\in A$\\
~\\
$\textcolor{col_var}{x}_{\textcolor{col_idx}{j}}$	&Quantit\`a di alimento \mbox{$j\in A$}\\
~\\
\end{tabular}
\begin{tabular}{ll}
\hspace*{0.5\textwidth}&\\
\textcolor{col_set}{\rule{1 em}{1 em} Insiemi} &
\textcolor{col_idx}{\rule{1 em}{1 em} Indici} \\
\textcolor{col_par}{\rule{1 em}{1 em} Parametri} &
\textcolor{col_var}{\rule{1 em}{1 em} Variabili}
\end{tabular}
}
\end{minipage}

\column{.51\textwidth}

\def\funzioneobiettivo{{\displaystyle \sum_{\textcolor{col_idx}{j} \in \textcolor{col_set}{A}}}
	\textcolor{col_par}{c}_{\textcolor{col_idx}{j}}\,\textcolor{col_var}{x}_{\textcolor{col_idx}{j}}}
\def\jinP{\textcolor{col_idx}{j} \in \textcolor{col_set}{A}}
\def\iinR{\textcolor{col_idx}{i} \in \textcolor{col_set}{N}}


\fbox{
{\footnotesize
$\begin{array}{rll}
\min z= & \funzioneobiettivo\\% Funzione obiettivo
	       & {\displaystyle \sum_{\textcolor{col_idx}{j} \in \textcolor{col_set}{A}}}
	       \textcolor{col_par}{a}_{\textcolor{col_idx}{ij}} \,
	       \textcolor{col_var}{x}_{\textcolor{col_idx}{j}} \geq \textcolor{col_par}{b}_{\textcolor{col_idx}{i}} & \iinR\\
  & \textcolor{col_var}{x}_{\textcolor{col_idx}{j}}\geq 0 &  \jinP
\end{array}$
}
}
\end{columns}%{}
\end{frame}

\begin{frame}[fragile]{Modello in LINGO -- modello}
\lstinputlisting{allegati/dieta_mod.lng}
\end{frame}


\begin{frame}[fragile]{Modello in LINGO -- dati}
\lstinputlisting{allegati/dieta_dat.ldt}
\end{frame}


\begin{frame}{Modello in GNU MathProg -- modello}
\lstinputlisting{allegati/dieta.mod}
\end{frame}

\begin{frame}[fragile]{Modello in GNU MathProg -- dati}
\lstinputlisting{allegati/dieta.dat}
\end{frame}

\end{document}

minoxidyl