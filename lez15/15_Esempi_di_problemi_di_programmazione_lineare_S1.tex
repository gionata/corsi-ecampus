\documentclass{beamer}
\def\presentationtype{1}
\input{../template/savoia_benincasa}

\begin{document}

\generatitolo

\section{Un problema d'investimento}

\begin{frame}%[allowframebreaks]
{Un problema d'investimento -- esempio}

\begin{block}{}
{\small   Un'azienda deve decidere una politica di investimento al fine di massimizzare il ritorno atteso stimato stimato in base alle condizioni interne e di mercato. Le opzioni di investimento sono di due tipi. L'investimento di tipo 1 ha un ritorno atteso del 15\% ed \`e poco rischioso mentre l'investimento di tipo 2 ha un ritorno atteso del 25\% e presenta una maggiore rischiosit\`a.
Il management ha stabilito che la quantit\`a massima di denaro che si pu\`o investire (budget)  \`e di 100 mila euro, che l'investimento di tipo 1 deve essere almeno $\frac{1}{4}$ del budget e che l'investimento di tipo 2 non pu\`o essere superiore al doppio dell'investimento di tipo 1.

  Determinare gli investimenti cui corrisponde un ritorno atteso massimo nel rispetto delle restrizioni imposte dal management.}
\end{block}
\end{frame}

\begin{frame}%[allowframebreaks]
{Un problema d'investimento -- istanza di esempio}

\begin{table}
\begin{tabular}{c|cc|c}\toprule
&\bf Minimo&\bf Massimo&\parbox{1.5cm}{{\bf \begin{center}Ritorno atteso\end{center}}}\\\midrule
\bf Tipo 1 & budget $/ 4$  &        --     & 15\% \\
\bf Tipo 2 &    --  &  2 Tipo 1     & 25\% \\\midrule
\bf Budget &\multicolumn{2}{c|}{100}&    \\\bottomrule
\end{tabular}
\caption{Parametri del problema d'investimento}
\label{tab:istanza}
\end{table}

\end{frame}

\section{Costruzione del modello}

\begin{frame}{Insiemi e parametri}
	\begin{block}{Insiemi}
		\begin{description}
			\item[Investimenti] insieme degli investimenti
%			\item[Relazioni] relazione tra gli investimenti, sottoinsieme del prodotto cartesiano dell'insieme investimenti per s\`e stesso
%			\item[Apporti] insieme $\rm Nutrienti \times \rm Alimenti$
		\end{description}
	\end{block}
	\begin{block}{Parametri}
		\begin{description}
			\item[ritorno] ritorno atteso di un investimento
			\item[minimo] importo minimo destinato all'investimento
			\item[proporzione] relazione tra due investimenti
			\item[budget] importo totale da investire
		\end{description}
	\end{block}
\end{frame}

\begin{frame}{Variabili decisionali}
Si deve determinare l'importo da destinare a ciascun investimento,
per cui definiamo una variabile decisionale per ogni investimento che
rappresenta il denaro da destinare all'investimento stesso.
\begin{block}{Variabili}
	\begin{description}
		\item[$x_1$]		denaro destinato all'investimento di tipo 1,
		\item[$x_2$]		denaro destinato all'investimento di tipo 2.
	\end{description}
\end{block}

  Le variabili sono continue e non negative.

\end{frame}

\begin{frame}{Funzione obiettivo}
Si deve determinare la ripartizione del budget per gli investimenti in modo
che il ritorno atteso totale sia massimo. Il ritorno atteso degli
investimenti, espresso in funzione delle variabili decisionale \`e:

$$z=0.15\ x_1 + 0.25\ x_2$$

In generale:

$$  \sum_{j \in {\rm \textcolor{col_set}{Investimenti}}} {\rm \textcolor{col_par}{ritorno}}_j \, x_j$$

Il ritorno è espresso da una funzione lineare: rispetta le ipotesi di proporzionalit\`a, additivit\`a e divisibilit\`a.
\end{frame}

\begin{frame}{Vincoli}

L'importo degli investimenti non pu\`o essere superiore al budget

$$  \sum_{j \in {\rm \textcolor{col_set}{Investimenti}}} x_j
\leq {\rm \textcolor{col_par}{budget}}$$


L'importo di ogni investimento deve $\geq$ al minimo
stabilito

$$  \sum_{j \in {\rm \textcolor{col_set}{Investimenti}}} x_j
\geq {\rm \textcolor{col_par}{minino}} \quad i \in {\rm \textcolor{col_set}{Investimenti}}$$

Le proporzioni fra gli investimenti devono sottostare al vincolo

$$ x_i \geq {\rm \textcolor{col_par}{proporzione}}_{ij} x_j\quad i,\ j \in {\rm \textcolor{col_set}{Investimenti}},\ i \neq j$$

Inoltre valgono i vincoli di non negativit\`a delle variabili
\end{frame}

\begin{frame}{Investimento ottimo: modello basato su insiemi}
\begin{columns}%{}

\column{.49\textwidth}

\begin{minipage}{\textwidth}
\footnotesize{
\begin{tabular}{lp{0.85\textwidth}}
\color{col_set}$I$		&Insieme degli investimenti\\
~\\
$\textcolor{col_par}{b}$ &Budget complessivo\\
$\textcolor{col_par}{r}_{\textcolor{col_idx}{j}}$	&Ritorno atteso dell'investimento $j\in I$\\
$\textcolor{col_par}{m}_{\textcolor{col_idx}{j}}$	&Importo minimo destinato all'investimento $j\in I$\\
$\textcolor{col_par}{p}_{\textcolor{col_idx}{ij}}$	&Proporzione tra gli
investimenti $i\in I$ e $j\in I$\\
~\\
$\textcolor{col_var}{x}_{\textcolor{col_idx}{j}}$	&Importo 
dell'investimento \mbox{$j\in I$}\\
~\\
\end{tabular}
\begin{tabular}{ll}
\hspace*{0.5\textwidth}&\\
\textcolor{col_set}{\rule{1 em}{1 em} Insiemi} &
\textcolor{col_idx}{\rule{1 em}{1 em} Indici} \\
\textcolor{col_par}{\rule{1 em}{1 em} Parametri} &
\textcolor{col_var}{\rule{1 em}{1 em} Variabili}
\end{tabular}
}
\end{minipage}

\column{.51\textwidth}

\def\funzioneobiettivo{{\displaystyle \sum_{\textcolor{col_idx}{j} \in \textcolor{col_set}{I}}}
	\textcolor{col_par}{r}_{\textcolor{col_idx}{j}}\,\textcolor{col_var}{x}_{\textcolor{col_idx}{j}}}
\def\jinI{\textcolor{col_idx}{j} \in \textcolor{col_set}{I}}
\def\iinI{\textcolor{col_idx}{i} \in \textcolor{col_set}{I}}

\fbox{
{\footnotesize
$\begin{array}{rll}
\max z= & \funzioneobiettivo\\% Funzione obiettivo

	       & {\displaystyle \sum_{\textcolor{col_idx}{j} \in \textcolor{col_set}{I}}}
	       \textcolor{col_var}{x}_{\textcolor{col_idx}{j}} \leq \textcolor{col_par}{b}\\% budget

	       & \textcolor{col_var}{x}_{\textcolor{col_idx}{j}} \geq \textcolor{col_par}{m}_{\textcolor{col_idx}{i}} & \jinI\\ % minimo
	       
	       & & \iinI,\\
	       & 
	       \textcolor{col_var}{x}_{\textcolor{col_idx}{i}} \geq \textcolor{col_par}{p}_{\textcolor{col_idx}{ij}}\, \textcolor{col_var}{x}_{\textcolor{col_idx}{j}}& \jinI,\\
	       & &  \textcolor{col_idx}{i} \neq \textcolor{col_idx}{j}\\

	       
  & \textcolor{col_var}{x}_{\textcolor{col_idx}{j}}\geq 0 &  \jinI
\end{array}$
}
}
\end{columns}%{}
\end{frame}

\begin{frame}[fragile]{Modello in LINGO -- modello}
\lstinputlisting{allegati/investimenti.lng}
\end{frame}

\begin{frame}[fragile]{Modello in LINGO -- dati}
\lstinputlisting{allegati/investimenti.ldt}
\end{frame}

\begin{frame}{Modello in GNU MathProg -- modello}
\lstinputlisting{allegati/investimenti.mod}
\end{frame}

\begin{frame}[fragile]{Modello in GNU MathProg -- dati}
\lstinputlisting{allegati/investimenti.dat}
\end{frame}

\end{document}
