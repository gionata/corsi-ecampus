\documentclass{beamer}
\def\presentationtype{0}
\input{../template/savoia_benincasa}

\begin{document}

\generatitolo

\section{Costruzione del modello: insiemi e parametri}

\begin{frame}{Stuttura dei problemi decisionali}
I modelli decisionali hanno tutti la stessa struttura:
\begin{itemize}
\item insiemi, sui quali scorrono degli indici;
\item dati numerici (di valore noto), scalari o vettori indicizzati su un insieme o matrici a due o pi\`u dimensioni (indicizzate su due o pi\`u insiemi);
\item variabili, scalari, o vettori o matrici

\item una funzione obiettivo $z$, che dipende dai dati e dalle variabili e pu\`o essere minimizzata o massimizzata;
\item un certo numero di vincoli, di uguaglianza (equazioni) o disuguaglianza
(disequazioni), singoli oppure raccolti in famiglie indicizzate su uno o pi\`u insiemi.
\end{itemize}
\end{frame}

\begin{frame}{Mix ottimo di produzione: modello basato su insiemi}
\definecolor{col_set}{RGB}{128,0,0}
\definecolor{col_par}{RGB}{91,0,91}
\definecolor{col_idx}{RGB}{0,91,91}
\definecolor{col_var}{RGB}{0,0,128}
\begin{columns}%{}

\column{.49\textwidth}

\begin{minipage}{\textwidth}
\footnotesize{
\begin{tabular}{lp{0.85\textwidth}}
\color{col_set}$P$		&Insieme dei prodotti\\
\color{col_set}$R$		&Insieme delle risorse\\
~\\
$\textcolor{col_par}{a}_{\textcolor{col_idx}{ij}}$	&Assorbimento della risorsa $i\in R$ da un prodotto $j\in P$\\
$\textcolor{col_par}{b}_{\textcolor{col_idx}{i}}$	&Disponibilit\`a della risorsa $i\in R$\\
$\textcolor{col_par}{c}_{\textcolor{col_idx}{j}}$	&Profitto unitario prodott $j\in P$\\
~\\
$\textcolor{col_var}{x}_{\textcolor{col_idx}{j}}$	&Livello di produzione prodotto \mbox{$j\in P$}\\
~\\
\end{tabular}
\begin{tabular}{ll}
\hspace*{0.5\textwidth}&\\
\textcolor{col_set}{\rule{1 em}{1 em} Insiemi} &
\textcolor{col_idx}{\rule{1 em}{1 em} Indici} \\
\textcolor{col_par}{\rule{1 em}{1 em} Parametri} &
\textcolor{col_var}{\rule{1 em}{1 em} Variabili}
\end{tabular}
}
\end{minipage}

\column{.51\textwidth}

\def\funzioneobiettivo{{\displaystyle \sum_{\textcolor{col_idx}{j} \in \textcolor{col_set}{P}}}
	\textcolor{col_par}{c}_{\textcolor{col_idx}{j}}\,\textcolor{col_var}{x}_{\textcolor{col_idx}{j}}}
\def\jinP{\textcolor{col_idx}{j} \in \textcolor{col_set}{P}}
\def\iinR{\textcolor{col_idx}{i} \in \textcolor{col_set}{R}}


\fbox{
{\footnotesize
$\begin{array}{rll}
\max z= & \funzioneobiettivo\\% Funzione obiettivo
	       & {\displaystyle \sum_{\textcolor{col_idx}{j} \in \textcolor{col_set}{P}}}
	       \textcolor{col_par}{a}_{\textcolor{col_idx}{ij}} \,
	       \textcolor{col_var}{x}_{\textcolor{col_idx}{j}} \leq \textcolor{col_par}{b}_{\textcolor{col_idx}{i}} & \iinR\\
  & \textcolor{col_var}{x}_{\textcolor{col_idx}{j}}\geq 0 &  \jinP, \, \iinR
\end{array}$
}
}
\end{columns}%{}
\end{frame}


\section{Insiemi, indici, parametri e iteratori}

\begin{frame}[allowframebreaks,fragile]{Insiemi}

\framebox{
\parbox[c][24pt]{0.85\textwidth}{
\hspace{6pt} {\tt set} {\it name} {\it alias} {\it domain} {\tt,}
{\it attrib} {\tt,} \dots {\tt,} {\it attrib} {\tt;}
}}

\begin{description}

\item[{\it name}] \`e il nome dell'insieme;

\item[{\it alias}] opzionale, altro nome dell'insieme;

\item[{\it domain}] opzionale, specifica meglio il dominio;

\item[{\it attrib}, \dots, {\it attrib}] attributi opzionali.
\end{description}

\framebreak

{\small Attributi opzionali}

\vspace*{-8pt}

{\footnotesize \begin{description}
\item[{\tt dimen} $n$]\hspace*{0pt}\\
la cardinalit\`a $n$ dell'insieme;
\item[{\tt within} {\it expression}]\hspace*{0pt}\\
un insieme che include l'insieme che si sta definendo;
\item[{\tt:=} {\it expression}]\hspace*{0pt}\\
i membri dell'insieme (enumerazione o propriet\`a);
\item[{\tt default} {\it expression}]\hspace*{0pt}\\
gli elementi dell'insieme se
non specificati nella sezione dati.
\end{description}}

\begin{block}{Insiemi ``Prodotti'' e ``Risorse''}
\begin{lstlisting}
set Prodotti;
set Risorse;
\end{lstlisting}
\end{block}
\end{frame}

\begin{frame}[allowframebreaks,fragile]{Indici}
Due forme delle espressioni di indicizzazione, che permettono di scorrere
gli indici:
$$
{
\begin{array}{l}
\mbox{{\tt\{} {\it entry}$_1${\tt,} {\it entry}$_2${\tt,} \dots{\tt,}
{\it entry}$_m$ {\tt\}}}\\
\mbox{{\tt\{} {\it entry}$_1${\tt,} {\it entry}$_2${\tt,} \dots{\tt,}
{\it entry}$_m$ {\tt:} {\it predicate} {\tt\}}}\\
\end{array}
}
$$
dove {\it entry}{$_1$}, {\it entry}{$_2$}, \dots, {\it entry}{$_m$}
sono elementi di indicizzazione e {\it predicate} \`e un'espressione
logica opzionale che deve essere verificata dagli indici.

Ogni {\it indexing entry} \`e in una delle forme:
$$
{
\begin{array}{l}
\mbox{$i$ {\tt in} $S$}\\
\mbox{{\tt(}$i_1${\tt,} $i_2${\tt,} \dots{\tt,}$i_n${\tt)} {\tt in}
$S$}\\
\mbox{$S$}\\
\end{array}
}
$$
dove $i_1$, $i_2$, \dots, $i_n$ sono indici, $S$ \`e un insieme base.

Sono usate, ad esempio, per indicizzare i parametri di un modello e di un iteratore

\begin{lstlisting}
{j in Prodotti};
{j in Prodotti : j <> prod_1};
\end{lstlisting}
\end{frame}

\begin{frame}[allowframebreaks,fragile]{Parametri}

\framebox{
\parbox[c][24pt]{0.85\textwidth}{
\hspace{6pt} {\tt param} {\it name} {\it alias} {\it domain} {\tt,}
{\it attrib} {\tt,} \dots {\tt,} {\it attrib} {\tt;}
}}

\begin{description}

\item[{\it name}] \`e il nome del parametro;

\item[{\it alias}] opzionale, altro nome del parametro;

\item[{\it domain}] opzionale, specifica meglio il dominio;

\item[{\it attrib}, \dots, {\it attrib}] attributi opzionali.
\end{description}

\framebreak

{\small Attributi opzionali}

\vspace*{-8pt}

{\footnotesize \begin{description}
\item[{\tt integer}]\hspace*{0pt}\\
il parametro \`e un numero intero;
\item[{\tt binary}]\hspace*{0pt}\\
il parametro \`e binario, 0/1;
\item[{\tt symbolic}]\hspace*{0pt}\\
il parametro \`e simbolico;
\item[{\it relation expression}]\hspace*{0pt}\\
(dove {\it relation} \`e un simbolo tra: {\tt<}, {\tt<=}, {\tt=}, {\tt==},
{\tt>=}, {\tt>}, {\tt<>}, {\tt!=})\\
una condizione che vincola il parametro o i suoi membri a soddisfare
quella condizione;
\item[{\tt in} {\it expression}]\hspace*{0pt}\\
un insieme che vincola il parametro o i suoi membri ad appartenere a tale
insieme;
\item[{\tt:=} {\it expression}]\hspace*{0pt}\\
un valore assegnato al parametro o ai sui membri;
\item[{\tt default} {\it expression}]\hspace*{0pt}\\
un valore assegnato al parametro o ai suoi membri se
non specificati nella sezione dati.
\end{description}}

Per riferirsi a una componete di un parametro definito come
matrice si usa il costrutto:
$$
\mbox{{\it name}{\tt[}$i_1${\tt,} $i_2${\tt,} \dots{\tt,} $i_n${\tt]}}
$$
dove {\it name} \`e il nome del parametro e $i_1$, $i_2$,
\dots, $i_n$ sono indici.

\begin{block}{Parametri ``profitto'', ``disponibilita'' e ``assorbimento'''}
\begin{lstlisting}
param profitto {j in Prodotti};
param disponibilita {i in Risorse};
param assorbimento {i in Risorse, j in Prodotti};
\end{lstlisting}
\end{block}
\end{frame}

\begin{frame}{Iteratori e quantificatori}
\noindent\hfil
\begin{tabular}{@{}lll@{}}
{\tt sum}&somma&$\displaystyle\sum_{(i_1,\dots,i_n)\in\Delta}
f(i_1,\dots,i_n)$\\
{\tt prod}&prodotto&$\displaystyle\prod_{(i_1,\dots,i_n)\in\Delta}
f(i_1,\dots,i_n)$\\
{\tt min}&minimo&$\displaystyle\min_{(i_1,\dots,i_n)\in\Delta}
f(i_1,\dots,i_n)$\\
{\tt max}&massimo&$\displaystyle\max_{(i_1,\dots,i_n)\in\Delta}
f(i_1,\dots,i_n)$\\
{\tt forall}&$\forall$&$\displaystyle
\forall(i_1,\dots,i_n)\in\Delta[f(i_1,\dots,i_n)],$\\
{\tt exists}&$\exists$&$\displaystyle
\exists(i_1,\dots,i_n)\in\Delta[f(i_1,\dots,i_n)],$\\
\end{tabular}

\noindent \\~\\ dove $i_1$, \dots, $i_n$ sono gli indici degli insiemi,
$\Delta$ \`e il dominio,
$f(i_1,\dots,i_n)$ \`e l'integrando, un espressione numerica 
il cui valore dipende dagli indici.
\end{frame}

\section{Descrizione di un modello}

\begin{frame}{Sezioni}
La dichiarazione del programma lineare \`e divisa nelle sezioni modello
(\emph{model section}) e dati (\emph{data section}).

\begin{description}
\item [{Model section}] descrive il modello matematico mediante
le dichiarazioni degli oggetti del modello; essi che sono comuni a
tutti i problemi definiti sullo stesso modello.
\item [{Data section}] opzionale, contiene i dati specifici di una
particolare istanza del problema.
\end{description}

Le sezioni ``modello'' e ``dati'' possono essere memorizzate in un
unico file o in due file separati.

La seconda opzione permettedi avere un numero arbitrario di sezioni
dati differenti da usare con la stessa sezione dati.
\end{frame}

\section{Implementazione del modello del MIX OTTIMO}

\begin{frame}{Formulazione GNU Mathprog del mix ottimo -- Model}
\lstinputlisting{allegati/modello_algebrico.mod}
\end{frame}

\begin{frame}{Formulazione GNU Mathprog del mix ottimo -- Data}
\lstinputlisting{allegati/dati_mix_ottimo.dat}
\end{frame}

\section{Per approfondire}

\begin{frame}{Per approfondire}

\beamertemplatebookbibitems
\begin{thebibliography}{References}
\bibitem{Cordone}R. Cordone.\newblock
\textit{Il linguaggio GNU MathProg con esempi commentati e risolti}.\newblock
2010. Disponibile su internet.

\bibitem{gmplManual} A. Makhorin.\newblock
\textit{Modeling Language GNU MathProg -- Language Reference}
\newblock 2013. Distribuito con GLPK.

\bibitem{amplbook} R. Fourer, D. M. Gay, B. W. Kernighan.%
\newblock \emph{A Modeling Language for Mathematical Programming.}%
\newblock Management Science, 36, 1990. Disponibile su internet.
\end{thebibliography}
\end{frame}

\end{document}
