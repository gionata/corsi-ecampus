\documentclass{beamer}
\def\presentationtype{0}
\input{../template/savoia_benincasa}

\begin{document}

\generatitolo

\section{Problemi di decisione}

\begin{frame}{Problema}
\begin{block}{Problema}
Un problema $\mathcal{P}$ \`e una domanda espressa in termini generali, la cui risposta
dipende da un certo numero di parametri e variabili
\end{block}
Un problema viene usualmente definito per mezzo di:
\begin{itemize}
\item una descrizione dei suoi parametri, in generale lasciati indeterminati
\item una descrizione delle propriet\`a che devono caratterizzare la risposta
o soluzione desiderata
\end{itemize} 
\end{frame}

\begin{frame}{Istanza di un problema}
\begin{block}{Istanza di un problema}
Un'istanza $\mathcal{I}$ di un dato problema $\mathcal{P}$ \`e quella particolare domanda
che si ottiene specificando particolari valori per tutti i parametri di $\mathcal{P}$
\end{block}
\end{frame}

\begin{frame}{Soluzioni realizzabili}

\begin{block}{Soluzioni realizzabili}
Molto spesso un problema viene definito fornendo l'insieme $\mathcal{F}$
delle possibili soluzioni. Di tale insieme, detto insieme realizzabile,
viene in generale data la struttura con i parametri da cui essa dipende;
i suoi elementi vengono detti \alert{soluzioni realizzabili}
\end{block}

Frequentemente l'insieme $\mathcal{F}$ viene specificato indicando un insieme
``base'' $\mathcal{F}^\prime$ tale che $\mathcal{F}\subseteq \mathcal{F}^\prime$, ed ulteriori condizioni
(vincoli) che gli elementi di $\mathcal{F}$ devono soddisfare. In questo
caso si parla spesso degli elementi di $\mathcal{F}^\prime\setminus \mathcal{F}$ come di \alert{soluzioni
non realizzabili}
\end{frame}

\begin{frame}{Problema di ottimizzazione}
Un \alert{problema di ottimizzazione} \`e un problema in cui sull'insieme
realizzabile $\mathcal{F}$ viene definita una funzione, detta \alert{funzione
obiettivo}
\[
c:\mathcal{F}\rightarrow\mathbb{R}
\]
che fornisce il costo o il beneficio associato ad ogni soluzione;
la \alert{soluzione ottima} del problema \`e un elemento di $\mathcal{F}$
che rende minima (oppure massima) la funzione obiettivo

Questo tipo di problema pu\`o essere matematicamente scritto come
\[
(P)\qquad\min\{c(\vec{x}):\vec{x}\in \mathcal{F}\}
\]
\end{frame}

\begin{frame}{Problema di ottimizzazione}
 Dato un problema $\mathcal{P}$, chiamiamo $z(\mathcal{P})$ il valore
ottimo della funzione obiettivo. Allora, una soluzione realizzabile
$\vec{x}^{*}\in\mathcal{F}$ tale che $c(\vec{x}^{*})=z(\mathcal{P})$ \`e detta soluzione ottima per
$\mathcal{P}$; si tratta cio\`e di un vettore $\vec{x}^{*}\in \mathcal{F}$ tale che
\[
c(\vec{x}^{*})\leq c(\vec{x})\qquad\forall \vec{x}\in \mathcal{F}
\]
\end{frame}

\begin{frame}{Problema decisionale}
In certi casi ci\`o che il problema richiede \`e semplicemente la determinazione
di una qualsiasi soluzione realizzabile, ovvero di determinare se
l'insieme realizzabile $\mathcal{F}$ sia vuoto o meno; in questo caso si
parla di \alert{problema decisionale} oppure di \alert{problema di esistenza}


Per tali problemi, si richiede di fornire un elemento $\vec{x}\in\mathcal{F}$, se
ne esiste uno, oppure di dichiarare che $\mathcal{F}$ \`e vuoto
\end{frame}


\section{Modelli nella Ricerca Operativa}

\begin{frame}{Modelli}

 Quando si procede alla \alert{formulazione} ed alla \alert{modellizzazione}
 di un problema decisionale si deve determinare:
  \begin{itemize}
    \item 
 cos'\`e realmente importante
    \item 
 quali sono gli aspetti rilevanti e quali ipotesi semplificative sono verificate
    \item 
 quali obiettivi ci si prefigge
    \item 
 quali sistemi vengono coinvolti, ossia quali sono le entit\`a e quali sono gli attori
    \item 
 quali sono i parametri di interesse
    \item 
 se esistono disturbi
    \item 
 quali condizioni e regole devono essere modellizzate
  \end{itemize} 
\end{frame}

\begin{frame}{Tipi di modelli}
 Nella ricerca operativa si impiegano vari tipi di modelli, tra cui:
 \begin{itemize}
   \item  Modelli di programmazione matematica
   \begin{itemize}
      \item Modelli di programmazione lineare (P.L.)
      \item Modelli di programmazione lineare binaria (P.L.B.)
      \item Modelli di programmazione lineare intera(P.L.I.)
      \item Modelli di programmazione non lineare (P.N.L.)
      \item Modelli di programmazione non lineare intera (P.N.L.I.)
    \end{itemize}
    \item  Modelli su reti e grafi 
    \item  Modelli di teoria delle code
    \item  Modelli della teoria dei giochi
    \item  Modelli di Inventory (gestione delle scorte)
    \item  Modelli di simulazione
  \end{itemize}
 \end{frame}

\begin{frame}{Tipi di modelli}
    \begin{block}{Lineari o non lineari}
    \begin{itemize}
     \item {le relazioni che legano le quantit\`a coinvolte sono lineari
     (somme, prodotto per coefficienti costanti) o non lineari
     (esponenziali, prodotti tra variabili, ...)}
    \end{itemize}
    \end{block}

    \begin{block}{Continui o discreti}
    \begin{itemize}
     \item {le quantit\`a da determinare possono variare con continuit\`a
     in un sottoinsieme di valori reali}
     (es., sono ammesse soluzioni rappresentate da numeri frazionali)
     oppure possono solamente assumere valori interi
    \end{itemize}
  \end{block}
\end{frame}

\begin{frame}{Tipi di modelli}
  \begin{block}{Deterministici o stocastici}
    \begin{itemize}
     \item {tutti parametri coinvolti sono noti a priori e non variano
     oppure il valore di alcuni di tali parametri \`e noto solo in probabilit\`a
     (sono variabili aleatorie)}
    \end{itemize}
  \end{block}
  
  \begin{block}{Misti}
   \begin{itemize}
     \item {i modelli includono pi\`u di una sola tra le precedenti caratteristiche
     alternative (esempio, modelli con parte delle quantit\`a incognite di tipo
     continuo e parte di  tipo discreto)}
    \end{itemize}
  \end{block}
\end{frame}

\section{Modelli di ottimizzazione e di simulazione}

\begin{frame}{Tipi di modelli -- ottimizzazione e simulazione}
  \begin{block}{Modelli di Simulazione}
    \begin{itemize}
      \item riproducono le caratteristiche salienti (di interesse) del sistema che si sta considerando (come si comporta)
      \item prendono decisioni osservando gli effetti di queste su un modello del sistema reale
    \end{itemize}
  \end{block}


  \begin{block}{Modelli di Ottimizzazione}
    \begin{itemize}
      \item descrivono in forma analitica le relazioni salienti che regolano il sistema considerato (come si comporta)
      \item descrivono in forma analitica le caratteristiche desiderate per le alternative decisionali (cosa si vuole ottenere)
    \end{itemize}
  \end{block}
\end{frame}

\begin{frame}{Modelli di Simulazione e di Ottimizzazione}
  \begin{block}{Modelli di simulazione}
   \begin{itemize}
    \item spesso risultano pi\`u semplici da costruire e da comprendere
    \item di solito sono pi\`u semplici da modificare
    \item sono pi\`u ``vicini'' al reale sistema considerato
    \item possono essere, a volte, gli unici disponibili
    \item richiedono una lunga campagna sperimentale per produrre informazioni statisticamente significative
   \end{itemize}
  \end{block}

 La cattiva esecuzione della campagna sperimentale e/o una inadeguata conoscenza statistica pu\`o produrre conclusioni erronee dall’utilizzo dei modelli di simulazione
\end{frame}

\begin{frame}{Modelli di Simulazione e di Ottimizzazione}
  \begin{block}{Modelli di ottimizzazione}
   \begin{itemize}
    \item sono di solito meno costosi da sviluppare e da validare
    \item sono solitamente focalizzati solo sugli aspetti realmente rilevanti del sistema considerato, ovvero quelli che influenzano principalmente la decisione
   \end{itemize}
  \end{block}
\end{frame}

\begin{frame}{Scelta del modello}
  \begin{itemize}
   \item Il modello deve risultare il pi\`u semplice possibile, ossia, la complessit\`a del modello deve essere tenuta sotto controllo
   \item Una buona pratica \`e partire da un modello base conosciuto per un caso pi\`u semplice di quello in esame, quindi adattarlo progressivamente al problema analizzato
  \end{itemize}
\end{frame}

\begin{frame}{Scelta del modello}
  \begin{itemize}
   \item \`E importante tenere in considerazione quali dati sono realmente disponibili e quanto questi possano essere considerati certi/affidabili (un modello dettagliato che si basa su dati mancanti o inaffidabili ha ovviamente poco senso)
   \item \`E sempre indispensabile coinvolgere il decisore (il responsabile dell'applicazione delle decisioni) nel processo di definizione del modello
  \end{itemize}
\end{frame}

\section{Problemi, modelli ed algoritmi}

\begin{frame}{Modello: descrizione formale}
  
  Un modello descrive formalmente un problema decisionale:

  \begin{itemize}
    \item come si comporta il sistema oggetto della decisione
  
    \item qual \`e la natura delle decisioni che si vogliono prendere (come deve essere ``fatta'' una soluzione al problema)
  \end{itemize}

\end{frame}

\begin{frame}{Algoritmo: metodo di soluzione}
\begin{itemize}
\item Un algoritmo \`e lo strumento utilizzato per determinare la/le soluzioni al problema
  
\item Compito del ricercatore operativo \`e, una volta costruito il modello del problema, individuare l’algoritmo pi\`u idoneo a determinarne le soluzioni, ovvero progettare tale algoritmo
  
\item Un algoritmo \`e una procedura iterativa costituita da un numero finito di passi
  
\item A parte situazioni molto semplici (pochi dati, poche alternative) l’esecuzione di un algoritmo \`e sempre affidata ad un computer
\end{itemize}
\end{frame}

\begin{frame}{Problemi, modelli ed algoritmi}
\begin{itemize}
\item 
  Problemi, modelli e algoritmi sono strettamente collegati tra loro
\item 
 Purtroppo la maggior parte dei problemi che emergono in contesti gestionali hanno una natura complessa (problemi difficili)
\item 
 La difficolt\`a di un problema non \`e legata all’incapacit\`a di modellizzarlo, né all’incapacit\`a di trovare una procedura che lo risolva
\item 
 La facilit\`a o difficolt\`a di un problema \`e legata all’esistenza di un algoritmo di soluzione cosiddetto efficiente
\end{itemize}
\end{frame}

\end{document}