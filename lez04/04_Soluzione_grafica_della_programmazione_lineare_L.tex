\documentclass{beamer}
\def\presentationtype{0}
\input{../template/savoia_benincasa}

\begin{document}

\generatitolo

\section{Risoluzione grafica}

\begin{frame}{Ipotesi risoluzione grafica}
  \begin{itemize}
  \item Quando un problema di P.L. presenta due variabili decisionali \`e possibile
      impiegare un metodo di risoluzione basato su considerazioni geometriche e 
      rappresentabile graficamente sul piano cartesiano $Ox_1x_2$
  \item tale metodo pu\`o essere esteso a problemi in tre variabili decisionali
  \item \`e, in generale, possibile risolvere graficamente problemi con funzione obiettivo
    lineare e insieme ammissibile convesso
  \end{itemize}
\end{frame}

\section{Programma lineare di esempio}

\begin{frame}[allowframebreaks]{Problema di esempio}
\fontsize{7}{8.4}{
\begin{block}{Mix ottimo di produzione}
Un'azienda di produzione vuole determinare il tasso di produzione mensile di due prodotti
in modo da massimizzare il profitto netto totale,  sapendo che:
 \begin{itemize}
  \item per produrre un quintale di prodotto 1 occorrono 40 quintali di materia prima e 8 ore di lavoro;
  \item per produrre un quintale di prodotto 2 occorrono 20 quintali di materia prima e 2 ore di lavoro;
  \item il commerciale ha stabilito che la produzione totale mensile non pu\`o superare 100 quintali;
  \item la disponibilit\`a mensile di materia prima \`e di 2200 quintali e quella di lavoro di 320 ore;
  \item il profitto netto per la vendita dei prodotti 1 e 2 sia rispettivamente 120 e 40 euro.
 \end{itemize}
\end{block}
}

\framebreak

\begin{block}{Definizione delle variabili}
si deve determinare un tasso di produzione mensile di due prodotti,
per cui definiamo due variabili che rappresentano la quantit\`a di
prodotto di tipo 1 e di tipo 2 realizzate (e vendute) mensilmente

\begin{description}
\item[$\quad x_1$]	quantit\`a di prodotto 1 da produrre (q/mese)
\item[$\quad x_2$]	quantit\`a di prodotto 2 da produrre (q/mese)
\end{description}

Le variabili sono continue e non negative.
\end{block}

\framebreak

\begin{block}{Definizione dell'obiettivo}
  si deve determinare un tasso di produzione mensile di due prodotti,
  che massimizzi il profitto totale.
  Il profitto, espresso in funzione delle variabili decisionale \`e:

\begin{description}
 \item[$\quad z = 120 x_1 + 40 x_2$]	funzione obiettivo: massimo profitto totale
\end{description}

  Il profitto \`e espresso da una funzione lineare
\end{block}

\begin{block}{Definizione dei vincoli}
la produzione deve rispettare  le condizioni rilevate
in fase di analisi del problema.\\

La quantit\`a di materia prima consumata in un mese non pu\`o superare la disponibilit\`a
\begin{description}
 \item[$\quad 40 x_1 + 20 x_2 \leq 2200$]
      vincoli sull'uso delle materie prime
\end{description}

Il numero di ore necessario per completare la produzione mensile non pu\`o eccedere
la disponibilit\`a oraria
\begin{description}
 \item[$\quad \phantom{4}8 x_1 + \phantom{2}2 x_2 \leq \phantom{2}320$]
      vincoli sull'uso della forza lavoro
\end{description}
 
 L'offerta non pu\`o superare la domanda
\begin{description}
 \item[$\quad \phantom{40} x_1 + \phantom{20} x_2 \leq \phantom{2}100$]
      vincoli conseguenti le indagini di mercato
\end{description}

La produzione non pu\`o essere negativa
\begin{description}
 \item[$\quad x_1 \geq 0,\ x_2 \geq 0$]
      vincoli di non negativit\`a delle variabili
\end{description}

  I vincoli sono espressi da funzioni lineari
\end{block}

\framebreak

\modelbox{Modello}{%
\[\begin{array}{crrrcr}
\max z=& 120 x_1 &+& 40 x_2\\
{\rm s.t.} & 40 x_1 &+& 20 x_2 & \leq & 2200 \\
           &  8 x_1 &+&  8 x_2 & \leq &  320  \\
           &    x_1 &+&    x_2 & \leq &  100  \\
           \multicolumn{6}{c}{x_1 \geq 0,\ x_2 \geq 0}
\end{array}%
\]%
}
\end{frame}

\section{Rappresentazione dei vincoli}

\begin{frame}[allowframebreaks]{Rappresentazione geometrica}
\small{

\begin{itemize}
\item 
Il problema ha variabili decisionali $x_1$ e $x_2$:
la risoluzione grafica verr\`a effettuate su di un piano cartesiano con assi
coordinati $x_1$ e $x_2$, che si indica con $Ox_1x_2$.
Essendo le variabili non negative, la soluzione sar\`a una coppia di punti
all'interno del primo quadrante

\item 
Ogni equazione rappresenta una retta sul piano cartesiano $Ox_1x_2$.
Una disequazione corrisponde ad un semipiano che ha come frontiera la retta
associata all'uguaglianza e comprende tutti i punti che soddisfano la
disequazione

\item 
L'insieme delle soluzioni ammissibili \`e dato dall'intersezione dei semipiani
individuati dalle disequazioni, ovvero l'insieme delle soluzioni che soddisfano
contemporaneamente tutti i vincoli

\item 
La frontiera fa parte della regione ammissibile in quanto tutti i vincoli
includono l'uguaglianza

\item 
\`E possibile tracciare la direzione di massima crescita della funzione obiettivo,
determinata dal gradiente, e le curve di livello della funzione obiettivo, le quali
sono ortogonali al gradiente

\item 
La soluzione del problema \`e la coppia di valori nel piano $Ox_1x_2$ che rende
massima (minima) la funzione obiettivo; pu\`o essere trovata graficamente traslando
il pi\`u possibile le curve di livello della f.o. lungo la direzione del gradiente
della f.o. (antigradiente della f.o.) in modo che abbia una intersezione con la
regione di ammissibilit\`a non vuota
\end{itemize}
}
\end{frame}

%%%%%%%%%%%
\begin{frame}
\definecolor{ffqqtt}{rgb}{1,0,0.2}
\definecolor{ccqqqq}{rgb}{0.8,0,0}
\definecolor{ttqqqq}{rgb}{0.2,0,0}
\definecolor{uuuuuu}{rgb}{0.27,0.27,0.27}
\definecolor{qqqqff}{rgb}{0,0,1}
\centering
\begin{tikzpicture}[line cap=round,line join=round,>=triangle 45,x=0.035cm,y=0.035cm]
\draw[->,color=black] (-20,0) -- (165,0);
\foreach \x in {-20,20,40,60,80,100,120,140,160}
\draw[shift={(\x,0)},color=black] (0pt,2pt) -- (0pt,-2pt) node[below] {\footnotesize $\x$};
\draw[color=black] (155.54,1.27) node [anchor=south west] {$x_1$};
\draw[->,color=black] (0,-20) -- (0,165);
\foreach \y in {-20,20,40,60,80,100,120,140,160}
\draw[shift={(0,\y)},color=black] (2pt,0pt) -- (-2pt,0pt) node[left] {\footnotesize $\y$};
\draw[color=black] (1.63,158.35) node [anchor=west] {$x_2$};
\draw[color=black] (0pt,-10pt) node[right] {\footnotesize $0$};
\clip(-20,-20) rectangle (180,170);

% P0
\onslide<2>{
\fill[color=blue!50!white,opacity=0.3] (0,0)--(165,0)--(165,165)--(0,165)--cycle; %,pattern=north west lines
}
% P1
\onslide<3>{
\fill[color=blue!50!white,opacity=0.3] (0,0)--(55,0)--(0,110)--cycle; %,pattern=north west lines
}
% P2
\onslide<4>{
\fill[color=blue!50!white,opacity=0.3] (0,0)--(40,0)--(25,60)--(0,110)--cycle; %,pattern=north west lines
}
% P3
\onslide<5->{
\fill[color=blue!50!white,opacity=0.3] (0,0)--(40,0)--(25,60)--(10,90)--(0,100)--cycle; %,pattern=north west lines
}
% vincolo 1
\onslide<3->{
  \draw [domain=-20:80] plot(\x,{(--2200-40*\x)/20});
}
% vincolo 2
\onslide<4->{
  \draw [domain=-5:55] plot(\x,{(--320-8*\x)/2});
}
% vincolo 3
\onslide<5->{
  \draw [domain=-20:125] plot(\x,{(--100-1*\x)/1});
}
% gradiente
\onslide<6->{
  \draw [->,color=ccqqqq] (0,0) -- (120,40);
}
% isoprofitto z = 0
\onslide<7>{
  \draw [dash pattern=on 1pt off 2pt on 3pt off 4pt,domain=-20:15] plot(\x,{(-0--120*\x)/-40});
}
% isoprofitto z = 4800
\onslide<8>{
  \draw [dash pattern=on 3pt off 3pt,domain=-20:55] plot(\x,{(-4800--120*\x)/-40});
}
% isoprofitto <= 5400
\onslide<9->{
  \draw [line width=1.2pt,dash pattern=on 1pt off 2pt on 3pt off 4pt,color=ffqqtt,domain=-20:105] plot(\x,{(-5400--120*\x)/-40});
}
\begin{scriptsize}
\onslide<1>{
  \draw[color=structure] (82,160) node {Il piano cartesiano $x_1$, $x_2$};
}
\onslide<2-5>{
  \draw[color=structure] (82,145) node {Il primo quadrante $x_1 \geq 0$ e $x_2 \geq 0$};
}
% punto O
\onslide<2->{
  \fill [color=qqqqff] (0,0) circle (1.5pt);
  \draw[color=qqqqff] (-5,-8) node {$O$};
}
\onslide<3-5>{
  \draw[color=structure] (82,130) node {$\cap$ il primo vincolo};
}
% punto E
\onslide<3->{
  \fill [color=uuuuuu] (55,0) circle (1.5pt);
  \draw[color=uuuuuu] (50,-8) node {$E$};
% punto I
  \fill [color=uuuuuu] (0,110) circle (1.5pt);
  \draw[color=uuuuuu] (-5,110) node {$I$};
}
\onslide<4-5>{
  \draw[color=structure] (82,115) node {$\cap$ il secondo vincolo};
}
% punto A
\onslide<4->{
  \fill [color=qqqqff] (40,0) circle (1.5pt);
  \draw[color=qqqqff] (30,-8) node {$A$};
% punto H
  \fill [color=uuuuuu] (0,160) circle (1.5pt);
  \draw[color=uuuuuu] (-5,150) node {$H$};
% punto B
  \fill [color=qqqqff] (25,60) circle (1.5pt);
  \draw[color=qqqqff] (30,60) node {$B$};
}
\onslide<5>{
  \draw[color=structure] (82,100) node {$\cap$ il terzo vincolo};
  \draw[color=structure] (82, 85) node {\textbf{REGIONE AMMISSIBILE}};
}
\onslide<5->{
  \fill [color=uuuuuu] (100,0) circle (1.5pt);
  \draw[color=uuuuuu] (90,-8) node {$F$};
  \fill [color=ttqqqq] (0,100) circle (1.5pt);
  \draw[color=ttqqqq] (-5,90) node {$D$};
  \fill [color=uuuuuu] (10,90) circle (1.5pt);
  \draw[color=uuuuuu] (12,95) node {$C$};
  \fill [color=uuuuuu] (20,80) circle (1.5pt);
  \draw[color=uuuuuu] (25,85) node {$G$};
}
\onslide<6>{
  \draw[color=structure] (110,85) node {il gradiente della funzione obiettivo};
  \draw[color=structure] (110,75) node {direzione di massima crescita};
  \draw[color=structure] (110,65) node {di $z(x_1,x_2)$};
}
\onslide<6->{
  \draw[color=ccqqqq] (120,50) node {$\nabla z$};
}
\onslide<7>{
  \draw[color=structure] (110,115) node {retta isoprofitto};
  \draw[color=structure] (110,105) node {$z(x_1, x_2) = 0$};
  \draw[color=structure] (110, 95) node {ortogonale alla f.o.};
  \draw[color=structure] (110, 85) node {passante per $O$};
  \draw[color=black] (-10,30) node {$z_0 = 0$};
}
\onslide<8>{
  \draw[color=structure] (110,115) node {retta isoprofitto};
  \draw[color=structure] (110,105) node {$z(x_1, x_2) = 4800$};
  \draw[color=structure] (110, 95) node {spostandosi in direzione di $\nabla z(x_1,x_2)$};
  \draw[color=black] (25,-17) node {$z_1= 4800$};
}
\onslide<9>{
  \draw[color=structure] (110,115) node {retta isoprofitto};
  \draw[color=structure] (110,105) node {$z(x_1, x_2) = 5400$};
  \draw[color=structure] (110, 95) node {$B$ ultimo punto ammissibile};
  \draw[color=structure] (110, 85) node {spostandosi in direzione di $\nabla z(x_1,x_2)$};
  \draw[color=structure] (110, 70) node {\textbf{OTTIMO}: $x^\star \equiv B = (25, 60),\ z^\star = 5400$};
  \draw[color=ffqqtt] (33,97) node {$z_2 = 5400$};
}
\end{scriptsize}
\end{tikzpicture}
\end{frame}
%%%%%%%%%%%

\section{Metodo}

\begin{frame}[allowframebreaks]{Presupposti}
\begin{itemize}
\item
Ogni disequazione rappresenta sul piano cartesiano un semipiano,
ogni equazione rappresenta una retta

\item
L'intersezione dei semipiani \`e la regione di ammissibilit\`a

\item
La funzione obiettivo \`e rappresentata da un fascio di rette parallele
fra di loro e ortogonali al vettore gradiente di $z$ (direzione di massima
crescita della funzione obiettivo)

\item
Per un problema di $\max$ ($\min$) la traslazione della funzione obiettivo
\`e nel verso del vettore gradiente (antigradiente)

\item
Trovato graficamente il punto di ottimo risolvere il sistema di equazioni dei
vincoli saturi per trovare la soluzione ottima che sostituita nella funzione
obiettivo dar\`a il valore dell'ottimo

\item
Il gradiente della funzione obiettivo \`e dato dai coefficienti delle variabili,
essendo la funzione lineare

\item
Si ricordi che la regione ammissibile pu\`o essere vuota, illimitata o limitata
e che nel caso di regione ammissibile illimitata l'ottimo pu\`o non esistere
\end{itemize}
\end{frame}

\begin{frame}[allowframebreaks]{Regione ammissibile}
\scriptsize{
I vincoli di non negativit\`a delle variabili consentono di considerare esclusivamente il
primo quadrante del sistema di assi cartesiano.

% Per rappresentare geometricamente la regione
% ammissibile del problema di P.L. 
Si utilizzi la seguente procedura:
\begin{description}

\item[Step 1: Trasformare la disequazione in equazione e rappresentare graficamente il vincolo] 
Si converta il vincolo di disuguaglianza in uguaglianza. Si rappresenti %sul piano cartesiano
tale retta

\item[Step 2: Selezionare un punto di valutazione]

Si scelga un punto del piano per valutare il vincolo (si consiglia di utilizzare 
l'origine degli assi, tranne nel caso in cui il vincolo passa per l'origine)

\item[Step 3: Determinare il semipiano di ammissibilit\`a]

Sostituendo le coordinate del punto di valutazione nel vincolo di disuguaglianza iniziale, si
possono verificare due casi

\begin{description}
\item[Caso 1]
La disuguaglianza \`e verificata, allora il semipiano ammissibile \`e quello che
contiene il punto di valutazione

\item[Caso 2]
La disuguaglianza non \`e verificata, allora il semipiano ammissibile \`e quello
che non contiene il punto di valutazione
\end{description}
\end{description}
}

\framebreak

\begin{description}
\item[Step 4: Ripetere i 3 step precedenti per ciascun vincolo]

\item[Step 5: Individuare la regione ammissibile]
La regione ammissibile \`e individuata dall’intersezione dei semipiani di ammissibilit\`a di tutti i 
vincoli, compresi quelli di non negativit\`a. Di conseguenza la regione ammissibile \`e un insieme 
convesso
\end{description}
\end{frame}


\begin{frame}{Funzione obiettivo e curve isoprofitto}

\[z=c_1 x_1 + c_2 x_2\]

\scriptsize{
\begin{description}
 \item[Step 6: Rappresentare il gradiente]
 Si tracci il vettore $\nabla z = \left|\begin{array}{c}c_1\\c_2\end{array}\right|$
 o un vettore paralllo

 \item[Step 7: Rappresentare le curve isoprofitto]
 Si tracci retta ortogonale al gradiente e passante per l'origine.
 Questa \`e una curva isoprofitto appartenente al fascio di rette $c_1 x_1 + c_2 x_2 = k,\ k\in\mathbb{R}$
 
 \item[Step 8: Valutazione]
 Se il problema \`e di
 \begin{description}
  \item[$\min$] l'obiettivo migliora spostandosi in direzione opposta al gradiente
  \item[$\max$] l'obiettivo migliora spostandosi in direzione del gradiente
 \end{description}
 \end{description}
}
\end{frame}

\begin{frame}{Determinazione della soluzione ottima}
\scriptsize{
Il punto di ottimo \`e quel vertice o quel segmento in corrispondenza del
quale la funzione obiettivo assume il valore massimo (problema di
massimizzazione) o minimo (problema di minimizzazione)

\begin{description}
 \item[Step 9: Determinare il vertice ottimo]
 Si determini, traslando il pi\`u possibile le curve di livello della f.o. lungo
 la direzione di massimo miglioramento (Step 8) in modo che abbia una intersezione
 con la regione di ammissibilit\`a non vuota. L'ultimo vertice, o segmento, \`e
 il luogo dei punti di ottimo.
 Si calcolino le coordinate del punto di ottimo (o di uno dei vertici nel caso
 di un segmento) come intersezione dei vincoli soddisfatti all'uguaglianza
 

 \item[Step 10: Determinare il valore della f.o. all'ottimo]
 Si valuti il valore della funzione obiettivo in corrispondenza del vertice
 determinato
 \end{description} 
}
\end{frame}

\end{document}