\documentclass{beamer}
\def\presentationtype{1}
\input{../template/savoia_benincasa}

\begin{document}

\generatitolo

\begin{frame}[allowframebreaks]{\domande}
Rispondere alle seguenti domande servendosi della lezione
e di altre fonti attendibili:

    \begin{itemize}
     \item Il modello detto Programma Lineare \`e un modello di simulazione?
     
     \item Un problema di PL in forma standard \`e nella forma
     di problema di minimo?
     
     \item La funzione obiettivo mappa gli elementi dallo spazio vettoriale
     $\mathbb{R}^n$ in $\mathbb{R}^m$, con $m > 1$ e $n \geq 1$?
     
     \item Un Programma Lineare ammette sempre soluzioni realizzabili?
     
     \item Un problema di PL in forma standard tutti i vincoli sono
     espressi in forma di \emph{maggiore o uguale}?
     
     \item \`E possibile esprimere un problema di PL in forma matriciale?
     
     \item Come possiamo classificare un'istanza di un problema di PL in
     base allo spazio alla/e sue soluzione/i?
     
     \item Tutti i problemi reali possono essere modellizzati come
     problemi di PL?
    
    \end{itemize}
\end{frame}

\end{document}