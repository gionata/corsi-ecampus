\documentclass{beamer}
\def\presentationtype{2}
\input{../template/savoia_benincasa}
\begin{document}

\generatitolo

\section{Esercizi}

\begin{frame}
{Esercizi}
Con riferimento all'esercizio presentato nelle slide della lezione e agli allegati

\begin{enumerate}
 \item commentare il modello ed i dati in GNU MathProg;
 \item descrivere come \`e stato realizzato il vincolo di equit\`a;
 \item modificare il modello per eliminare l'ultimo dei vincoli di
	equit\`a;
 \item controllare che le soluzioni dei due modelli, con e senza vincolo ridondante,
       siano equivalenti.
\end{enumerate}

\end{frame}

\section{Allegati}


\begin{frame}[allowframebreaks]{Modello in GNU MathProg -- modello}
\lstinputlisting{allegati/prod_ittica.mod}
\end{frame}


\begin{frame}[allowframebreaks]{Modello in GNU MathProg -- dati}
\lstinputlisting{allegati/prod_ittica.dat}
\end{frame}


\end{document}
