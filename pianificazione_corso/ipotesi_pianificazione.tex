\documentclass[a4paper,10pt]{article}

\usepackage[utf8]{inputenc}
\usepackage[italian]{babel}
\usepackage{fontenc}
%\usepackage{graphicx}
\usepackage{xcolor}
\usepackage{hyperref}
\usepackage{enumerate}

\date{\today}

\newcommand{\copiare}[2]{%
  \begin{flushright}
  \textcolor{blue}{Copiare da }\textcolor{red}{#1}:\textcolor{orange}{#2}    
  \end{flushright}%
}

\begin{document}
 
\section{Ipotesi sui moduli del programma}

\def\modIntro{\hfill\hyperlink{mod:Intro}{\textcolor{red}{Modulo 0}}}
\def\modRichiami{\hfill\hyperlink{mod:Richiami}{\textcolor{red}{Modulo 1}}}
\def\modSw{\hfill\hyperlink{mod:Software}{\textcolor{red}{Modulo 2}}}

\begin{enumerate}
 \item \hypertarget{mod:Intro}{Introduzione (1 lezione)}\hfill\textcolor{red}{Modulo~0}
 
 \item \hypertarget{mod:Richiami}{Richiami (10 lezioni: algebra lineare e PL. Metodo grafico e simplesso, dualit\`a)}\hfill\textcolor{red}{Modulo~1}
 
  \item \hypertarget{mod:Software}{Uso di strumenti software per l'ottimizzazione (5 lezioni: 1 risolutore LibreOffice, 2 LINGO e 2 AMPL)}\hfill\textcolor{red}{Modulo~2}
  
 \item Ottimizzazione su rete (11 lezioni:
 	1 teoria grafi,
 	3 trasporto,
 	inizializzazione,
 	Dantzig,
 	1 assegnamentoe Ungherese,
 	1 cammino minimo, Dijstra,
 	1 Massimo Flusso, Fork-Fulkerson,
 	1 Albero ricoprente di costo minimo, Prim, Kruskal,
 	1 CPM, 
 	1 Flusso a costo minimo, Simplesso su rete,
 	1 Ripasso.
 ) \hfill\textcolor{red}{Modulo~3}
 
 \item Modelli statistici per le decisioni di gestione della logistica della produzione\hfill\textcolor{red}{Modulo~4}
 
 \item Problemi di gestione e controllo delle scorte
 (4 lezioni. Da Gianluca 2010 lez 5 e 2011 lez 2: 1 perch\'e, classificazione e codifica, 1 analisi ABC, 1 Lotto economico EOQ, 1 Punto di riordino) \hfill\textcolor{red}{Modulo~5}
 
 \item Modelli di simulazione per la gestione delle scorte
 (Da Gianluca 2011 lez 1: ) \hfill\textcolor{red}{Modulo~6}
 
 \item Metodi previsionali a breve termine
 (Analisi delle serie storiche, smoothing, regressione lineare)
 \hfill\textcolor{red}{Modulo~7}
 
  \item Ripasso (una lezione al termine di ogni modulo e due a fine corso)
  \hfill\textcolor{red}{Modulo~8}
  
  \item Esercizi svolti (ogni tanto)
  \hfill\textcolor{red}{Modulo~9}
\end{enumerate}

\section{Ipotesi sui supporti didattici}

Da \cite{winston2004};
\begin{itemize}
\item Cap. 7: trasporto, inizializzazione, Dantzig, assegnamento, Ungherese
\item Cap. 8:
	cammino minimo,Dijkstra,
	Massimo flusso, Ford-Fulkerson,
	Albero ricoprente di costo minimo, Prim, Kruskal
	CPM,
	Flusso a costo minimo,
	Simplesso su rete
\end{itemize}
 
\section{Ipotesi sulle lezioni}

\begin{enumerate}
%  1
 \item \textcolor{gray}{Introduzione}\modIntro
%  2
 \item \textcolor{gray}{Modelli e metodi per il supporto alle decisioni}\modRichiami
%  3
 \item \textcolor{gray}{Il modello di Programmazione Lineare}\modRichiami
%  4
 \item \textcolor{gray}{Risoluzione geometrica della Programmazione Lineare}\modRichiami
%  5
 \item \textcolor{gray}{Risoluzione geometrica della Programmazione Lineare -- Post--ottimalit\`a}\modRichiami
%  6
 \item \textcolor{gray}{Sistemi di equazioni lineari}\modRichiami
%  7
 \item \textcolor{gray}{Programmazione Lineare, la forma canonica}\modRichiami
%  8
 \item \textcolor{gray}{Metodo del Simplesso: il ``tableau'' e l'operazione ``pivot''}\modRichiami
%  9
 \item \textcolor{gray}{Metodo del Simplesso: algoritmo primale standard}\modRichiami
% 10
 \item \textcolor{gray}{Formulare e risolvere modelli di PL mediante fogli elettronici}\modSw
% 11
 \item \textcolor{gray}{Introduzione all'ambiente LINGO}\modSw
% 12
 \item \textcolor{gray}{Formulare e risolvere modelli di PL mediante LINGO}\modSw
% 13
 \item \textcolor{gray}{Linguaggio GNU MathProg -- Introduzione}\modSw
% 14
 \item \textcolor{gray}{Formulare e risolvere modelli di PL mediante GNU MathProg}\modSw
% 15
 \item  \textcolor{gray}{Esempi di problemi modellizzabili come Programmi Lineari}\modSw
% 16 
 \item Impostazione di modelli in programmazione lineare\modSw
 %\item Metodo del Simplesso in forma matriciale \copiare{20}{tutto}
 \item Teoria della dualit\`a \copiare{21--24}{sintesi}
% \item Metodo duale del Simplesso \copiare{25}{sintesi}
 \item Ripasso
 \item Reti \copiare{30}{tutto} \copiare{31}{tutto}
 \item Il problema del trasporto \copiare{32}{tutto}
 \copiare{Modelli operativi applicati alla logistica}{203}
 \item Inizializzazione del trasporto \copiare{33}{tutto}
	\copiare{35}{tutto}
% \item Esercizi trasporto Nord Ovest, Minimi Costi, Vogel
 \item Il simplesso per problema del trasporto: metodo di Dantzing
 \item Assegnamento e Metodo Ungherese \copiare{36}{tutto} \copiare{37}{tutto}
 \item Albero ricoprente, Prim, Kruskal
 \item Cammini minimi, Dijkstra
 \item Massimo flusso Ford-Fulkerson
 \item Flusso a costo minimo, simplesso su rete
 \item Tecniche reticolari, CPM
 \item Ripasso
 
 \item Modelli statistici per le decisioni di gestione della logistica della produzione
 \item Ripasso
 \item Problemi di gestione e controllo delle scorte \copiare{Modelli operativi applicati alla logistica}{118}
 \item Ripasso
 \item Modelli di simulazione per la gestione delle scorte
 \item Ripasso
 \item Metodi previsionali a breve termine
 \item Ripasso
 \item Domande d'esame
 \item Domande d'esame

\end{enumerate}

\begin{thebibliography}{9}

\bibitem{winston2004}{
	{Winston, W.L. and Goldberg, J.B.},
    \emph{Operations research: applications and algorithms},
  {9780534423582},
  {Thomson Brooks/Cole},
  {2004}.
}

\end{thebibliography}

\end{document}
