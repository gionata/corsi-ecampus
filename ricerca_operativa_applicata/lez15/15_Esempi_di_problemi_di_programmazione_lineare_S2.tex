\documentclass{beamer}
\usepackage{booktabs}
\def\presentationtype{2}
\input{../template/savoia_benincasa}
\setbeamersize{description width=10pt}

\setbeamertemplate{enumerate item}{Es. \nolezione.\presentationtype.\arabic{enumi})}
\begin{document}

\generatitolo


\begin{frame}[allowframebreaks]
{Esercizi}

Scrivere il modello di PL e i file dati per GNU MathProg
per risolvere il seguente problema:

\begin{block}{Preparazione di un cocktail}
Un barista deve preparare una bevanda che contenga
almeno il 20\% di succo d'arancia,
almeno il 10\% di succo di pompelmo e
almeno il  5\% di succo di mirtillo
utilizzando 5 bevande che ha in magazzino.
Determinare le percentuali dei vari succhi per produrre
almeno 2300 litri di bevanda note le disponibilit\`a
riportate in tabella alla slide seguente.
\end{block}

\framebreak

\begin{table}
\begin{tabular}{crrrrr}\toprule
{\scriptsize \bf Bevanda} & {\scriptsize \bf Arancia \%} & {\scriptsize \bf Pompelmo \%} & {\scriptsize \bf Mirtillo \%} &{\scriptsize \bf Disponibilit\`a (l)}& {\scriptsize \bf Costo (\EUR{})}\\\midrule
A &  40 &  40 &   0 &  900 & 2,25\\
B &   5 &  10 &  20 & 1800 & 1,20\\
C & 100 &   0 &   0 &  450 & 3,00\\
D &   0 & 100 &   0 &  200 & 2,60\\
E &   0 &   0 &   0 & 3600 & 0,40\\\midrule
  &$\ge$ 20&$\ge$ 10&$\ge$  5&$\ge$ 2300\\\bottomrule
\end{tabular}
\caption{Dati del problema}
\end{table}
\end{frame}
\end{document}
