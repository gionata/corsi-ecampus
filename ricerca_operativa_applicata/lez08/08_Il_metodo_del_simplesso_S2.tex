\documentclass{beamer}
\usepackage{booktabs}
\def\presentationtype{2}
\input{../template/savoia_benincasa}
\setbeamersize{description width=10pt}

\setbeamertemplate{enumerate item}{Es. \nolezione.\presentationtype.\arabic{enumi})}
\begin{document}

\generatitolo


\begin{frame}{Relazione sull'operazione di pivot}

    \begin{enumerate}
    
     \item Documentarsi, consultando fonti affidabili, sul modo
     in cui sia possibile rappresentare l'operazione di pivot
     attraverso una premoltiplicazione del tableau per una
     opportuna matrice (detta \emph{matrice pivot} e spesso indicata
     come ``$\matr{Q}$'').\\~\\
     Si studino le matrici elementari di Gauss.\\~\\
     % Produrre una relazione e caricarla sulla piattaforma di e-learning
    \end{enumerate}
\end{frame}
\end{document}
