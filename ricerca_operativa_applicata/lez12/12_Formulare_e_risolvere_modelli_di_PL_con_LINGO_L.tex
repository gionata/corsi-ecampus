\documentclass{beamer}
\def\presentationtype{0}
\input{../template/savoia_benincasa}

\definecolor{comment_c}{RGB}{60,128,49}
\definecolor{rowname_c}{RGB}{128,60,49}
\definecolor{reserved_c}{RGB}{49,60,128}

\makeatletter

%%%%%%%%%%%%%%%%%%%%%%%%%%%%%% LyX specific LaTeX commands.
\newcommand{\noun}[1]{\textsc{#1}}

%%%%%%%%%%%%%%%%%%%%%%%%%%%%%% Textclass specific LaTeX commands.
 % this default might be overridden by plain title style
 \newcommand\makebeamertitle{\frame{\maketitle}}%
 \AtBeginDocument{
   \let\origtableofcontents=\tableofcontents
   \def\tableofcontents{\@ifnextchar[{\origtableofcontents}{\gobbletableofcontents}}
   \def\gobbletableofcontents#1{\origtableofcontents}
 }
 \long\def\lyxframe#1{\@lyxframe#1\@lyxframestop}%
 \def\@lyxframe{\@ifnextchar<{\@@lyxframe}{\@@lyxframe<*>}}%
 \def\@@lyxframe<#1>{\@ifnextchar[{\@@@lyxframe<#1>}{\@@@lyxframe<#1>[]}}
 \def\@@@lyxframe<#1>[{\@ifnextchar<{\@@@@@lyxframe<#1>[}{\@@@@lyxframe<#1>[<*>][}}
 \def\@@@@@lyxframe<#1>[#2]{\@ifnextchar[{\@@@@lyxframe<#1>[#2]}{\@@@@lyxframe<#1>[#2][]}}
 \long\def\@@@@lyxframe<#1>[#2][#3]#4\@lyxframestop#5\lyxframeend{%
   \frame<#1>[#2][#3]{\frametitle{#4}#5}}
 \def\lyxframeend{} % In case there is a superfluous frame end

\makeatother

\begin{document}

\generatitolo

\lyxframeend{}
%\section{Uso del linguaggio di modellazione}

\lyxframeend{}\section{Costruzione del modello: insiemi e parametri}

\lyxframeend{}\lyxframe{Mix ottimo di produzione: modello scalare}
\begin{columns}%{}

\column{.45\textwidth}

%\footnotesize{
\begin{tabular}{lp{0.95\textwidth}}
\color{magenta}$n$		&Numero dei prodotti\\
\color{magenta}$m$		&Numero delle risorse\\
\color{magenta}$a_{ij}$	&Assorbimento della risorsa $i$ da un prodotto $j$\\
\color{magenta}$b_{i}$ 	&Disponibilit\`a della risorsa $i$\\
\color{magenta}$c_{j}$	&Profitto unitario prodotto $j$\\
\color{magenta}$x_{j}$	&Livello di produzione del prodotto $j$
\end{tabular}
%}

\column{.55\textwidth}

\fbox{%
{\footnotesize
$\begin{array}{cccc}
\max z= & {\displaystyle \sum_{\textcolor{magenta}{j=1}}^{\textcolor{magenta}{n}}}c_{j}\,x_{j}\\
	       & {\displaystyle \sum_{\textcolor{magenta}{j=1}}^{\textcolor{magenta}{n}}}a_{ij}\,x_{j} & \leq b_{i} & \color{magenta}i=1,2,\ldots,m\\
 & x_{j}\geq0 &  & \color{magenta}j=1,2,\ldots,n
\end{array}$
}}

\end{columns}%{}

\lyxframeend{}\lyxframe{Parametri e variabili del modello PL come vettori}
{\small
$\left.\mathbf{c}=\left[\begin{array}{c}
c_{1}\\
c_{2}\\
\vdots\\
c_{n}
\end{array}\right]\quad\mathbf{x}=\left[\begin{array}{c}
x_{1}\\
x_{2}\\
\vdots\\
x_{n}
\end{array}\right]\right\} $prodotti,$\left.\mathbf{b}=\left[\begin{array}{c}
b1\\
b_{2}\\
\vdots\\
b_{m}
\end{array}\right]\right\} $risorse,

$\left.\mathbf{A}=\left[\begin{array}{cccc}
a_{11} & a_{12} & \cdots & a_{1n}\\
a_{21} & a_{22} & \cdots & a_{2n}\\
\vdots & \vdots & \ddots & \vdots\\
a_{m1} & a_{m2} & \cdots & a_{mn}
\end{array}\right]\right\} $risorsa$\times$prodotto

\begin{center}
\fbox{%
$\begin{array}{rrl}
\max z = & \mathbf{c}^T \, \mathbf{x}\\
         & \mathbf{A}   \, \mathbf{x} & \leq \mathbf{b}\\
         & \mathbf{x}                 & \geq \mathbf{0}
\end{array}$%
}
\end{center}
}

\lyxframeend{}\lyxframe{Mix ottimo di produzione: modello basato su insiemi}
\begin{columns}%{}

\column{.50\textwidth}

\begin{minipage}{\textwidth}
\footnotesize{
\begin{tabular}{lp{0.9\textwidth}}
\color{magenta}$P$		&Insieme dei prodotti\\
\color{magenta}$R$		&Insieme delle risorse\\
\color{magenta}$a_{ij}$	&Assorbimento della risorsa $i\in P$ da un prodotto $j\in R$\\
\color{magenta}$b_{i}$	&Disponibilit\`a della risorsa $i\in R$\\
\color{magenta}$c_{j}$	&Profitto unitario prodott $j\in P$\\
\color{magenta}$x_{j}$	&Livello di produzione prodotto $j\in P$
\end{tabular}
}
\end{minipage}

\column{.45\textwidth}


\fbox{
\footnotesize{
$\begin{array}{cccc}
\max z= & {\displaystyle \sum_{\textcolor{magenta}{j\in P}}}c_{j}\,x_{j}\\
	       & {\displaystyle \sum_{\textcolor{magenta}{j\in P}}}a_{ij} \,x_{j} & \leq b_{i} & \textcolor{magenta}{i\in R}\\
  & x_{j}\geq0 &  & \textcolor{magenta}{j\in P}
\end{array}$
}
}

\end{columns}%{}

\lyxframeend{}\lyxframe{[allowframebreaks] formulazioni scalari e basate su insiemi}
\begin{block}{Insiemi}
Gli indici sono stati fatti variare in un dominio, quello $P$ dei
prodotti e quello $R$ delle risorse

\begin{itemize}
\item $P=$\{$prod_{1},prod_{2}$\}
\item $R=\mbox{\{}row\_mat,labour,market$\}
\end{itemize}
\end{block}

\begin{block}{Attributi}
Gli elementi degli insiemi hanno degli attributi di interesse

\begin{itemize}
\item $P$: profitto per unit\`a prodotta ($c_{j}$), numero di prodotti
da realizzare ($x_{j}$)
\item $R$: disponibilit\`a della risorsa ($b_{i}$) 
\end{itemize}
\end{block}

\lyxframeend{}\section{Insiemi e attributi}


\lyxframeend{}\lyxframe{Insiemi e attributi in LINGO}

\begin{block}{Costrutto per la dichiarazione di insiemi primitivi}
Gli insiemi e i relativi attributi sono indicati come \\

\noun{insieme}: \noun{attributo}{[},\noun{ attributo}{]}{*};
\end{block}

\begin{exampleblock}{Esempio}
\structure{Prodotto: Profitto, X;}\\
\structure{Risorsa: Disponibilita;}
\end{exampleblock}

\lyxframeend{}\lyxframe{Insiemi derivati}

\begin{itemize}
\item Le quantit\`a $a_{ij}$ di risorsa assorbita dalla realizzazione
di un prodotto sono definite su tutte le coppie ordinate di una risorsa
e un prodotto.
\item Sono attributi del prodotto cartesiano di pi\`u insiemi e come tale
\`e detto insieme derivato.
\end{itemize}

\begin{block}{Costrutto per la dichiarazione di insiemi derivati}
 \noun{insieme\_derivato}(\noun{insieme}, \noun{insieme}{[}, \noun{insieme}{]}{*}):\\
\noun{attributo}{[}, \noun{attributo}{]}{*};
\end{block}

\begin{exampleblock}{Esempio}
\structure{RisorsaProdotto(Risorsa, Prodotto): Assorbimento;}
\end{exampleblock}

\lyxframeend{}\section{Combinazioni lineari e vincoli: iteratori}


\lyxframeend{}\lyxframe{Iterare sugli elementi di un insieme: somme}
\begin{columns}%{}


\column{4.5cm}


{\small Si vuole generare la combinazione lineare che moltiplica profitti
unitari e quantit\`a da produrre}{\small \par}
\begin{columns}%{}
\bigskip{}

\end{columns}%{}

{\Large
 \begin{equation*}
 {\displaystyle
 {\color{magenta!23!green!40!blue!70} \sum_{{\color{green!70!black} j} {\color{black} \in} {\color{magenta} P}}}}
    c_{\color{green!70!black} j}\,x_{{\color{green!70!black} j}} 
  \end{equation*}
}


\column{6.5cm}


{\small Si usa il costrutto

\structure{{\small @SUM(}} }\\
\ \ \  \noun{\small Insieme}{\small {[}(}\noun{\small indice}{\small {[},}\noun{\small indice}{\small {]}{*})\\
\ \ \ 
{[}| }\noun{\small condizioni}{\small {]]}\structure{{\small :}} }\noun{\small espressione}\\
{\small \structure{{\small );}}}{\small \par}
\begin{columns}%{}
\bigskip{}

\end{columns}%{}

{\small
  \structure{@SUM(} 
    {\color{magenta} Prodotto}({\color{green!70!black} j}):
      Profitto({\color{green!70!black} j}){*}X({\color{green!70!black} j})
   \structure{);}
}

\end{columns}%{}

\lyxframeend{}\lyxframe{Iterare sugli elementi di un insieme: cicli}
\begin{columns}%{}


\column{5cm}


{\small Si vogliono generare i vincoli sulle disponibilit\`a delle
risorse, uno per ogni risorsa $i$}\bigskip{}



{\Large  \begin{equation*}   {\displaystyle \sum_{j\in P}} a_{ij} \, x_{j} \leq b_{i} \quad {\color{green!70!black} i}\in {\color{magenta} R}  \end{equation*} }


\column{6cm}


{\small Si usa il costrutto

\structure{{\small @FOR(}} }\\
\ \ \ \noun{\small Insieme}{\small {[}(}\noun{\small indice}{\small {[},
}\noun{\small indice}{\small {]}{*})\\
\ \ \ {[}| }\noun{\small condizioni}{\small {]]}\structure{{\small :}}
}\noun{\small espressione}{\small\\
\structure{{\small );}}}{\small \par}
\begin{columns}%{}
\bigskip{}

\end{columns}%{}

\structure{@FOR(}\textcolor{magenta}{\ Risorsa}(\textcolor{green!70!black}{i}): $<$espressione$>$\structure{);}

\end{columns}%{}

\lyxframeend{}\section{Descrizione di un modello}


\lyxframeend{}\lyxframe{Descrizione di un modello}

\begin{block}{Linguaggio dichiarativo}
si dichiarano l'obiettivo e i dati con cui operare
\end{block}

\begin{block}{Algoritmi di soluzione gi\`a implementati}
la scelta dell'algoritmo di soluzione e la sua implementazione \`e
realizzata dal LINGO stesso
\end{block}

\begin{block}{Definizione di nuovi algoritmi}
si possono definire algoritmi più efficienti per risolvere classi di problemi,
in particolare definire sotto-problemi e implementare algoritmi di Column
Generation
\end{block}

\lyxframeend{}\lyxframe{Sezioni}
La dichiarazione del modello \`e divisa in sezioni, che si dichiarano con
 \structure{$<$Sezione$>$:} e terminano con \structure{$<$ENDSezione$>$}

\begin{description}
\item [{Model}] descrive il modello matematico
\item [{Sets}] descrive gli insiemi e relativi attributi
\item [{Data}] i parametri (istanza) del problema
\end{description}

Altre sezioni
\begin{description}
\item [{Init}] inserimento di soluzioni iniziali per prog.\ intera {[}mista{]}
e non lineare
\item [{Calc}] elaborazioni (per calcolare parametri derivati, per eseguire
istruzioni o descrivere algoritmi\ldots{})
\item [{Submodel}] un modello per un problema che pu\`o essere richiamato
nella sezione Calc
\end{description}

\lyxframeend{}\section{Implementazione del modello del MIX OTTIMO}


\lyxframeend{}

\begin{frame}{Formulazione LINGO del mix ottimo -- Insiemi}

\begin{exampleblock}
{Codice - 1 - Insiemi}

\textcolor{comment_c}{! Mix di produzione ottimo;}

\textcolor{comment_c}{! Sezione SETS;}

\structure{SETS}: 

\textcolor{comment_c}{! Gli insiemi primitivi;}

Risorsa: Disponibilita;

Prodotto: Profitto, X;\\

\textcolor{comment_c}{! Un insieme derivato;}

RisorsaProdotto(Risorsa, Prodotto): Assorbimento;

\structure{ENDSETS}
\end{exampleblock}
\end{frame}

\begin{frame}{Formulazione LINGO del mix ottimo -- Formulazione}
\begin{exampleblock}
{Codice - 2 - Formulazione}

\textcolor{comment_c}{! Il modello del problema di mix ottimo di produzione;}\\
\textcolor{comment_c}{! Massimizza il profitto totale;}

\textcolor{rowname_c}{{[}PROFITTO\_TOTALE{]}} \structure{MAX =} \structure{@SUM}( Prodotto(j):
Profitto(j){*}X(j));\\

\structure{@FOR}( Risorsa(i): \textcolor{comment_c}{! Per ogni risorsa i;}

~~~\textcolor{rowname_c}{{[}RISORSA\_{]}}

~~~\textcolor{comment_c}{! La quantit\`a utilizzata;}

~~~\structure{@SUM}( Prodotto(j): Assorbimento(i,j){*}X(j)) <= 

~~~~~Disponibilita(i); \textcolor{comment_c}{! deve essere <= di quella disponibile;}\\
); 
\end{exampleblock}

\end{frame}

\begin{frame}{Formulazione LINGO del mix ottimo -- Dati}

\begin{exampleblock}
{\scriptsize Codice - 3 - Dati

\textcolor{comment_c}{! Sezione dati;}\\
\structure{DATA}:\\
\textcolor{comment_c}{! I nomi delle risorse e la loro disponibilita';}

\textcolor{black}{%
Risorsa, Disponibilita =\\
~~~row\_mat 2200\\
~~~labour~~~~~~~320\\
~~~market~~~~~~100;}

\textcolor{comment_c}{! I nomi dei prodotti e i loro profitti unitari;}

\textcolor{black}{%
Prodotto, Profitto =\\
~~~prod1 120\\
~~~prod2~~~40;}

\textcolor{comment_c}{! Matrice dei coefficienti tecnologici;}

\textcolor{black}{%
Assorbimento = 40 20}~~\textcolor{comment_c}{! Materie prime;}\\
~~~~~~~~~~~~~~~~~~~~~~~~~~~~\textcolor{black}{8~~~2}~~\textcolor{comment_c}{! Lavoro;}\\
~~~~~~~~~~~~~~~~~~~~~~~~~~~~\textcolor{black}{1~~~1;}~\textcolor{comment_c}{! Mercato;}

\structure{ENDDATA}
}
\end{exampleblock}

\end{frame}

\lyxframe{Comandi}

Selezionare del men\`u LINGO:

\begin{description}
\item [{Generate$\dasharrow$Display~model}] genera il modello istanziato
(\structure{@GEN();});
\item [{Generate$\dasharrow$Dual~model}] genera il modello del problema
duale associato (\structure{@GENDUAL();});
\item [{Solve}] determina una soluzione del modello (\structure{@SOLVE();});
\item [{Range}] determina gli intervalli di stabilit\`a;
\end{description}

\lyxframeend{}\lyxframe{Lettura dati da file}

LINGO permette di leggere dati da:

\begin{itemize}
\item file di testo nel formato LDT 

\begin{itemize}
\item \structure{@FILE( '\noun{filename}')}
\end{itemize}
\item fogli di calcolo Microsoft Excel

\begin{itemize}
\item \structure{@OLE( {[}'\noun{spreadsheet\_file}'{]} {[}, \noun{range\_name\_list}{]})}
\end{itemize}
\item basi di dati con ODBC

\begin{itemize}
\item \structure{@ODBC(\ldots{})}
\end{itemize}
\end{itemize}

\lyxframeend{}\lyxframe{Lettura dati da file di testo}

Il file della formulazione nella sezione dati dichiara il file esterno
da cui leggere 

\begin{itemize}
\item \noun{dato }= \structure{@FILE( '\noun{filename'})};
\end{itemize}

File di testo nel formato LDT

\begin{itemize}
\item Il file \noun{filename} contiene i dati

\begin{itemize}
\item i dati di uno stesso gruppo sono separati da caratteri di spazio
\item i dati di gruppo distinti sono separati simbolo end-of-record (\structure{\textasciitilde{}}).
\end{itemize}
\end{itemize}

\lyxframeend{}\lyxframe{Esempio lettura dati esterni}
\begin{exampleblock}
{\small Codice - Dichiarazione dati esterni}

{\small \textcolor{comment_c}{! Sezione dati;}

\structure{\small DATA}:

Risorsa = \structure{@FILE}('mix\_ottimo\_data.ldt');

Disponibilita = \structure{@FILE}('mix\_ottimo\_data.ldt');

Prodotto = \structure{@FILE}('mix\_ottimo\_data.ldt');

Profitto = \structure{@FILE}('mix\_ottimo\_data.ldt');

Assorbimento = \structure{@FILE}('mix\_ottimo\_data.ldt');

\structure{\small ENDDATA}}

\end{exampleblock}

\lyxframeend{}\lyxframe{Esempio lettura dati esterni}
\begin{exampleblock}
{\small File dati esterni - mix\_ottimo\_data.ldt}

{\small \textcolor{comment_c}{! I nomi delle risorse;}\\
row\_mat labour market \textasciitilde{}\\
\textcolor{comment_c}{! Le disponibilita' delle risorse;}\\
2200 320 100 \textasciitilde{}\\
\textcolor{comment_c}{! I nomi dei prodotti;}\\
prod1 prod2 \textasciitilde{}\\
\textcolor{comment_c}{! I profitti unitari dei prodotti;}\\
120 40 \textasciitilde{}\\
\textcolor{comment_c}{! Matrice dei coefficienti tecnologici;}\\
40 20 \textcolor{comment_c}{! Materie prime;}\\
~~8~~ 2 \textcolor{comment_c}{! Lavoro;}\\
~~1~~ 1 \textcolor{comment_c}{! Mercato; }
}
\end{exampleblock}

\lyxframeend{}

\lyxframeend{}\section{Per approfondire}


\lyxframeend{}\lyxframe{Per approfondire}

\beamertemplatebookbibitems
\begin{thebibliography}{References}
\bibitem{HillierLiebermanS1C3}F.S. Hillier, G. J. Lieberman.\newblock
\textit{Ricerca Operativa}.\newblock Supplemento 1 al Capitolo 3:
The LINGO modeling language. McGraw Hill, 2010.

\bibitem{Schrage2006} L. Schrage.\newblock \emph{Optimization Modeling
with LINGO.} \newblock LINDO Systems, Inc., 2006

\bibitem{LingoUserManual} LINDO Systems Inc.\newblock \emph{LINGO
User's guide.} \newblock LINDO Systems, Inc., 2011

\end{thebibliography}

\lyxframeend{}


\end{document}
