\documentclass{beamer}
\def\presentationtype{1}
\input{../template/savoia_benincasa}
\setbeamertemplate{enumerate item}{Es. \nolezione.\presentationtype.\arabic{enumi})}

\begin{document}

\generatitolo

\def\rombo{$$
\begin{array}{cccrcrcc}
	\max z	&=	& 	&5 x_{1}	&+&5 x_{2}\\
	\mbox{s.t.}	&	& & x_{1}	&-& x_{2}	&\leq	& 5\\
				&	& & x_{1}	& &     	&\leq	&10\\
				&	& &			&+& x_{2}	&\leq	&10\\
				&	&-& x_{1}	&+& x_{2}	&\leq	& 5\\
	&	&\multicolumn{4}{c}{x_{1} \geq 0,\ x_{2}\geq 0}
\end{array}
$$}

\begin{frame}%[allowframebreaks]
{Esercizi}

Risolvere il seguente problema di PL con GNU MathProg
utilizzando il modello del problema del mix ottimo
di produzione e scrivendone i dati sullo stesso
file:

\begin{enumerate}
{\footnotesize \item \rombo}
\end{enumerate}

%~\\~\\Caricare la soluzione sulla piattaforma di e-learning.
\end{frame}

\end{document}
