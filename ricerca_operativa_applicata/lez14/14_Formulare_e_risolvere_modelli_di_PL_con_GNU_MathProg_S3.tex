\documentclass{beamer}
\usepackage{booktabs}
\def\presentationtype{3}
\input{../template/savoia_benincasa}
\setbeamersize{description width=10pt}

\setbeamertemplate{enumerate item}{Es. \nolezione.\presentationtype.\arabic{enumi})}
\begin{document}

\generatitolo
\begin{frame}%[allowframebreaks]
{Esercizi}

\begin{enumerate}
\item
  Si determini un modello di programmazione lineare per
  il seguente problema e lo si implementi in GNU MathProg.
\end{enumerate}

\begin{block}{}
Un'azienda produce mangimi per animali a partire da quattro prodotti grezzi (orzo, avena, sesamo, arachidi). Le proteine e i grassi contenuti per unità nei materiali grezzi, insieme al costo unitario, sono riportati in tabella.%~\ref{tab:istanza}

Si vuole determinare la composizione di una mistura alimentare di minimo costo soddisfacente le esigenze nutritive. 
\end{block}
\end{frame}

\begin{frame}%[allowframebreaks]
{Esercizi}

\begin{table}
\begin{tabular}{c|cccc|c}\toprule
&\bf Orzo&\bf Avena&\bf Sesamo&\bf Arachidi&\parbox{1.5cm}{{\bf \begin{center}Requisiti nutritivi\end{center}}}\\\midrule
\bf Proteine & 12& 12 & 40 & 60 & 20 \\
\bf Grassi   &  2&  6 & 12 &  2 &  5 \\\midrule
\bf Costi    & 24& 30 & 40 & 50 &    \\\bottomrule
\end{tabular}
\caption{Composizione degli alimenti, costo e requisiti nutritivi}
\label{tab:istanza}
\end{table}

% Caricare la soluzione sulla piattaforma di e-learning.
\end{frame}

\end{document}
