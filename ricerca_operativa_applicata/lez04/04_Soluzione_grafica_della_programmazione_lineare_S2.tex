\documentclass{beamer}
\usepackage{booktabs}
\def\presentationtype{2}
\input{../template/savoia_benincasa}
\setbeamersize{description width=10pt}

\setbeamertemplate{enumerate item}{Es. \nolezione.\presentationtype.\arabic{enumi})}
\begin{document}

\generatitolo

\begin{frame}{\esercizi}
Risolvere su carta, evidenziando 
\begin{itemize}
 \item regione ammissibile
 \item gradiente della funzione obiettivo
 \item vertice/segmento ottimo
 \item curva isoprofitto passante per l'ottimo
\end{itemize}
i seguenti problemi di PL:
\end{frame}

\begin{frame}[allowframebreaks]{\esercizi}
%\begin{listaesercizi}
\begin{enumerate}
% massimo, ammissibile, limitato, vertice, non degenere
\item
\[\begin{array}{crrrcr}
\max z=& 20 x_1 &+& 30 x_2\\
{\rm s.t.} & x_1 & &       & \leq &  60 \\
           &     & &   x_2 & \leq &  50  \\
           & x_1 &+& 2 x_2 & \leq & 120  \\
           \multicolumn{6}{c}{x_1 \geq 0,\ x_2 \geq 0}
\end{array}%
\]%

\framebreak

% massimo, ammissibile, limitato, segmento
\item
\[\begin{array}{crrrcr}
\max z=& 15 x_1 &+& 30 x_2\\
{\rm s.t.} & x_1 & &       & \leq &  60 \\
           &     & &   x_2 & \leq &  50  \\
           & x_1 &+& 2 x_2 & \leq & 120  \\
           \multicolumn{6}{c}{x_1 \geq 0,\ x_2 \geq 0}
\end{array}%
\]%


% massimo, ammissibile, limitato, vertice, degenere
\item
\[\begin{array}{crrrcr}
\max z=& 20 x_1 &+& 30 x_2\\
{\rm s.t.} & x_1 & &       & \leq &  60 \\
           &     & &   x_2 & \leq &  50  \\
           & x_1 &+& 2 x_2 & \leq & 120  \\
           & x_1 &+&   x_2 & \leq &  90  \\
           \multicolumn{6}{c}{x_1 \geq 0,\ x_2 \geq 0}
\end{array}%
\]%
%\end{listaesercizi}
\end{enumerate}

\framebreak

Verificare con GeoGebra la correttezza del proprio lavoro.
\end{frame}

\end{document}