\documentclass{beamer}
\usepackage{booktabs}
\def\presentationtype{2}
\input{../template/savoia_benincasa}

\setbeamersize{description width=10pt}

\begin{document}

\generatitolo

\begin{frame}{Il problema dell'assegnamento -- problema}

\modelbox{Problema $\mathcal{P}$}{
Ci sono un certo numero di lavoratori ed un ugual numero di attivit\`a da svolgere.
Ogni lavoratore pu\`o svolgere qualsiasi attivit\`a, con un costo che pu\`o variare a seconda dell'assegnazione lavoratore-compito.
Si vogliono realizzare tutte le attivit\`a, assegnando ad ogni lavoratore una sola attivit\`a e un'attivit\`a ad ogni lavoratore, in modo tale che il costo totale dell'assegnamento sia minimizzato.
Il costo totale della assegnamento di tutte le attività \`e pari alla somma dei costi per ogni lavoratore %
}
\end{frame}


\begin{frame}{Il problema dell'assegnamento -- istanza}

\modelbox{Istanza $\mathcal{I}$}{
Il numero di lavoratori e di attivit\`a da svolgere: \[m\]

I costi dell'assegnazione del lavoratore $i$-esimo al compito $j-$esimo:
\[c_{ij},\text{ per }\ i=1,2,\ldots,m\text{ e }j=1,2,\ldots,m\]
}
\end{frame}

\begin{frame}{Il problema dell'assegnamento -- modello}
\modelbox{Modello matematico -- cosa cerchiamo}{
\tiny{
Trovare i valori
\[x_{ij}\in\{0,1\}\text{ per }\ i=1,2,\ldots,m\text{ e }j=1,2,\ldots,m\]
 tali che 
  \[\sum_{j=1}^{m} x_{ij}=1\quad\text{ per }\ i=1,2,\ldots,m\]
  \[\sum_{i=1}^{m} x_{ij}=1\quad\text{ per }\ j=1,2,\ldots,m\]
  \[z(\vec{x}) = \sum_{i=1}^{m}\sum_{j=1}^{m} c_{ij} x_{ij}\quad\text{ sia minima}\]}}
\end{frame}

\begin{frame}{Il problema dell'assegnamento -- modello}
\modelbox{Modello matematico -- come lo scriveremo}{
\scriptsize{
 \[\min z(\vec{x}) = \sum_{i=1}^{m}\sum_{j=1}^{m} c_{ij} x_{ij}\quad\quad\quad\]

  \[\quad\quad\sum_{j=1}^{m} x_{ij}=1\quad i=1,2,\ldots,m\]
  \[\quad\quad\sum_{i=1}^{m} x_{ij}=1\quad j=1,2,\ldots,m\]
  \[\quad\quad x_{ij}\in\{0,1\} \quad i=1,2,\ldots,m\ ,\ \ j=1,2,\ldots,m\]
  
 }
}
\end{frame}

\begin{frame}{Il problema dell'assegnamento -- soluzioni}
L'insieme base delle soluzioni \`e
  \[\mathcal{F}^\prime \equiv \{0,1\}^{m^2} \]
  
L'insieme delle soluzioni contiene ``solo' $m!$ soluzioni realizzabili,
che esploreremo con l'algoritmo della ricerca esaustiva
\end{frame}

\begin{frame}[allowframebreaks]{Il problema dell'assegnamento -- algoritmo}

Si codificano le soluzioni in modo diverso da quanto
descritto nel modello matematico delle slide precedenti,
nelle quali le soluzioni erano elementi di una matrice
$m\times m$ i cui elementi sono o 0 o 1

Si utilizza un vettore $\vec{a} \in \mathbb{Z}^m$,
in modo che $a_i = j$ rappresenta l'assegnamento
del lavoratore $i$-esimo al compito $j$-esimo.

Le componenti del vettore $\vec{a}$ sono numeri
interi nell'intervallo $[1,2,\ldots,m]$ e una soluzione
\`e realizzabile se tutte le componenti sono
differenti tra loro.

Le soluzioni realizzabili sono tutte e sole le permutazioni
del vettore $[1, 2,\ldots,m]$. Il numero di permutazioni
\`e $m!$.

Occorre calcolare la funzione obiettivo per ogni soluzione
realizzabile e considerare quella cui compete il minimo
per risolvere il problema.

\end{frame}

\begin{frame}{Il problema dell'assegnamento -- esempio}

Si consideri la seguente instanza

il numero di lavoratori e di attivit\`a da svolgere: $m \gets 4$

i costi dell'assegnazione
%del lavoratore $i$-esimo al compito $j-$esimo
come da tabella:

\centering\small{
\begin{tabular}{crrrr} \toprule
	& \multicolumn{4}{c}{Costo [ore]}\\
\cmidrule(r){2-5}
Lavoratore & Compito 1 & Compito 2 & Compito 3 & Compito 4\\ \midrule
1 & 14 &   5 & 8 &   7 \\
2 &   2 & 12 & 6 &   5 \\
3 &   7 &   8 & 3 &   9 \\
4 &   2 &   4 & 6 & 10 \\ \bottomrule
\end{tabular}
}

\end{frame}

\begin{frame}{Il problema dell'assegnamento -- esempio}

\modelbox{Modello matematico -- istanza}{
\tiny{
 \[\min z(\vec{x}) =  14 x_{11} +5x_{12} + 8 x_{13} + 7 x_{14} + 2 x_{12} + \cdots + 10 x_{44}\]
%
\[ x_{11} + x_{12} + x_{13} + x_{14} = 1 \]
\[ x_{21} + x_{22} + x_{23} + x_{24} = 1 \]
\[ x_{31} + x_{32} + x_{33} + x_{34} = 1 \]
\[ x_{41} + x_{42} + x_{43} + x_{44} = 1 \]
\[ x_{11} + x_{21} + x_{31} + x_{41} = 1 \]
\[ x_{12} + x_{22} + x_{32} + x_{42} = 1 \]
\[ x_{13} + x_{23} + x_{33} + x_{43} = 1 \]
\[ x_{14} + x_{24} + x_{34} + x_{44} = 1 \]
%
 \[\quad\quad x_{11}\in\{0,1\}, x_{12}\in\{0,1\}, \ldots, x_{44}\in\{0,1\}\]
 }
}

\end{frame}

\begin{frame}

\modelbox{Soluzioni ammissibili e valore della funzione obiettivo}{
{\footnotesize 
\begin{tabular}{llll}
$z\left([1234]\right) = 20$ & $z\left([1243]\right) = 23$ & $z\left([1324]\right) = 13$ & $z\left([1342]\right) = 18$ \\
$z\left([1432]\right) = 27$ & $z\left([1423]\right) = 25$ & $z\left([2134]\right) = 29$ & $z\left([2143]\right) = 32$ \\
$z\left([2314]\right) = 21$ & $z\left([2341]\right) = 19$ & $z\left([2431]\right) = 28$ & $z\left([2413]\right) = 33$ \\
$z\left([3214]\right) = 21$ & $z\left([3241]\right) = 19$ & $z\left([3124]\right) = 22$ & $z\left([3142]\right) = 27$ \\
$z\left([3412]\right) = 28$ & $z\left([3421]\right) = 21$ & $z\left([4231]\right) = 21$ & $z\left([4213]\right) = 26$ \\
$z\left([4321]\right) = 14$ & $z\left([4312]\right) = 21$ & $z\left([4132]\right) = 29$ & $z\left([4123]\right) = 27$ \\
\end{tabular}}}

La soluzione ottima \`e l'assegnamento del lavoratore
1al compito 1, del lavoratore 2 al compito 3,
del lavoratore 3 al compito 2 e del lavoratore 4 al
compito 4.

Il valore della soluzione ottima \`e $z\left([1324]\right) = 13$
\end{frame}

\end{document}
