\documentclass{beamer}
\usepackage{booktabs}
\def\presentationtype{3}
\input{../template/savoia_benincasa}

\setbeamersize{description width=10pt}

\begin{document}

\generatitolo

\begin{frame}{\esercizi}

Si consideri la seguente instanza

il numero di lavoratori e di attivit\`a da svolgere: $m \gets 4$

i costi dell'assegnazione
%del lavoratore $i$-esimo al compito $j-$esimo
come da tabella:

\centering\small{
\begin{tabular}{crrrr} \toprule
	& \multicolumn{4}{c}{Costo}\\
\cmidrule(r){2-5}
Lavoratore & Compito 1 & Compito 2 & Compito 3 & Compito 4\\ \midrule
1 & 22 &  18 & 30 &  18 \\
2 & 18 & --- & 27 &   22 \\
3 & 16 &   22 & --- &  14 \\
4 & 21 &  --- & 25 & 28 \\ \bottomrule
\end{tabular}
}

--- assegnamento non realizzabile
\end{frame}

\begin{frame}{\esercizi}

\begin{itemize}
\item Come gestire gli assegnamenti non realizzabili (indicati con ---)?
\item Si scriva il modello matematico istanziato per il problema in esame 
\item Si generino le soluzioni realizzabili e si determini la soluzione ottima del problema
\end{itemize}
\end{frame}

\end{document}
