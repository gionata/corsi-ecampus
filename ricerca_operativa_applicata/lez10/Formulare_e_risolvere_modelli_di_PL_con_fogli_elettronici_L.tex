\documentclass{beamer}

\usepackage[italian]{babel}
%
% Hyperref info for PDF
%
\usepackage{hyperref}

\definecolor{ecampus@engred}{RGB}{227,28,23}
\setbeamercolor{normal text}{fg=black,bg=}
\setbeamercolor{alerted text}{fg=red}
\setbeamercolor{example text}{fg=green!50!black}
\setbeamercolor{structure}{fg=ecampus@engred,bg=}
\usetheme{default}
\useinnertheme{rounded}%{circles}
\setbeamertemplate{blocks}[default]
\setbeamercolor{block title}{bg=}
\setbeamercolor{block body}{bg=}
\usepackage{microtype}
\usepackage[utf8]{inputenc}
\usepackage[T1]{fontenc}
%\usepackage[osfss]{libertine}
\usepackage{palatino}
\usepackage[scaled=.77]{beramono}
\usepackage{booktabs}
% \usepackage{attachfile}

% Toglie l'enorme spazione prima di description
%\setbeamersize{description width=0pt}

\usepackage{tikz}
\usetikzlibrary{decorations,arrows,shapes,backgrounds,matrix,positioning}
\usepackage{multirow}
\usepackage{verbatim}
\usepackage{eurosym}
\usepackage{pgfpages}
\title{\titolo}
\author{Gionata Massi}
\date{\today}

\setbeamercolor*{frametitle}{fg=black}
\setbeamercolor*{title}{fg=black}
\setbeamerfont{frametitle}{shape=\scshape,family=\rmfamily,size=\large,series=\bfseries}

\setbeamersize{}
\makeatletter
\newcommand\thefontsize[1]{{#1 The current font size is: \f@size pt\par}}
\makeatother

\tikzstyle{nicebox}=[draw=gray!100, fill=blue!10, very thick,
rounded corners, rectangle, inner sep=4pt, inner ysep=16pt]
\tikzstyle{niceboxtitle}=[draw=gray!100, fill=white, text=black,
rounded corners, very thick, rectangle]
\newcommand\nicebox[2]{
	{\centering
		\begin{tikzpicture}
		\node [nicebox](box){
			\begin{minipage}{0.95\textwidth}\centering
			\begin{minipage}{0.95\textwidth}
			#2
			\end{minipage}\end{minipage}};
		\node[niceboxtitle, right=10pt] at (box.north west)
		{\small\textbf{#1}};
		\end{tikzpicture}\par}
}

\tikzstyle{modelbox}=[draw=structure!100, fill=white!50, very thick,
rounded corners, rectangle, inner sep=4pt, inner ysep=16pt, text=blue!50!black]
\tikzstyle{modelboxtitle}=[draw=structure!100, fill=white!50, text=blue!50!black,
rounded corners, very thick, rectangle]
\newcommand\modelbox[2]{
	{\centering
		\begin{tikzpicture}
		\node [modelbox](box){
			\begin{minipage}{0.95\textwidth}\centering
			\begin{minipage}{0.95\textwidth}
			#2
			\end{minipage}\end{minipage}};
		\node[modelboxtitle, right=10pt] at (box.north west)
		{\small\textbf{\mbox{#1}}};
		\end{tikzpicture}\par}
}




\begin{document}


\section{I fogli elettronici e la PL}

\begin{frame}{Fogli elettronici}
\begin{itemize}
\item I fogli elettronici (fra cui \emph{Microsoft Excel} e \emph{OpenOffice Calc}) sono
strumenti molto diffusi che permettono analizzare e risolvere
problemi di PL di piccole dimensioni.

\item Un foglio elettronico pu\`o essere utilizzato per:
\begin{itemize}
	\item visualizzare i dati;
	\item analizzare rapidamente le possibili soluzioni.
\end{itemize}

\item Il Risolutore incluso in \emph{Excel}, opportunamente configurato,
applica il metodo del simplesso per determinare una soluzione ottima
per l'instanza del modello di PL. Per altre tipologie di problemi di
ottimizzazione utilizza altri algoritmi.

\item \emph{OpenOffice} (o \emph{LibreOffice}) \emph{Calc} utilizzano il metodo del
simplesso per risolvere problemi di PL.
\end{itemize}
\end{frame}

\begin{frame}{Programmi Lineari}
\begin{itemize}
\item Quando si inserisce un programma lineare su un foglio elettronico
\`e necessario conoscere i parametri del modello e distinguere:
\begin{enumerate}
\item le decisioni che devono essere prese (\alert{variabili decisionali});
\item le restrizioni sul valore che pu\`o essere assunto delle variabili
	decisionali (\alert{vincoli});
\item la misura di prestazione complessiva del sistema in funzione delle decisioni (\alert{funzione obiettivo})
\end{enumerate}
\end{itemize}
\end{frame}

\section{Esempio}

\begin{frame}[allowframebreaks]{Problema di esempio}
\fontsize{7}{8.4}{
\begin{block}{Mix ottimo di produzione}
Un'azienda di produzione vuole determinare il tasso di produzione mensile di due prodotti
in modo da massimizzare il profitto netto totale,  sapendo che:
 \begin{itemize}
  \item per produrre un quintale di prodotto 1 occorrono 40 quintali di materia prima e 8 ore di lavoro;
  \item per produrre un quintale di prodotto 2 occorrono 20 quintali di materia prima e 2 ore di lavoro;
  \item il commerciale ha stabilito che la produzione totale mensile non pu\`o superare 100 quintali;
  \item la disponibilit\`a mensile di materia prima \`e di 2200 quintali e quella di lavoro di 320 ore;
  \item il profitto netto per la vendita dei prodotti 1 e 2 sia rispettivamente 120 e 40 euro.
 \end{itemize}
\end{block}
}

\framebreak

\modelbox{Modello}{%
\small{%
$$\begin{array}{crrrlcr}
\max z=& 120 x_1 &+& 40 x_2    &           & (0) & [\text{profitto}]\\
{\rm s.t.} & 40 x_1 &+& 20 x_2 & \leq 2200 & (1) & [\text{materie prime}]\\
           &  8 x_1 &+&  2 x_2 & \leq 320  & (2) & [\text{lavoro}]\\
           &    x_1 &+&    x_2 & \leq 100  & (3) & [\text{mercato}]\\
           \multicolumn{5}{c}{x_1 \geq 0,\ x_2 \geq 0} & & [\text{non negativit\`a}]
\end{array}$$}}
\end{frame}

\section{Formulare il problema}

\begin{frame}{Foglio di calcolo}
\begin{itemize}
\item Nelle diapositive seguenti saranno presentate le schermate di
   \emph{LibreOffice Calc}, un foglio elettronico gratuito scaricabile da
   \href{http://it.libreoffice.org}{http://it.libreoffice.org},
   disponibile gratuitamente per numerose architetture hw/sw.
   
\item Le informazioni generali vanno bene per qualsiasi foglio elettronico
	mentre quelle sul risolutore sono specifiche per \emph{Calc} versione 4.
	
\item Per altri fogli elettronici controllare la documentazione appropriata.
\end{itemize}
\end{frame}


\begin{frame}{Inserire i parametri del problema}
\only<1>{
Per rappresentare un problema di PL servono
i suo elementi strutturali (variabili decisionali,
vincoli, funzione obiettivo) e i parametri del problema
specifico (istanza):

\begin{itemize}
\item un insieme di celle conterr\`a i valori delle variabili
	decisionali;
\item una cella del foglio elettronico conterr\`a il valore
	della funzione obiettivo calcolata in funzione del valore
	delle variabili decisionali;
\item un insieme di celle conterra i parametri del problema
	($\matrA,\ \vecB,\ \vecC,\ d$);
\item un gruppo di celle conterr\`a il valore calcolato dei
	valori dei termini di sinistra di ogni vincolo.
\end{itemize}
}
\includegraphics<2>[height=0.85\textheight]{./img/schermata1}
\end{frame}

\begin{frame}{Calcolare funzione obiettivo e primi membri}
\only<1>{
Abbiamo bisogno di calcolare la funzione obiettivo (lineare)
e il valore del primo membro di ogni vincolo lineare
(risultato dei prodotti scalari
$\vecC^T \vecX=z$ e $A_i \vecX \leq {\vecB}_i,\ i \in \{1, 2,\ldots,m\}$).

\begin{itemize}
\item \emph{Calc} ed \emph{Excel} offrono la funzione \emph{MATR.SOMMA.PRODOTTO}
	che permette di effettuare il prodotto scalare;
\item tale funzione vuole entrambi i vettori definiti come vettore riga
	o come vettore colonna;
\item per comodit\`a nel foglio di calcolo sono stati introdotti anche
	i valori delle variabili ausiliarie (slack).
\end{itemize}
}
\includegraphics<2>[height=0.85\textheight]{./img/schermata2}
\end{frame}

\begin{frame}{Verifica del corretto inserimento delle formule}
\only<1>{
\begin{itemize}
\item \`E consigliabile verificare di aver inserito le formule
correttamente nel foglio di calcolo;
\item un modo \`e quello di inserire dei valori
per le variabili decisionali e controllare che i risultati
siano quelli attesi.
\end{itemize}
}
\includegraphics<2>[height=0.85\textheight]{./img/schermata3}
\end{frame}

\section{Risolvere il problema}

\begin{frame}{Il risolutore}
\only<1>{
\begin{itemize}
\item Il risolutore \`e raggiungibile dal men\`u strumenti;
\item richiede di inserire i dati sul problema di PL:
\begin{itemize}
\item quale cella contiene il valore della funzione obiettivo;
\item la direzione di ottimizzazione ($\max$, $\min$ oppure
la minimizzazione dello scarto rispetto ad un certo valore);
\item le espressioni dei vincoli (primo termine, secondo termine,
	relazione tra loro).
\end{itemize}
\item La risoluzione di un modello di PL presuppone che tutte
	le variabili decisionali siano non negative.
\end{itemize}
}
\includegraphics<2>[height=0.85\textheight]{./img/schermata4}
\includegraphics<3>[height=0.85\textheight]{./img/schermata5}
\includegraphics<4>[height=0.85\textheight]{./img/schermata6}
\end{frame}


\begin{frame}{Il risolutore}
\only<1>{
\begin{itemize}
\item Il pulsante ``Risolvi'' permette di calcolare
	la soluzione ottima del problema utilizzando il
	metodo del simplesso.
\item Confrontando il valore della soluzione ottenuta
	col foglio di calcolo rispetto a quelle della
	risoluzione grafica e mediante il metodo del
	simplesso tableau notiamo che otteniamo lo stesso
	risultato:
	$\vecX^{T\star}=\left[\begin{array}{ccccc}25&60&0&0&15\end{array}\right]$
	in corrispondenza del quale $z^\star = 5400$.
	
\item Potremmo provare a modificare i valori per
	simulare decisioni diverse o variazioni dei
	parametri del modello.
\end{itemize}
}
\includegraphics<2>[height=0.85\textheight]{./img/schermata7}
\includegraphics<3>[height=0.85\textheight]{./img/schermata8}
\end{frame}

\section{Per approfondire}

\begin{frame}{Altro informazioni}
\begin{itemize}
\item Il risolutore di \emph{Excel} permette di effettuare
	anche l'analisi post-ottimale.
	
\item Esistono varie estensioni per aggiungere
	funzionalit\`a al risolutore di \emph{Calc}
	e di \emph{Excel}.
	
\item Entrambi i risolutori possono lavorare anche con
variabili intere (PLI e PLB).
	
\item \emph{Excel} e LibreoOffice con l'estensione
	\emph{nlpsolver} permettono di risolvere
	problemi non lineari.
\end{itemize}
\end{frame}

\begin{frame}{Per approfondire}
\beamertemplatebookbibitems
\begin{thebibliography}{References}
\bibitem{HillierLiebermanS1C3}F.S. Hillier, G. J. Lieberman.\newblock
\textit{Ricerca Operativa}. Capitolo 2.\newblock McGraw Hill, 2010.

\bibitem{ooo3solverug}\href{http://wiki.openoffice.org/wiki/Documentation/OOo3\_User\_Guides/Calc\_Guide/Solver}{Using the solver} \newblock \textit{OpenOffice.org} \newblock Pagina wiki consultata il 3 settembre 2013.

%\bibitem{Schrage2006} L. Schrage.\newblock \emph{Optimization Modeling
%with LINGO.} \newblock LINDO Systems, Inc., 2006

%\bibitem{LingoUserManual} LINDO Systems Inc.\newblock \emph{LINGO

%User's guide.} \newblock LINDO Systems, Inc., 2011
\end{thebibliography}
\end{frame}

\end{document}
