\documentclass[italian,12pt]{article}
\usepackage[T1]{fontenc}
\usepackage[latin9]{inputenc}
\usepackage{amsmath}
\usepackage{amssymb}
\usepackage{amsfonts}
\usepackage[italian]{babel}
\usepackage[a4paper, margin=2.5cm,noheadfoot]{geometry}
\begin{document}

\pagestyle{empty}

\section*{Esercitazione n. 1 - Ottimizzazione non vincolata}

\subsection*{Funzioni di una variabile}

Determinare i punti di ottimo, negli intervalli dati o in $\mathbb{R}$, e calcolare il valore assunto dalla funzione in tali punti.

\begin{enumerate}
	\item 
	\begin{minipage}{0.70\textwidth}
		\begin{gather*}
			\max z = x (5\pi - x)\\
			0 \leq x \leq 20
		\end{gather*}
	\end{minipage}
	\begin{minipage}{0.20\textwidth}
		\begin{flushright}
			R: $x^* = \frac{5}{2}\pi$\\
			$z^* = 61.69$
		\end{flushright}
	\end{minipage}
	
	\item 
	\begin{minipage}{0.70\textwidth}
		\begin{gather*}
			\max z = |x^2 - 8|\\
			-4 \leq x \leq 4
		\end{gather*}
	\end{minipage}
	\begin{minipage}{0.20\textwidth}
		\begin{flushright}
			R: $x^* = \pm 4$\\
			$x^*=0$\\
			$z^* = 8$
		\end{flushright}
	\end{minipage}
	
	\item 
	\begin{minipage}{0.70\textwidth}
		\begin{gather*}
			\min z = \begin{cases}
			1 & x=0\\
			x & 0 < x \leq 1
			\end{cases}\\
			0 \leq x \leq 1
		\end{gather*}
	\end{minipage}
	\begin{minipage}{0.20\textwidth}
		\begin{flushright}
			R: $\nexists \min$
		\end{flushright}
	\end{minipage}
	
	\item 
	\begin{minipage}{0.70\textwidth}
		\begin{gather*}
			\max z = x e^{-x^2}
		\end{gather*}
	\end{minipage}
	\begin{minipage}{0.20\textwidth}
		\begin{flushright}
			R: $x^* = \frac{1}{\sqrt{2}}$\\
			$z^* = 0.429$
		\end{flushright}
	\end{minipage}
	
	\item 
	\begin{minipage}{0.70\textwidth}
		\begin{gather*}
			\max z = x \sin(4 x)\\
			0 \leq x \leq 3
		\end{gather*}
		\begin{center}
		Studiare il grafico
		\end{center}
	\end{minipage}
	\begin{minipage}{0.20\textwidth}
		\begin{flushright}
			R: $x^* = \frac{7}{8}\pi$\\
			$z^* = -2.75$
		\end{flushright}
	\end{minipage}
	
	\item 
	Si trovino tutti gli ottimi locali e globali di $f(x) = x^3-6x^2+9x+6$ nell'intervallo $[-1,5]$.
	
	\item
	Si trovino tutti gli ottimi locali e globali di $f(x) = x^4-4x^3+6x^2-4x+1$ nell'intervallo $[0,\infty)$.
	
	\item
	Si trovino tutti gli ottimi locali e globali di $f(x) = x+x^{-1}$ nell'intervallo~$[5,10]$.
	
	\item
	Si mostri che $f(x) = x^3 - 6 x^2 + 9 x + 6$ � strettamente concava nell'intervallo~$(-\infty,2)$ e
	strettamente convessa in~$(2,+\infty)$.
	
	\item
	Si determinino gli intervalli nei quali $f(x) = x + 4 x^{-1}$ � concava o convessa.
\end{enumerate}

\end{document}