\documentclass{beamer}
\def\presentationtype{0}
\input{../template/savoia_benincasa}
\setbeamertemplate{caption}[numbered]

\begin{document}

\generatitolo


\section{Ciclo semaforico ottimo}

\begin{frame}{Ciclo semaforico}

Tratto da: Gennaro Improta, \textit{Programmazione Lineare.} \textsl{Capitolo quarto -- Impostazione di modelli in programmazione lineare}, pp. 92--100. Edizioni Scientifiche Italiane, 2004
\\~\\

Si consideri un incrocio stradale in cui un insieme di semafori dirige il traffico.
Il flusso di attraversamento dell'incrocio da parte dei veicoli provenienti da una stessa corsia pu\`o essere modellizzato come in fig.\ref{fig:webster}.
\end{frame}

\begin{frame}{Convenzione sui tempi di verde I}
\begin{figure}
\includegraphics[height=0.75\textheight]{img/webster}
\caption{Grafico di Webster}\label{fig:webster}
\end{figure}
\end{frame}

\begin{frame}{Convenzione sui tempi di verde II}

L'andamento reale dei veicoli, rappresentato molto approssimativamente dalla linea continua, pu\`o essere semplificato utilizzando un'onda quadra di uguale area. In tal modo si semplifica il fenomeno
considerando l'attraversamento della linea di stop da parte di un flusso costante per un tempo,
detto di verde effettivo, minore del tempo di ciclo meno il tempo di rosso.
\end{frame}

\begin{frame}{Convenzione sui tempi di verde III}
\begin{description}
	\item[$G_s$] la durata del verde semaforico;
	\item[$G_e$] la durata del verde effettivo;
	\item[$A$] la durata del giallo;
	\item[$L$] il tempo perso ($L=l_1+l_2$).
\end{description}
\end{frame}

\begin{frame}{Matrice delle fasi}

Si dice \emph{fase} di un ciclo semaforico un intervallo di tempo durante il quale un sottoinsieme di manovre ha verde.
\\~\\
Si pu\`o indicare la composizione del ciclo semaforico mediante una matrice booleana $\matr{A}$ avente un numero di righe pari alle fasi ed un numero di colonne pari agli accessi che concorrono
all'intersezione. Il generico elemento $a_{ij}$ vale 0 se il ramo $j$ non ha verde, 1 altrimenti.
\\~\\
Si dice \emph{durata del ciclo semaforico} il tempo necessario affinch\'e si completino tutte le
diverse del ciclo.
\end{frame}

\begin{frame}[allowframebreaks]{Problema}

Si consideri l'intersezione della fig.~\ref{fig:intersezione} per la quale si vuole progettare un
ciclo semaforico con durate ottime del tempo di verde, ossia che
massimizzino la capacit\`a di smaltimento di veicolo da parte dell'intersezione, noti:
\begin{itemize}
  \item le portate medie in arrivo ($q_j , j=1,2,\ldots,6$);
  \item i flussi di saturazione ($s_j , j=1,2,\ldots,6$);
  \item la composizione delle fasi ($\matr{A}$);
  \item la durata del ciclo semaforico ($C_0$);
  \item altri parametri di progetto.
\end{itemize}

\begin{figure}
\includegraphics[height=0.75\textheight]{img/incrocio_all}
\caption{Intersezione stadale oggetto dello studio}\label{fig:intersezione}
\end{figure}
\end{frame}

\begin{frame}{Problema III}
\begin{table}
 \begin{tabular}{c|cccccc}
  \toprule
  \multicolumn{1}{c}{}& \multicolumn{6}{c}{\centering\textbf{Accessi}}\\
  \textbf{Fasi} & \textbf{1} & \textbf{2} & \textbf{3} & \textbf{4} & \textbf{5} & \textbf{6}\\  
  \midrule
  1 & 1 & 1 & 0 & 1 & 0 & 0\\
  2 & 0 & 0 & 1 & 1 & 0 & 1\\
  3 & 0 & 1 & 0 & 0 & 1 & 1\\
  \bottomrule
 \end{tabular}
  \caption{Matrice delle fasi}\label{tbl:fasi}
\end{table}
In tab.~\ref{tbl:fasi} sono fornite le fasi del ciclo semaforico,
\end{frame}

\begin{frame}{Problema IV}
\begin{figure}
\includegraphics[height=0.75\textheight]{img/incrocio_p1}
\caption{Fase 1}\label{fig:fase1}
\end{figure}
\end{frame}

\begin{frame}{Problema V}
\begin{figure}
\includegraphics[height=0.75\textheight]{img/incrocio_p2}
\caption{Fase 2}\label{fig:fase2}
\end{figure}
\end{frame}

\begin{frame}{Problema VI}
\begin{figure}
\includegraphics[height=0.75\textheight]{img/incrocio_p3}
\caption{Fase 3}\label{fig:fase3}
\end{figure}
\end{frame}

\begin{frame}[allowframebreaks]{Simboli}
\begin{description}
	\item[$F$] insieme delle fasi;
	\item[$M$] insieme delle manovre;\\~
	\item[$i$] indice delle fasi;
	\item[$j$] indice delle manovre;\\~
%	\item[$G_s$] la durata del verde semaforico;
	\item[$G_i$] la durata del verde effettivo;
	\item[$G_{i_{miin}}$] la durata minima del verde effettivo;
%	\item[$A$] la durata del giallo;
	\item[$L_i$] il tempo perso complessivo;
\end{description}
\framebreak
\begin{description}
  \item[$q_j$] le portate medie in arrivo;
  \item[$s_j$] i flussi di saturazione;
  \item[$g_j$] la durata del verde effettivo;
  \item[$\matr{A}$] la composizione delle fasi;
  \item[$C_0$] la durata del ciclo semaforico;
  \item[$N_j$] numero di veicoli che giungono all'accesso $j$ nel tempo $C_0$. ($N_j = q_j \cdot C_0$);
  \item[$p_j$] coefficiente di sicurezza ($\leq 1$) per evitare saturazione;
  \item[$\mu$] moltiplicatore.
  
\end{description}\end{frame}

\section{Costruzione del modello}

\begin{frame}{Insiemi}
  \begin{description}
	  \item[Fasi] insieme delle fasi, indice \alert{$i$};
	  \item[Manovre] insieme delle manovre, indice \alert{$j$};
	  \item[MatriceFasi] insieme $\rm Fasi \times \rm Manovre$, indice \alert{$(i, j)$}.
  \end{description}
\end{frame}

 \begin{frame}{Parametri}
		\begin{description}
			\item[$q_j$] portata media in arrivo;
			\item[$s_j$] flusso di saturazione;
			\item[$p_j$] coefficiente correttivo;
			\item[$G_{i_{min}}$] durata minima del verde effettivo;
			\item[$L_i$] il tempo perso complessivo;
			\item[$C_0$] la durata del ciclo semaforico;
			\item[$b_j$] parametro calcolato: $b_j = \dfrac{q_j C_0}{p_j s_j}$;	
			\item[$a_{ij}$] la matrice delle fasi;
		\end{description}
\end{frame}

\begin{frame}{Variabili decisionali}
\begin{block}{Variabili}
	\begin{description}
		\item[$G_i$] la durata del verde effettivo della fase $i$-esima;
		\item[$\mu$] fattore moltiplicatore;
	\end{description}
\end{block}

  Le variabili sono continue e non negative.

\end{frame}

\begin{frame}[allowframebreaks]{Vincoli}

\textbf{Bilancio tra verde effettivo e verde richiesto}

La durata del verde visto da ogni 

$$ \sum_{\parametro{i} \in \insieme{Specie}} \variabile{x}_{\indice{ij}} \leq \parametro{estensione}_{\indice{j}} \quad \indice{j} \in \insieme{Vasche}$$

\textbf{Disponibilit\`a acqua di rinnovo}

$$ \sum_{\indice{i} \in \insieme{Specie}} \parametro{fabb\_acqua}_{\indice{i}} \variabile{x}_{\indice{ij}} \leq \parametro{disp\_acqua}_{\indice{j}} \quad \indice{j} \in \insieme{Vasche}$$

\framebreak

\textbf{Disponibilit\`a di mangime}

$$ \sum_{\indice{i} \in \parametro{Specie}} \parametro{fabb\_mangime}_{\indice{i}} \cdot
      \left( \sum_{\indice{j} \in \insieme{Vasche}} \variabile{x}_{\indice{ij}} \right) \leq \parametro{disp\_mangime}$$

\textbf{Domanda}

$$ \sum_{\indice{j} \in \insieme{Vasche}} \variabile{x}_{\indice{ij}} \leq \parametro{quota}_{\indice{i}} \quad \indice{i} \in \insieme{Specie}$$

\textbf{Equit\`a}

$$ \frac{\sum\limits_{\indice{i} \in \insieme{Specie}} \variabile{x}_{\indice{i,j}}}{\parametro{estensione}_{\indice{j}}} =
   \frac{\sum\limits_{\indice{i} \in \insieme{Specie}} \variabile{x}_{\indice{i,(j+1)\%|\insieme{Vasche}|}}}{\parametro{estensione}_{\indice{i,(j+1)\%|\insieme{Vasche}|}}}
   \quad \indice{j} \in \insieme{Vasche}$$

\textbf{Fisica realizzabilit\`a (non negativit\`a)}

$$ \variabile{x}_{\indice{ij}} \geq 0 $$
\end{frame}

\begin{frame}{Funzione obiettivo}
Si deve determinare la superficie di ogni impianto da destinare ad ogni specie ittica.

In generale:

$$  \sum_{\indice{i} \in \insieme{Specie}} \parametro{ricavo}_{\indice{i}}\cdot
         \left(\sum_{\indice{j} \in \insieme{Vasche}} \variabile{x}_{\indice{ij}}\right)$$

Il ricavo \`e espresso da una funzione lineare: rispetta le ipotesi di proporzionalit\`a, additivit\`a e divisibilit\`a.
\end{frame}

\end{document}
