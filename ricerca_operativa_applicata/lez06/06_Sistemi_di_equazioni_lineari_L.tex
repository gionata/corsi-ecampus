\documentclass{beamer}
\def\presentationtype{0}
\input{../template/savoia_benincasa}

\begin{document}

\generatitolo

\section{Sistemi di equazioni lineari e matrici}

\begin{frame}[allowframebreaks]{Forma generale di un sistema lineare}
  
  \modelbox{Sistema lineare di $m$ equazioni in $n$ incognite}{%
  \[%
% \left\{\begin{array}{lclccclcr}
%   a_{11} x_1 &+& a_{12} x_2 &+& \cdots &+& a_{1n} x_n &=& b_1\\
%   a_{21} x_1 &+& a_{22} x_2 &+& \cdots &+& a_{2n} x_n &=& b_2\\
%              & &            & & \ddots & &            &\vdots& \vdots\\
%   a_{m1} x_1 &+& a_{m2} x_2 &+& \cdots &+& a_{mn} x_n &=& b_m
% \end{array}\right.%
\left\{
\begin{matrix} 
a_{11}x_1 + a_{12}x_2 +\cdots + a_{1n}x_n = b_1\\
a_{21}x_1 + a_{22}x_2 +\cdots + a_{2n}x_n = b_2\\
\vdots\\
a_{m1}x_1 + a_{m2}x_2 +\cdots + a_{mn}x_n = b_m\\
\end{matrix}
\right.
\]}

  \framebreak
  
  \begin{itemize}
  \item  Il numero $n$ delle incognite \`e detto ordine del sistema

  \item  I numeri $a_{11},\ a_{12}, \ldots,\ a_{mn}$  sono i coefficienti del sistema

  \item  I numeri $b_1,\ b_2, \ldots,\ b_m$ sono i termini noti del sistema
  
  \item  Noti $m$ ed $n$ \`e possibile scrivere un sistema di equazioni lineari usando
  la notazione matriciale
  \end{itemize}
  
  \framebreak
  
  I coefficienti del sistema possono essere raccolti in una tabella di numeri detta \emph{matrice dei coefficienti}

  \modelbox{Matrice dei coefficienti}{%
  \[
  \matr{A} = 
  \begin{bmatrix}\begin{array}{cccc}
  a_{11} & a_{12} & \cdots & a_{1n}\\
  a_{21} & a_{22} & \cdots & a_{2n}\\
         &        & \ddots \\
  a_{m1} & a_{m2} & \cdots & a_{mn}
  \end{array}\end{bmatrix}
  \]
  }

  \framebreak
  
   I termini noti del sistema possono essere raccolti nel vettore colonna detto \emph{vettore dei termini noti}

  \modelbox{Vettore dei termini noti}{%
  \[
  \vec{b} = 
  \begin{bmatrix}\begin{array}{c}
  b_{1}\\
  b_{2}\\
  \vdots \\
  b_{m}
  \end{array}\end{bmatrix}
  \]
  }
  
  \framebreak
  
  Le incognite possono essere rappresentate dal \emph{vettore delle incognite}

  \modelbox{Vettore delle incognite}{%
  \[
  \vec{x} = 
  \begin{bmatrix}\begin{array}{c}
  x_{1}\\
  x_{2}\\
  \vdots \\
  x_{n}
  \end{array}\end{bmatrix}
  \]
  }

  \small{L'intero sistema lineare  di $m$ equazioni in $n$ incognite \`e rappresentato dalla \emph{matrice completa} (o \emph{matrice aumentata}) del sistema}

  \modelbox{Matrice completa}{%
  \[
  [\matr{A}|\vec{b}] = 
  \begin{bmatrix}
  \left.\begin{array}{cccc}
  a_{11} & a_{12} & \cdots & a_{1n}\\
  a_{21} & a_{22} & \cdots & a_{2n}\\
         &        & \ddots \\
  a_{m1} & a_{m2} & \cdots & a_{mn}
  \end{array}\right|
  \begin{array}{c}
  b_{1}\\
  b_{2}\\
  \vdots\\
  b_{m}
  \end{array}
  \end{bmatrix}
  \]
  }
\end{frame}

\section{Prodotto di una matrice per un vettore}

\begin{frame}{Matrici}
\footnotesize{
  \begin{itemize}
  \item In generale una matrice con $m$ righe e $n$ colonne \`e
  una tabella rettangolare di numeri con $m$ righe e $n$
  colonne
  
  \item L'insieme delle matrici con $m$ righe e $n$ colonne ($m$x$n$) a coefficienti
  reali \`e indicato con $M_{m,n}(\mathbb{R})$

  \item Sia $\matr{A} \in M_{m,n}(\mathbb{R})$ una matrice $m$x$n$
  \begin{description}
   \item[$a_{ij}$] \`e il coefficiente di posto $(i, j)$ della matrice $\matr{A}$
   \item[$i$]  \`e l'indice di riga
   \item[$j$] \`e l'indice di colonna
   \item[$A_i$] \`e la riga $i$-esima; $A_i=\left[\begin{array}{cccc}a_{i1}&a_{i2}&\cdots&a_{in}\end{array}\right]$
   \item[$A^j$] \`e la colonna $j$-esima; $A^j=\left[\begin{array}{c}a_{1j}\\a_{2j}\\\vdots\\a_{mj}\end{array}\right]$
  \end{description}
  \end{itemize}
}
\end{frame}

\begin{frame}[allowframebreaks]{Prodotto di una matrice per un vettore}
Si definisce il prodotto di una matrice $\matr{A} \in M_{m,n}(\mathbb{R})$
per un vettore $x \in \mathbb{R}^n$ come:

\modelbox{Prodotto di una matrice per un vettore}{%
  \small\[
  \matr{A} \vec{x} = 
  \begin{bmatrix}\begin{array}{cccc}
  a_{11} & a_{12} & \cdots & a_{1n}\\
  a_{21} & a_{22} & \cdots & a_{2n}\\
         &        & \ddots \\
  a_{m1} & a_{m2} & \cdots & a_{mn}
  \end{array}\end{bmatrix}
  \begin{bmatrix}\begin{array}{c}
  x_{1}\\
  x_{2}\\
  \vdots \\
  x_{n}
  \end{array}\end{bmatrix} =
  x_{1} A^1 + x_{2} A^2 + \cdots + x_{n} A^n 
  \]
}

\framebreak

 che può essere riscritto come il vettore 

\modelbox{Prodotto di una matrice per un vettore}{%
  \small\[
  \matr{A} \vec{x} = 
  \begin{bmatrix}\begin{array}{cccc}
  a_{11} x_{1} +& a_{12} x_{2} +& \cdots +& a_{1n} x_{n}\\
  a_{21} x_{1} +& a_{22} x_{1} +& \cdots +& a_{2n} x_{n}\\
         &        & \ddots \\
  a_{m1} x_{1} +& a_{m2} x_{2} +& \cdots +& a_{mn} x_{n}
  \end{array}\end{bmatrix}
  \]
}
\end{frame}

\begin{frame}{Forma matriciale di un sistema di equazioni lineari}

\scriptsize{
 Dalla precedente definizione di prodotto di una matrice per un vettore abbiamo che il sistema di equazioni lineari

\[
\left\{\begin{array}{lclccclcc}
  a_{11} x_1 &+& a_{12} x_2 &+& \cdots &+& a_{1n} x_n &=& b_1\\
  a_{21} x_1 &+& a_{22} x_2 &+& \cdots &+& a_{2n} x_n &=& b_2\\
             & &            & & \ddots & &            &\vdots& \vdots\\
  a_{m1} x_1 &+& a_{m2} x_2 &+& \cdots &+& a_{mn} x_n &=& b_m
\end{array}\right.
\]

può essere espresso come

\[
\begin{bmatrix}\begin{array}{cccc}
  a_{11} x_{1} +& a_{12} x_{2} +& \cdots +& a_{1n} x_{n}\\
  a_{21} x_{1} +& a_{22} x_{1} +& \cdots +& a_{2n} x_{n}\\
         &        & \ddots \\
  a_{m1} x_{1} +& a_{m2} x_{2} +& \cdots +& a_{mn} x_{m}
\end{array}\end{bmatrix} 
=
  \begin{bmatrix}\begin{array}{c}
  b_{1}\\
  b_{2}\\
  \vdots \\
  b_{m}
  \end{array}\end{bmatrix}
\]

o anche: \begin{center}{$\matr{A} \vec{x} = \vec{b}$}\end{center}
}
\end{frame}

\section{Definizioni e teoremi di Algebra Lineare}

\begin{frame}[allowframebreaks]{Definizioni}

\begin{definition}[soluzione]
   Una soluzione del sistema di equazioni lineari \`e una
   $n$-upla $(v_1, v_2, \ldots, v_n)$ di numeri che sostituiti
   ordinatamente alle incognite $(x_1, x_2, \ldots, x_n)$
   soddisfano  tutte le equazioni del sistema.
\end{definition}

\begin{definition}[combinazione lineare]
  Una combinazione lineare dei vettori $v_1, v_2, \ldots, v_k \in V$
  \`e un vettore $w = \alpha_1 v_1+ \alpha_2 v_2 +\ldots+ \alpha_k v_k\quad\alpha_j \in \mathbb{R} \text{ per } j = 1, 2, \ldots, k$
\end{definition}

\framebreak

\begin{definition}[dipendenza lineare]
  Un insieme di vettori $v_1, v_2, \cdots, v_k \in V$ si  dicono
  linearmente dipendenti se esistono 
  $\alpha_1\in \mathbb{R}, \alpha_2\in \mathbb{R},\ldots, \alpha_k \in \mathbb{R}$
  non tutti nulli tali che
  $\alpha_1 v_1+ \alpha_2 v_2 +\cdots+ \alpha_k v_k = 0 $
\end{definition}

\begin{definition}[lineare indipendenza]
  Un insieme di vettori $v_1, v_2, \cdots, v_k$ si dicono linearmente indipendenti se 
  $\alpha_1 v_1+ \alpha_2 v_2 +\cdots+ \alpha_k v_k = 0$ implica
  $\alpha_1 = \alpha_2 = \cdots = \alpha_k = 0$
\end{definition}

\begin{definition}[base]
  Una base per un insieme di vettori \`e un sottoinsieme di vettori linearmente indipendenti tali che ogni altro vettore dell'insieme può essere espresso come una loro combinazione lineare.
\end{definition}

\begin{definition}[rango]
  Il rango di un insieme di vettori \`e il più grande numero di vettori linearmente indipendenti che si possono scegliere nell'insieme.

  Il rango di riga di una matrice quadrata \`e il rango dell'insieme dei suoi vettori riga. Il rango di colonna di una matrice quadrata \`e il rango dell'insieme dei suoi vettori colonna. 

  Il rango di riga e il rango di colonna di una matrice sono uguali tra loro e sono uguali al numero di pivot (calcolabili con una qualsiasi eliminazione di Gauss) 
\end{definition}

\begin{definition}[singolarit\`a]
  Una matrice \`e detta non singolare se il rango \`e uguale sia al suo numero di righe che al suo numero di colonne, cio\`e possiede tutti i pivot. E detta singolare altrimenti.
\end{definition}

\begin{definition}[sottomatrice]
   Una sottomatrice di una matrice
   $\matr{A} \in M_{m,n}(\mathbb{R})$
   \`e una matrice
   $\matr{A} \in M_{r,s}(\mathbb{R})$
   ottenuta da $\matr{A}$
   rimuovendo $m - r$ righe e $n - s$ colonne
\end{definition}
   
\begin{definition}[minore]
  Un minore \`e una sottomatrice quadrata, cio\`e con $r = s$.
  Il numero $r$ \`e definito ordine del minore
\end{definition}
   
\begin{definition}[minore complementare]  
  Un minore complementare  di una matrice
  $\matr{A} \in M_{m,n}(\mathbb{R})$  \`e un suo minore ottenuto togliendo
  una sola riga e una sola colonna.
  Il minore ottenuto togliendo l'$i$-esima riga e la $j$-esima colonna si indica con
  $A(i,j)$
\end{definition}
   
\begin{definition}[rango di una matrice rettangolare]
  Se una matrice non \`e quadrata il rango \`e il
  massimo ordine di un minore non singolare
\end{definition}
\end{frame}

\begin{frame}[allowframebreaks]{Teoremi}
\begin{theorem}[Rouché-Capelli]
 dato un sistema di equazioni lineari
 $\matr{A}\vec{x} = \vec{b}$, con
 $\matr{A} \in M_{m,n}(\mathbb{R}),\ \vec{x} \in \mathbb{R}^n$ e $\vec{b} \in \mathbb{R}^m$
  esistono soluzioni per il sistema se
  e solo se il rango della matrice completa $[\matr{A}|\vec{b}]$
  è uguale al rango della matrice $\matr{A}$;
  se esistono soluzioni e il rango di $\matr{A}$
  è uguale a $n$, allora la soluzione è unica;
  se esistono soluzioni e il rango di $\matr{A}$
  è minore di $n$, esistono infinite soluzioni
\end{theorem}
il teorema afferma che le soluzioni formano
  un sottospazio affine di dimensione $n - \operatorname{rango}\matr{A}$
\framebreak

\begin{theorem}[Regola di Cramer]
    dato un sistema di equazioni lineari
    $\matr{A}\vec{x} = \vec{b}$, con
    $\matr{A} \in M_{n,n}(\mathbb{R})$ e invertibile,
    $\vec{x} \in \mathbb{R}^n$  e $\vec{b} \in \mathbb{R}^n$
    è possibile calcolare gli elementi del vettore soluzione come

    \[x_i=\frac{\det(\matr{A}(i))}{\det(\matr{A}	)}\]
 
    dove $\matr{A}(i)$ è la matrice formata sostituendo la $i$-esima colonna di
    $\matr{A}$ con il vettore $\vec{b}$.

\end{theorem}

\framebreak

\begin{exampleblock}{Sistemi di equazioni rettangolari $m < n$}
    Dato un sistema di $m$ equazioni in $n$ incognite ($m$ < $n$)
    
  sia $\operatorname{rango}\matr{A} = \operatorname{rango}[\matr{A}|\vec{b}]$;
  
  sia $\operatorname{rango}\matr{A} = m$ (se non è uguale si eliminino le equazioni linearmente dipendenti);
  
      allora il sistema ha infinite soluzioni in quanto $n - m$ variabili sono arbitrarie.
 \end{exampleblock}
\end{frame}

\end{document}
