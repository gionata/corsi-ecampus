\documentclass{beamer}
\def\presentationtype{0}
\input{../template/savoia_benincasa}

\begin{document}

\generatitolo

\section{La storia, in breve}

\begin{frame}[allowframebreaks]{Breve storia della Programmazione Lineare}
% Anteprima del sorgente dal paragrafo 241 al 264

La \emph{Programmazione Lineare} (brevemente, PL) è una disciplina
matematica relativamente giovane, che si fa comunemente risalire al
1947, quando G. B. \mbox{Dantzig} propose un algoritmo, l'\emph{algoritmo
del simplesso}, come metodo efficiente per risolvere problemi coinvolgenti
grandezze lineari. Al tempo Dantzig lavorava nello SCOOP (\emph{Scientific
Computation of Optimum Programs}), un gruppo di ricerca americano
risultante da una intensa attività scientifica durante la seconda
guerra mondiale, che mirava a razionalizzare la logistica legata alle
operazioni di guerra.

\framebreak

Nel 1939 il matematico sovietico Kantorovitch aveva già proposto un
simile metodo per l'analisi di piani economici, ma il suo contributo
è stato a lungo ignorato dalla comunità scientifica occidentale. Sembra
che persino Fourier (nel 1827) avesse già pensato a metodi di questo
genere per trovare soluzioni realizzabili di un sistema di disuguaglianze
lineari.

\framebreak

Quello che rende il contributo di G.B. Dantzig così importante
rispetto ai precedenti, è la concomitanza con altri due fenomeni:
\begin{itemize}
\item il considerevole sviluppo del computer che ha permesso l'implementazione
di algoritmi per la risoluzione di problemi reali di dimensioni considerevoli;
\item il parallelo sviluppo di una struttura matriciale (\emph{Interindustry
Input-Output Model}) per lo studio dell'economia proposto da W.A.
Leontieff, che con il suo lavoro ha mostrato come l'intera economia
potesse essere rappresentata in una sorta di struttura di programmazione
lineare.
\end{itemize}

Al contrario dei metodi proposti precedentemente, il metodo del simplesso
si rivelò efficiente nella pratica e questo, unitamente al simultaneo
avvento dei calcolatori elettronici, decretò il successo della PL.

Per questi motivi il metodo di Programmazione Lineare ha cominciato
ad essere considerato sia uno strumento efficiente per il calcolo
delle soluzioni di una larga scala di problemi di ottimizzazione,
sia un paradigma generale di equilibrio economico tra diversi settori
di scambio di risorse e servizi, attirando così l'interesse di larga
parte della ricerca scientifica.

\framebreak

Lo sviluppo della PL è guidato dalle sue applicazioni
nell'economia e nella gestione. Inizialmente Dantzig ha sviluppato
il metodo del simplesso per risolvere i problemi di pianificazione
dei voli della Air Force americana, e problemi di pianificazione e
di scheduling ancora dominano le applicazioni di PL.

\framebreak

Il successo è invece dovuto al fatto che essa si presta
a modellare con facilità una larga varietà di attività economiche,
come ad esempio la produzione da risorse scarse (allocazione ottima
di risorse), la logistica (molti aspetti del trasporto e dispiegamento
di risorse), lo scheduling (assegnamento di personale e macchine ai
lavori), l'equilibrio economico (equilibrio domanda/offerta in un
mercato competitivo), la copertura dai rischi (analisi di portfolio),
etc.
\end{frame}

\section{Il modello}

\begin{frame}[allowframebreaks]{Il modello di programmazione lineare}

Il problema generale della PL è detto \emph{programma lineare} e consiste
in un problema di ottimizzazione caratterizzato dalle seguenti proprietà:
\begin{enumerate}
\item un numero finito $n$ di variabili $x_{j}$ , che possono assumere
valori reali non negativi;
\item una funzione obiettivo lineare, cioè del tipo: 
\[
f(x)=\vec{c}^{T}\vec{x}=c_{1}x_{1}+c_{2}x_{2}+\cdots+c_{n}x_{n}=\sum_{j=1}^{n}c_{j}x_{j},
\]
dove $\vec{c}\in\mathbb{R}^{n}$ è il vettore dei costi (fissato) ed $\vec{x}\in\mathbb{R}^{n}$
è il vettore delle variabili;
\item l'insieme realizzabile è definito da un insieme finito di equazioni
e/o disequazioni, cioè vincoli lineari del tipo $\vec{a}\vec{x}=b$, $\vec{a}\vec{x}\leq b$
e $\vec{a}\vec{x}\geq b$, dove $\vec{a}\in\mathbb{R}^{n}$ e $b\in\mathbb{R}$.
\end{enumerate}
Un problema di PL può sempre essere espresso per mezzo della formulazione
seguente: 
\[
\max\{\vec{c}^{T}\vec{x}:\matr{A}\vec{x}\leq b,\vec{x}\geq\vec{0}\},
\]
dove $\matr{A}$ è una matrice reale $m\times n$ e $\vec{b}\in\mathbb{R}^{m}$
ed $m$ è il numero di vincoli. Notiamo che un problema in cui
si voglia minimizzare anziché massimizzare la funzione obiettivo,
si può riscrivere in \emph{forma standard} (cioè di minimo
e con vincoli di uguaglianza)
cambiando il segno al vettore $\vec{c}$ dei coefficienti di costo
e introducendo opportune variabili \emph{ausiliarie}.

\framebreak

Un qualsiasi problema di PL può essere scritto come problema
di minimizzazione.

Un vettore $\vec{x}$ che soddisfi i vincoli del programma lineare, rappresentati
dall'espressione $\matr{A}x\leq \vec{b}$, $\vec{x}\geq\vec{0}$, è detto \emph{soluzione realizzabile}
del programma, e tale vettore è detto \emph{soluzione ottima} se oltre
ad essere una soluzione realizzabile massimizza (o minimizza, a seconda
del problema) la funzione obiettivo.
\end{frame}

\section{La ``forma standard''}

\begin{frame}{Forma standard -- notazione algebrica}
$$
   \begin{array}{rlllllr}
    \min z = & c_{1\phantom{1}} x_1 & + c_{2\phantom{2}} x_2 & + \cdots  & + c_{n\phantom{n}} x_n \\
     & a_{11}x_1 & +a_{12}x_2 & + \cdots  & + a_{1n}x_n & = & b_1 \\
     & a_{21}x_1 & +a_{22}x_2 & + \cdots  & + a_{2n}x_n & = & b_2 \\
	 &  &  &  \vdots \\
     & a_{m1}x_1 &  +a_{m2}x_2 &  + \cdots &  + a_{mn}x_n &  = &  b_m \\
      &	x_1 \geq 0,  &  x_2 \geq 0,  &  \cdots  & x_n \geq 0
   \end{array}
$$
\end{frame}

\begin{frame}{Forma standard -- notazione algebrica compatta}
   \begin{gather*}
     \min z =  \sum_{j=1}^{n}c_{j}x_{j} \\
      \sum_{j=1}^{n} a_{ij}x_j  =  b_i,\quad i = 1, 2, \ldots, m \\
     	x_{j} \geq 0, \quad j = 1, 2, \ldots, n    
   \end{gather*}
\end{frame}

\begin{frame}{Forma standard -- notazione matriciale}
   \begin{gather*}
     \min z =  \vec{c}^T \vec{x} \\
	\matr{A}\vec{x} = \vec{b} \\
     	\vec{x} \geq \vec{0}    
   \end{gather*}
\end{frame}

\section{Lo spazio delle soluzioni}

\begin{frame}[allowframebreaks]{Soluzioni della Programmazione Lineare}
Ogni istanza di un programma lineare rientra in una fra tre categorie
distinte; infatti, un programma lineare si dice:
\begin{enumerate}
\item \emph{irrealizzabile}, se non esiste alcuna soluzione realizzabile
al problema, non ci sono cioè vettori $x$ per i quali tutti i vincoli
del problema siano soddisfatti;
\item \emph{illimitato}, se i vincoli non contengono sufficientemente la
funzione di costo in modo che per ogni soluzione realizzabile data,
possa essere trovata un'altra soluzione realizzabile che produca un
ulteriore miglioramento della funzione di costo;
\item \emph{ottimo}, se non è irrealizzabile o illimitato, cioè se la funzione
di costo ha un unico valore minimo (o massimo); si noti però che questo
non significa che i valori delle variabili che producono la soluzione
ottima siano unici.
\end{enumerate}
\end{frame}

\section{Perché ci interessa}

\begin{frame}[allowframebreaks]{Importanza della Programmazione Lineare}
I problemi di PL formano una delle più importanti classi di modelli
di ottimizzazione, per i seguenti motivi:
\begin{itemize}
\item molti problemi di ottimizzazione nella pratica possono essere modellizzati
come problemi di PL;
\item la PL appartiene alla classe dei problemi polinomiali, cioè ``facili'';
in effetti, per tali problemi sono disponibili algoritmi estremamente
efficienti in grado di risolvere anche istanze di dimensione molto
grande;
\item la PL fornisce gli strumenti fondamentali per l'analisi e la costruzione
di algoritmi efficienti per molte altre classi di problemi.
\end{itemize}
\end{frame}

\end{document}