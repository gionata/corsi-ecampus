\documentclass{beamer}
\def\presentationtype{0}
\input{../template/savoia_benincasa}
\setbeamertemplate{caption}[numbered]

\begin{document}

\generatitolo

\section{Un problema di pianificazione della produzione ittica}

\begin{frame}[allowframebreaks]
{Pianificare la produzione ittica -- descrizione}

Tratto da: Gennaro Improta, \textit{Programmazione Lineare.} \textsl{Capitolo quarto -- Impostazione di modelli in programmazione lineare}, pp. 87--90. Edizioni Scientifiche Italiane, 2004
\\~\\

Un'azienda dedita all'allevamento di specie ittiche dispone di tre vasche costruite con la medesima geometria.
Le vasche possono essere suddivise in settori separati per allevare fino a 4 specie diverse.
Ogni vasca opera con personale locale.

\begin{table}
 \begin{tabular}{ccc}
  \toprule
  \multicolumn{1}{c}{\centering\textbf{Vasca}} &
  \multicolumn{1}{p{2cm}}{\centering\textbf{Estensione (ha)}} &
  \multicolumn{1}{p{2.5cm}}{\centering\textbf{Acqua disponibile (m$^3$)}}\\
  \midrule
  1 & 1,3 & 156000\\
  2 & 2,6 & 273000\\
  3 & 1,4 & 189000\\
  \bottomrule
 \end{tabular}
  \caption{Estensione delle vasche e acqua disponibile}\label{tbl:estensione}
\end{table}

In tab.~\ref{tbl:estensione} sono fornite le estensioni, in ettari, delle vasche e, con riferimento ad un ciclo di produzione,
i quantitativi disponibili, in m$^3$, di acqua da utilizzare per il rinnovo.
Il mangime complessivamente utilizzabile nel periodo \`e stato stimato pari a 5500 t.
Le condizioni climatiche, il quantitativo di mangime disponibile ed un'analisi della domanda hanno indotto i responsabili
dell'azienda a puntare sull'allevamento di tre specie: orate (specie 1), spigole (specie 2) e trote salmonate (specie 3).
Su di esse hanno deciso di fondare il piano annuale di produzione.
Le specie ittiche si differenziano per il ricavo che sono in grado di produrre, per i fabbisogni di acqua e di mangime
necessari per ogni ettaro di vasca, come indicato in tab.~\ref{tbl:fabbisogni}.


\begin{table}
 \begin{tabular}{rrrr}
  \toprule
  \multicolumn{1}{c}{\textbf{Specie}} &
  \multicolumn{1}{p{2cm}}{\centering\textbf{Ricavo per ha (\EUR{})}} &
  \multicolumn{1}{p{2cm}}{\centering\textbf{Acqua di rinnovo (m$^3$/ha)}} &
  \multicolumn{1}{p{2cm}}{\centering\textbf{Mangime (t/ha)}}\\
  \midrule
  1 & 250000 & 110000 &  875\\
  2 & 200000 &  80000 & 1300\\
  3 & 150000 &  60000 &  700\\
  \bottomrule
 \end{tabular}
  \caption{Ricavi, fabbisogni di acqua e di mangime}\label{tbl:fabbisogni}
\end{table}

Le estensioni complessive delle vasche destinabili a ciascuna specie sono state definite, a priori, come presentato
in tab.~\ref{tbl:quote}.

\begin{table}
 \begin{tabular}{rr}
  \toprule
  \multicolumn{1}{c}{\textbf{Specie}} &
  \multicolumn{1}{p{2.5cm}}{\centering\textbf{Quantit\`a max (ha)}}\\
  \midrule
  1 & 2,8\\
  2 & 2,5\\
  3 & 1,2\\
  \bottomrule
 \end{tabular}
  \caption{Quote di vasche}\label{tbl:quote}
\end{table}

Il personale richiesto \`e proporzionale all'estensione della vasca e per motivi di equit\`a
si vuole che il rapporto tra il numero di lavoratori per stabilimento coincida col
rapporto tra la superficie della vasca utilizzata e quella disponibile.\\~\\

\textbf{Determinare}

Determinare per ogni impianto la superficie da destinare ad ogni specie,
sotto le restrizioni introdotte e con l'obiettivo di massimizzare il ricavo.
\end{frame}

\section{Costruzione del modello}

\begin{frame}{Insiemi}
  \begin{description}
	  \item[Specie] insieme delle specie ittiche
	  \item[Vasche] insieme degli impianti (vasche)
	  \item[Allocazione] insieme $\rm Specie \times \rm Vasche$
  \end{description}
\end{frame}

 \begin{frame}{Parametri}
		\begin{description}
			\item[ricavo] ricavo dalla vendita della produzione di un ettaro di vasca della specie $i$-esima;
			\item[quota] superficie riservata alla specie $i$-esima;
			\item[fabb\_acqua] fabbisogno rinnovo acqua della specie $i$-esima per ha;
			\item[fabb\_mangime] fabbisogno di mangime della specie $i$-esima per ha;
			\item[estensione] estensione dell'impianto $j$-esimo;
			\item[disp\_acqua] disponibilit\`a di acqua di rinnovo dell'impianto $j$-esimo;
			\item[disp\_mangime] disponibilit\`a di mangime.
		\end{description}
\end{frame}

\begin{frame}{Variabili decisionali}
\begin{block}{Variabili}
	\begin{description}
		\item[$x_{ij}$] superficie da destinare alla specie $i$ della vasca dell'impianto $j$.
	\end{description}
\end{block}

  Le variabili sono continue e non negative.

\end{frame}

\begin{frame}{Funzione obiettivo}
Si deve determinare la superficie di ogni impianto da destinare ad ogni specie ittica.

In generale:

$$  \sum_{\indice{i} \in \insieme{Specie}} \parametro{ricavo}_{\indice{i}}\cdot
         \left(\sum_{\indice{j} \in \insieme{Vasche}} \variabile{x}_{\indice{ij}}\right)$$

Il ricavo \`e espresso da una funzione lineare: rispetta le ipotesi di proporzionalit\`a, additivit\`a e divisibilit\`a.
\end{frame}

\begin{frame}[allowframebreaks]{Vincoli}

\textbf{Estensione delle vasche}

$$ \sum_{\parametro{i} \in \insieme{Specie}} \variabile{x}_{\indice{ij}} \leq \parametro{estensione}_{\indice{j}} \quad \indice{j} \in \insieme{Vasche}$$

\textbf{Disponibilit\`a acqua di rinnovo}

$$ \sum_{\indice{i} \in \insieme{Specie}} \parametro{fabb\_acqua}_{\indice{i}} \variabile{x}_{\indice{ij}} \leq \parametro{disp\_acqua}_{\indice{j}} \quad \indice{j} \in \insieme{Vasche}$$

\framebreak

\textbf{Disponibilit\`a di mangime}

$$ \sum_{\indice{i} \in \parametro{Specie}} \parametro{fabb\_mangime}_{\indice{i}} \cdot
      \left( \sum_{\indice{j} \in \insieme{Vasche}} \variabile{x}_{\indice{ij}} \right) \leq \parametro{disp\_mangime}$$

\textbf{Domanda}

$$ \sum_{\indice{j} \in \insieme{Vasche}} \variabile{x}_{\indice{ij}} \leq \parametro{quota}_{\indice{i}} \quad \indice{i} \in \insieme{Specie}$$

\textbf{Equit\`a}

$$ \frac{\sum\limits_{\indice{i} \in \insieme{Specie}} \variabile{x}_{\indice{i,j}}}{\parametro{estensione}_{\indice{j}}} =
   \frac{\sum\limits_{\indice{i} \in \insieme{Specie}} \variabile{x}_{\indice{i,(j+1)\%|\insieme{Vasche}|}}}{\parametro{estensione}_{\indice{i,(j+1)\%|\insieme{Vasche}|}}}
   \quad \indice{j} \in \insieme{Vasche}$$

\textbf{Fisica realizzabilit\`a (non negativit\`a)}

$$ \variabile{x}_{\indice{ij}} \geq 0 $$
\end{frame}

\end{document}
