\documentclass{beamer}
\usepackage{booktabs}
\def\presentationtype{3}
\input{../template/savoia_benincasa}
\setbeamersize{description width=10pt}

\setbeamertemplate{enumerate item}{Es. \nolezione.\presentationtype.\arabic{enumi})}
\begin{document}

\generatitolo

\begin{frame}{\esercizi}

Risolvere il sistema con i metodi di Gauss e di Gauss-Jordan:

\[
\left\lbrace
\begin{array}{rrrrrrrcr}
 x_1 & &       &+x_3& & & & = &  60 \\
     & &   x_2 &&+x_4 & & & = &  50  \\
 x_1 &+& 2 x_2 && & &+x_5& = & 120
\end{array}
\right.
\]

Il sistema \`e rattangolare?


Esistono soluzioni? Se s\`i, quante?

Cosa fa il comando \alert{\tt MatriceRigheRidotte} di GeoGebra?

\end{frame}

\end{document}

