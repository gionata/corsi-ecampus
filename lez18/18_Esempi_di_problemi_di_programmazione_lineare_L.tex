\documentclass{beamer}
\def\presentationtype{0}
\input{../template/savoia_benincasa}
\setbeamertemplate{caption}[numbered]

\begin{document}

\generatitolo

\section{Un problema di assunzioni ed addestramento}

\begin{frame}[allowframebreaks]
{Assunzioni ed addestramento -- descrizione}

Tratto da: Gennaro Improta, \textit{Programmazione Lineare.} \textsl{Capitolo quarto -- Impostazione di modelli in programmazione lineare}, pp. 100--104. Edizioni Scientifiche Italiane, 2004
\\~\\

Un'azienda specializzata nella gestione di impianti di riscaldamento si accinge, in vista della stagione invernale, a realizzare un programma di addestramento per tecnici da specializzare nella conduzione degli impianti.

Un'analisi dei dati storici e la consistenza del parco impianti hanno consentito di stimare i fabbisogni dei tecnici, per i primi tre mesi freddi, riportati in tab.~\ref{tbl:fabbisogno}.

\begin{table}
 \begin{tabular}{lr}
  \toprule
  \multicolumn{1}{c}{\centering\textbf{Mese}} &
  \multicolumn{1}{p{\widthof{necessari\ \ }}}{\centering\textbf{Tecnici necessari}}\\
  \midrule
  Ottobre  & 100 \\
  Novembre & 150 \\
  Dicembre & 200 \\
  \bottomrule
 \end{tabular}
  \caption{Fabbisogno dei tecnici negli impianti}\label{tbl:fabbisogno}
\end{table}

L'azienda ha disponibili, alla fine del mese di settembre, 130 tecnici gi\`a addestrati ed ha, inoltre, la necessit\`a di disporre di 250 tecnici per il mese di gennaio.

Ogni corso di addestramento ha la durata di un mese e il 10\% dei tecnici formati divengono istruttori per i corsi successivi. In media il 30\% degli iscritti ad un corso non lo completa con successo e viene estromesso dall'organizzazione.

Le paghe previste, espresse in Euro, sono mostrate in tab.~\ref{tbl:paghe}.

\begin{table}
 \begin{tabular}{lr}
  \toprule
  \multicolumn{1}{c}{\centering\textbf{Posizione di lavoro}} &
  \multicolumn{1}{p{\widthof{mensile\ \ }}}{\centering\textbf{Paga mensile (\EUR{})}}\\
  \midrule
  In addestramento  & 400 \\
  Addestrati        & 700 \\
  Inattivi          & 500 \\
  \bottomrule
 \end{tabular}
  \caption{Paghe previste}\label{tbl:paghe}
\end{table}

\textbf{Determinare}

Determinare il programma di assunzione e addestramento che consente di rispondere, al minimo costo, ai bisogni dell'impresa.
\end{frame}

\section{Costruzione del modello}

\begin{frame}{Insiemi}
  \begin{description}
	  \item[Specie] insieme delle specie ittiche
	  \item[Vasche] insieme degli impianti (vasche)
	  \item[Allocazione] insieme $\rm Specie \times \rm Vasche$
  \end{description}
\end{frame}

 \begin{frame}{Parametri}
		\begin{description}
			\item[ricavo] ricavo dalla vendita della produzione di un ettaro di vasca della specie $i$-esima;
			\item[quota] superficie riservata alla specie $i$-esima;
			\item[fabb\_acqua] fabbisogno rinnovo acqua della specie $i$-esima per ha;
			\item[fabb\_mangime] fabbisogno di mangime della specie $i$-esima per ha;
			\item[estensione] estensione dell'impianto $j$-esimo;
			\item[disp\_acqua] disponibilit\`a di acqua di rinnovo dell'impianto $j$-esimo;
			\item[disp\_mangime] disponibilit\`a di mangime.
		\end{description}
\end{frame}

\begin{frame}{Variabili decisionali}
\begin{block}{Variabili}
	\begin{description}
		\item[$x_{ij}$] superficie da destinare alla specie $i$ della vasca dell'impianto $j$.
	\end{description}
\end{block}

  Le variabili sono continue e non negative.

\end{frame}

\begin{frame}{Funzione obiettivo}
Si deve determinare la superficie di ogni impianto da destinare ad ogni specie ittica.

In generale:

$$  \sum_{\indice{i} \in \insieme{Specie}} \parametro{ricavo}_{\indice{i}}\cdot
         \left(\sum_{\indice{j} \in \insieme{Vasche}} \variabile{x}_{\indice{ij}}\right)$$

Il ricavo \`e espresso da una funzione lineare: rispetta le ipotesi di proporzionalit\`a, additivit\`a e divisibilit\`a.
\end{frame}

\begin{frame}[allowframebreaks]{Vincoli}

\textbf{Estensione delle vasche}

$$ \sum_{\parametro{i} \in \insieme{Specie}} \variabile{x}_{\indice{ij}} \leq \parametro{estensione}_{\indice{j}} \quad \indice{j} \in \insieme{Vasche}$$

\textbf{Disponibilit\`a acqua di rinnovo}

$$ \sum_{\indice{i} \in \insieme{Specie}} \parametro{fabb\_acqua}_{\indice{i}} \variabile{x}_{\indice{ij}} \leq \parametro{disp\_acqua}_{\indice{j}} \quad \indice{j} \in \insieme{Vasche}$$

\framebreak

\textbf{Disponibilit\`a di mangime}

$$ \sum_{\indice{i} \in \parametro{Specie}} \parametro{fabb\_mangime}_{\indice{i}} \cdot
      \left( \sum_{\indice{j} \in \insieme{Vasche}} \variabile{x}_{\indice{ij}} \right) \leq \parametro{disp\_mangime}$$

\textbf{Domanda}

$$ \sum_{\indice{j} \in \insieme{Vasche}} \variabile{x}_{\indice{ij}} \leq \parametro{quota}_{\indice{i}} \quad \indice{i} \in \insieme{Specie}$$

\textbf{Equit\`a}

$$ \frac{\sum\limits_{\indice{i} \in \insieme{Specie}} \variabile{x}_{\indice{i,j}}}{\parametro{estensione}_{\indice{j}}} =
   \frac{\sum\limits_{\indice{i} \in \insieme{Specie}} \variabile{x}_{\indice{i,(j+1)\%|\insieme{Vasche}|}}}{\parametro{estensione}_{\indice{i,(j+1)\%|\insieme{Vasche}|}}}
   \quad \indice{j} \in \insieme{Vasche}$$

\textbf{Fisica realizzabilit\`a (non negativit\`a)}

$$ \variabile{x}_{\indice{ij}} \geq 0 $$
\end{frame}

\end{document}
