\begin{frame}{PL in forma canonica}
{\small
$$
\begin{array}{cccrcrcrcrcc}
	\min z	&=	&-&20 x_{1}	&-&30 x_{2}	&	&	&	&\\
	\mbox{s.t.}	&	& & x_{1}	&-&1 x_{2}	&+& x_{3}	&	&	&=	&5\\
	&	&-&1 x_{1}	&+& x_{2}	&	&	&+& x_{4}	&=	&5\\
	&	&\multicolumn{8}{c}{x_{1} \geq 0,\ x_{2} \geq 0,\ x_{3} \geq 0,\ x_{4} \geq 0}
\end{array}
$$
}

\end{frame}

\begin{frame}[fragile]{Iterazione 1: soluzione migliorabile}
\centering
\begin{tikzpicture}
{\tiny
\node[%
  align=center,
  text width=3.5em,
  text height=5ex,
  row 1/.style={gray, text height=1em},
  row 2/.style={blue!50!gray, text height=1em},
  row 3/.style={blue},
  column 1/.style={gray,text width=1em},
  column 2/.style={red!50!gray,text width=1.5em},
  column 3/.style={red},
  column 8/.style={gray},
  row 3 column 1/.style={gray},
  row 3 column 2/.style={red!50!gray},
  row 3 column 3/.style={red},
  matrix of math nodes] (M)
{%
% Indice delle righe (M-1)
~&~&0&1&2&3&4\\
% Intestazione delle colonne (M-2)
~ &~&\mbox {rhs}&x_ {1}&x_ {2}&x_ {3}&x_ {4}&Rapp.\\
% Riga 0
0&z&0&-20&-30&0&0\\
% Riga 1
1&x_ {3}&5&1&-1&1&0&\\
% Riga 2
2&x_ {4}&5&-1&1&0&1&5\\
}; %end of matrix node
% riquadro
\draw(M-3-3.north west) -- (M-3-7.north east) -- (M-5-7.south east) -- (M-5-3.south west) -- cycle;
% separatore orizzontale
\draw[dashed] (M-4-3.north west) -- (M-4-7.north east);
% separatore verticale
\draw[dashed] (M-3-4.north west) -- (M-5-4.south west);
% entra x_k
\draw[->,thick,red!50!blue] (M-1-5.north) to (M-2-5.north);
% esce x_{s_h}
\draw[->,thick,red!50!blue] (M-2-7.north) to (M-1-7.north);
% il pivot
\fill[yellow!50,fill opacity=0.25,thick,draw=blue!60!black] (M-5-5.base) circle (1.5 em);
}
\end{tikzpicture}
%sono fragile: lasciami uno spazio vuoto

\end{frame}

\begin{frame}[fragile]{Iterazione 2: soluzione illimitata}
\centering
\begin{tikzpicture}
{\tiny
\node[%
  align=center,
  text width=3.5em,
  text height=5ex,
  row 1/.style={gray, text height=1em},
  row 2/.style={blue!50!gray, text height=1em},
  row 3/.style={blue},
  column 1/.style={gray,text width=1em},
  column 2/.style={red!50!gray,text width=1.5em},
  column 3/.style={red},
  column 8/.style={gray},
  row 3 column 1/.style={gray},
  row 3 column 2/.style={red!50!gray},
  row 3 column 3/.style={red},
  matrix of math nodes] (M)
{%
% Indice delle righe (M-1)
~&~&0&1&2&3&4\\
% Intestazione delle colonne (M-2)
~ &~&\mbox {rhs}&x_ {1}&x_ {2}&x_ {3}&x_ {4}\\
% Riga 0
0&z&150&-50&0&0&30\\
% Riga 1
1&x_ {3}&10&0&0&1&1\\
% Riga 2
2&x_ {2}&5&-1&1&0&1\\
}; %end of matrix node
% riquadro
\draw(M-3-3.north west) -- (M-3-7.north east) -- (M-5-7.south east) -- (M-5-3.south west) -- cycle;
% separatore orizzontale
\draw[dashed] (M-4-3.north west) -- (M-4-7.north east);
% separatore verticale
\draw[dashed] (M-3-4.north west) -- (M-5-4.south west);
% entra x_k
\draw[->,thick,red!50!blue] (M-1-4.north) to (M-2-4.north);
% esce x_{s_h}
\draw[->,thick,red!50!blue] (M-2-3.north) to (M-1-3.north);
}
\end{tikzpicture}
%sono fragile: lasciami uno spazio vuoto

\end{frame}

