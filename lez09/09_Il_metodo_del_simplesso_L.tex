\documentclass{beamer}
\def\presentationtype{0}
\input{../template/savoia_benincasa}

\begin{document}

\generatitolo

\section{Il teorema fondamentale del simplesso}

\begin{frame}[allowframebreaks]{Il teorema fondamentale del simplesso}

Si consideri un programma lineare di minimo in forma canonica forte rispetto ad una sequenza di indici di base $S$. Vale una delle seguenti alternative:

\begin{theorem}[del metodo del simplesso]
\begin{enumerate}
\item se  $c_j \geq 0$ per ogni $j \notin S$, allora la soluzione di base $\vecX^S$ associata alla forma canonica \`e ottima;

\item se se esiste un indice $k \notin S$ tale che $c_k < 0$ e $\matrA^k \leq \vec{0}$ la funzione obiettivo \`e inferiormente illimitata;

\item se esiste un indice $k \notin S$ tale che $c_k < 0$ e in corrispondenza esiste almeno un indice $i, 1 \leq i \leq m$, tale che $a_{ik} > 0$, allora si pu\`o fare un'operazione pivot su un opportuno elemento della colonna $k$ trasformando il problema in uno equivalente e in forma canonica forte rispetto ad una sequenza di indici di base $S^\prime$ per cui vale $z^{S^\prime} \leq z^S$.
\end{enumerate}
\end{theorem}
\end{frame}

\subsection{Test di ottimalit\`a}

\begin{frame}[allowframebreaks]{Soluzione ottima (di valore finito)}
\begin{block}{\circled{1} Test di ottimalit\`a}
se  $c_j \geq 0$ per ogni $j \notin S$, allora la soluzione di base $\vecX^S$ associata alla forma canonica \`e ottima
\end{block}

Permette di certificare ottimalit\`a di una soluzione.

{\small Il valore assunto all'ottimo dalle variabili decisionali (il vettore $\vecX^\star$), riordinato, vale
$$
\begin{array}{ccc}
\left[
	\begin{array}{c}
			\vecX_B\\
			\vecX_N
	\end{array}
\right]
&=&
\left[
	\begin{array}{c}
		\left[\begin{array}{cccc}x_{s_1}&x_{s_2}&\cdots&x_{s_m}\end{array}\right]^T \\
		\left[\begin{array}{cccc}0&0&\cdots&0\end{array}\right]^T
	\end{array}
\right]\\
&&\underbrace{%
		\phantom{\begin{array}{cccc}0&0&\cdots&0\end{array}}
}_{m-n}
\end{array}
$$

il valore assunto all'ottimo dalla funzione obiettivo ($z^\star$) \`e quello della soluzione di base ammissibile corrente $S$: sul tableau di un problema di minimo \`e il valore $-\bar{d}$, $\bar{d}$ se il problema di partenza era di massimo.}

\framebreak

Far entrare in base la variabile fuori base $x^S_k (=0)$  comporta la variazione del valore della funzione obiettivo di $c_k x_k^{S^\prime}$, dove $x_k^{S^\prime}$ \`e non negativo. Se tutti i coefficienti di costo delle variabili fuori base sono non negativi allora il valore assunto dalla funzione obiettivo in corrispondenza di qualunque altra soluzione di base ammissibile sar\`a maggiore o uguale (peggiore o uguale) del valore in corrispondenza della sequenza di indici di base $S$.

\framebreak

{\footnotesize
Analizziamo il passaggio dalla base $S$ alla base adiacente $S^\prime$
in cui sostituiamo la variabile di base $x_{j}$ 
nella posizione $h$ della lista degli indici di base
($j = s_h$) con la variabile $x_k$ precedentemente fuori base.}

{\small
\[
\bordermatrix{
   S    &  -z   & x_{1}    &  x_{2}	&\cdots& x_{k}       &\cdots& x_{j}&\cdots&x_{n} \cr
   f.o.\   &  -z^S      &  c_1       & c_2         & \cdots&c_k>0      &\cdots&     0      &\cdots&c_{n} \cr
	1    &  b_{1}& a_{11} & a_{12}    & \cdots&a_{1k}    &\cdots&     0      &\cdots&a_{1n} \cr
\vdots & \vdots &\vdots &\vdots &\vdots &\vdots &\vdots &\vdots &\vdots &\vdots \cr
	h    &  b_{h}& a_{h1} & a_{h2}    & \cdots&a_{hk}>0&\cdots&     1      &\cdots&a_{hn} \cr
\vdots & \vdots &\vdots &\vdots &\vdots &\vdots &\vdots &\vdots &\vdots &\vdots \cr
	m    &  b_{m}& a_{m1} & a_{m2}    & \cdots&a_{mk}&\cdots&     0      &\cdots&a_{mn} \cr
}
\]
}

{\tiny 
\[
\bordermatrix{
   S^\prime     &  -z    & x_{1}    &  x_{2}	&\cdots& x_{k}       &\cdots& x_{j}&\cdots&x_{n} \cr
   f.o.\   &  -z^S-\frac{c_k}{a_{hk}}b_{h}      & c_1-\frac{c_k}{a_{hk}}a_{h1}       & c_2-\frac{c_k}{a_{hk}}a_{h2}         & \cdots&c_k-\frac{c_k}{a_{hk}}a_{hk}=0      &\cdots&     -\frac{c_k}{a_{hk}}a_{hj}      &\cdots&c_{n}-\frac{c_k}{a_{hk}}a_{hn} \cr
	1    &  b_{1}-\frac{a_{1k}}{a_{hk}}b_h& a_{11}-\frac{a_{1k}}{a_{hk}}a_{h1} & a_{12}-\frac{a_{1k}}{a_{hk}}a_{h2} & \cdots&a_{1k}-\frac{a_{1k}}{a_{hk}}a_{hk}=0 &\cdots& -\frac{a_{1k}}{a_{hk}}a_{hj} &\cdots&a_{1n}-\frac{a_{1k}}{a_{hk}}a_{hn} \cr
\vdots & \vdots &\vdots &\vdots &\ddots &\vdots &\ddots &\vdots &\ddots &\vdots \cr
	h    &\frac{b_{h}}{a_{hk}}&\frac{a_{h1}}{a_{hk}} &\frac{a_{h2}}{a_{hk}}    & \cdots&\frac{a_{hk}}{a_{hk}}\phantom{-\frac{a_{mk}}{a_{hk}}a_{hj}}=1&\cdots&\frac{1}{a_{hk}}      &\cdots&\frac{a_{hn}}{a_{hk}} \cr	
\vdots & \vdots &\vdots &\vdots &\ddots &\vdots &\ddots &\vdots &\ddots &\vdots \cr
	m    &  b_{m}-\frac{a_{mk}}{a_{hk}}b_h& a_{m1}-\frac{a_{mk}}{a_{hk}}a_{h1} & a_{m2}-\frac{a_{mk}}{a_{hk}}a_{h2} & \cdots&a_{mk}-\frac{a_{mk}}{a_{hk}}a_{hk}=0&\cdots&     -\frac{a_{mk}}{a_{hk}}a_{hj}      &\cdots&a_{mn}-\frac{a_{mk}}{a_{hk}}a_{hn} \cr
}
\]
}

\begin{itemize}
\item se $c_j \leq 0,\ j=1,2,\ldots,n$ la soluzione \`e
	localmente ottima, per la propriet\`a di
	convessit\`a della funzione obiettivo e della regione
	ammissibile, \`e globalmente ottima
\item se esiste un $c_j$ negativo, $j \notin S$ allora esiste
	una soluzione adiacente che migliora il valore della f.o.\ di
	$-\frac{c_k}{a_{hk}}b_{h}$, dove $\frac{b_{h}}{a_{hk}}$ \`e
	il valore assunto dalla variabile entrante in base nella
	nuova soluzione di base ammissibile
\end{itemize}
\end{frame}

\begin{frame}[allowframebreaks]{Soluzione di base ammissibili -- costi ridotti}

\begin{definition}[costo ridotto]
Dato un problema di P.L. in forma canonica rispetto a una base $\matr{B}$, il coefficiente $\bar{c_j}$ della variabile $x_j$ nella funzione obiettivo si dice costo ridotto della variabile $x_j$ rispetto alla base $\matr{B}$
\end{definition}

\begin{itemize}
\item   Il costo ridotto $\bar{c_j}$ di una variabile fuori base rappresenta l'incremento marginale del costo complessivo (funzione obiettivo di minimo) per ogni unit\`a di variazione in aumento della variabile $x_j$. Il coefficiente di costo si dice ridotto in quanto si determina sottraendo al costo originario $c_j$ una quantit\`a che dipende dalla base $\matr{B}$

\item   I costi ridotti delle variabili in base sono tutti pari a 0 per definizione di forma canonica

\item Il punto \circled{1} del teorema fondamentale del simplesso, pu\`o essere enunciato come:
   
\begin{block}{Test di ottimalit\`a}
Se tutti i costi ridotti rispetto alla base $\matr{B}$ sono non negativi ($\geq 0$) allora la soluzione di base
   associata a $\matr{B}$ \`e ottima.
\end{block}

%\item   Il teorema \`e valido solo nella direzione  $(\bar{c_j}\geq 0, j \notin S) \implies$ (soluzione ottima), mentre l'inversa non \`e necessariamente verificata.
\end{itemize}
\end{frame}

\subsection{Test di limitatezza}

\begin{frame}[allowframebreaks]{Soluzione illimitata}

\begin{block}{\circled{2} Test di limitatezza}
se se esiste un indice $k \notin S$ tale che $c_k < 0$ e $\matrA^k \leq \vec{0}$ la funzione obiettivo \`e inferiormente illimitata
\end{block}

Permette di dimostrare che il problema ha soluzione illimitata.

Essendo negativo il valore di $c_k$ incrementare il valore della variabile fuori base $x_k$ comporta un miglioramento della funzione obiettivo.
Le variabili di base, passando dalla sequenza degli indici $S$ ad $S^\prime$ si modificano come

$$
x^S_{s_i} = b_{i}^S \quad\mapsto\quad x_{s_i}^{S^\prime}=b_i^S-a_{ik}\frac{b_h}{a_{hk}} =b_i^S-a_{ik} x^{S^\prime}_{k}
$$


Comunque si aumenti il valore $x^{S^\prime}_k$ \`e possibile incrementare il valore delle altri variabili
di base senza che queste diventino negative in quanto il loro valore incrementer\`a di
$ -a_{ik} x^{S^\prime}_k,\  (a_{ik} \leq 0)$

\begin{align}
A^k \leq 0,\ x_k \to +\infty &\implies x_{s_i} \to +\infty \nonumber\\
c_k \leq 0, \ x_k \to +\infty &\implies z \to +\infty \nonumber
\end{align}
\end{frame}

\subsection{Test di ammissibilit\`a}

\begin{frame}[allowframebreaks]{Soluzioni adiacenti non peggiori}
\begin{block}{\circled{3} Test di ammissibilit\`a}
se esiste un indice $k \notin S$ tale che $c_k < 0$ e in corrispondenza esiste almeno un indice $i, 1 \leq i \leq m$, tale che $a_{ik} > 0$, allora si pu\`o fare un'operazione pivot su un opportuno elemento della colonna $k$ trasformando il problema in uno equivalente e in forma canonica forte rispetto ad una sequenza di indici di base $S^\prime$ per cui vale $z^{S^\prime} \leq z^S$.
\end{block}

Indica l'esistenza di una soluzione di base ammissibile adiacente a quella determinata dalla sequenza degli indici di base S che corrisponde ad un valore della funzione obiettivo migliore.

Essendo $c_k < 0$ e $x_k$ fuori base sappiamo che la funzione obiettivo migliorer\`a all'aumentare di $x_k$. Il massimo valore che pu\`o assumere la variabile candidata ad entrare in base \`e limitato dal fatto che le altre variabili devono rimanere ammissibili. Facendo entrare in base $x_k$ il valore delle altre variabili varia secondo:
$
x^S_{s_i} = b_{i}^S \quad\mapsto\quad x_{s_i}^{S^\prime}=b_i^S-a_{ik}\frac{b_h}{a_{hk}} =b_i^S-a_{ik} x^{S^\prime}_{k}
$

Dovendo preservare l'ammissibilit\`a delle variabili in $S$ si ha
che queste devono rimanere non negative nella nuova sequenza
degli indici di base $S^\prime$.

Le equazioni per cui i coefficienti $a_{ik} $ sono nulli o positivi
non pongono restrizioni all'incremento di $x_k$.
\`E possibile considerare solo gli indici $i$ per cui $a_{ik}$ \`e
positivo.

Sia $I = \{i : a_{ik} > 0\}$. 
{\small
\begin{eqnarray}
x_{s_i}^{S^\prime}=b_i^S-a_{ik}\frac{b_h}{a_{hk}} =b_i^S-a_{ik} x^{S^\prime}_{k} & i=1,2,\ldots,m \nonumber\\
x_{s_i}^{S^\prime} \geq 0 \implies b_i^S-a_{ik}\frac{b_h}{a_{hk}} \geq 0 & \ i \in I  \nonumber \\
\frac{b_h}{a_{hk}} \leq  \frac{b_i^S}{a_{ik}} & \ i \in I  \nonumber\\
\frac{b_h}{a_{hk}} = \min_{i \in I} \left\lbrace\frac{b_i^S}{a_{ik}}\right\rbrace  \nonumber\\
h = \arg\min_{i \in I} \left\lbrace\frac{b_i^S}{a_{ik}}\right\rbrace  \nonumber
\end{eqnarray}
}
\end{frame}

\begin{frame}[allowframebreaks]{Casi particolari}
Esistono dei casi particolari.

\circled{a} Pi\`u soluzioni corrispondono ad un minimo rapporto
$\min_{i \in I} \left\lbrace\frac{b_i^S}{a_{ik}}\right\rbrace$.

In questo caso \`e possibile scegliere  arbitrariamente la variabile
uscente dalla base per\`o non solo tale variabile assumer\`a valore
nullo ma anche le altre variabili corrispondenti al minimo rapporto
assumeranno valore 0 pur rimanendo in base.
Si ottiene il passaggio ad una soluzione di base ammissibile degenere.

\circled{b} Il minimo rapporto vale 0.
Accade quando una variabile di base vale 0 e il corrispondente
coefficiente $a_{ik} $ \`e positivo. Cambiando base a seguito
dell'operazione di pivot non cambia il valore della funzione
obiettivo.

Sono possibili due casi:
\begin{itemize}
\item la soluzione corrente non \`e ottima:
successivi cambiamenti di base permettono
di migliorare la funzione obiettivo;

\item la soluzione corrente \`e ottima:
con i cambiamenti di base si arriver\`a ad una nuova base,
sempre degenere, con tutti i coefficienti di costo ridotto
positivi.
La combinazione lineare convessa dei vertici esplorati a partire
dalla soluzione corrente fino alla determinazione della soluzione
con costi ridotti positivi contiene infiniti punti di ottimo.
\end{itemize}
\end{frame}

\section{L'algoritmo del simplesso}

\begin{frame}[allowframebreaks]{L'algoritmo del simplesso}
\begin{enumerate}
\item Dato un problema di P.L. lo si trasformi in un problema
	equivalente in forma standard (sempre possibile).

\item Se la forma standard non \`e forma canonica forte, si
	trasformi il problema il forma canonica forte per avere una
	soluzione di base ammissibile.

\item Si esegua il test di ottimalit\`a controllando il segno dei
	costi ridotti delle variabili fuori base.
	
\item Si esegua il test di limitatezza controllando l'esistenza di
	colonne negative della matrice dei coefficienti tecnologici
	in corrispondenza di variabili di base con costo ridotto negativo.


\item Si scelga la variabile entrante in base fra quelle fuori base
	con costo ridotto negativo. Sia tale variabile $x_k$.

\item Si determini la variabile uscente dalla base con il test dei
	minimi rapporti. Sia $h$ la sua posizione nella sequenza degli
	indici di base.

\item Si effettui un'operazione di pivot in posizione $(h, k)$.
	Il nuovo sistema di equazioni preserva la forma canonica.
	Si torni al passo 3.
\end{enumerate}
\end{frame}

\end{document}
