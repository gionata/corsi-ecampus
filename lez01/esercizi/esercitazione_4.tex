\documentclass[italian,12pt]{article}
\usepackage[T1]{fontenc}
\usepackage[latin9]{inputenc}
\usepackage{amsmath}
\usepackage{amssymb}
\usepackage{amsfonts}
\usepackage[italian]{babel}
\usepackage[a4paper, margin=2.5cm,noheadfoot]{geometry}
\usepackage{enumerate}
\begin{document}

\pagestyle{empty}

\section*{Esercitazione n. 4 - Vincoli di eguaglianza}

Determinare i punti di ottimo e calcolare il valore assunto dalla funzione in tali punti.

\begin{enumerate}[1)]
	\item % 1
		\begin{minipage}[c]{0.8\textwidth}%
			\begin{align*}
				\min f(x_{1},x_{2},x_{3}) & =(x_{1}-3)^{2}+(x_{2}+1)^{2}+(x_{3}-2)^{2}\\
					h_{1}(x) & =9-3x_{1}+2x_{2}-4x_{3}=0\\
					h_{2}(x) & =3-x_{1}-2x_{2}=0\end{align*}
			\end{minipage}
			
		\item % 2
		\begin{minipage}[c]{0.8\textwidth}%
			\begin{align*}
				\min f(x_{1},x_{2}) & =(x_{1}-1)^{2}+x_{2}^{2}\\
					h(x) & =x_{1}^2+x_{2}^2-4=0\end{align*}
			\end{minipage}
			
		\item % 3
		\begin{minipage}[c]{0.8\textwidth}%
			\begin{align*}
				\min f(x_{1},x_{2}) & =x_{1}^{2}+x_1 x_2+x_{2}^{2}\\
					h(x) & =x_{1}^2+x_{2}-1=0\end{align*}
			\end{minipage}
			
		\item % 4
		\begin{minipage}[c]{0.8\textwidth}%
			\begin{align*}
				\min f(x_{1},x_{2}) & =x_{1}^{2}-x_{2}^{2}\\
					h(x) & =x_{1}^2+x_{2}^2-1=0\end{align*}
			\end{minipage}
		
		\item %	
			\begin{minipage}[c]{0.8\textwidth}%
			\begin{align*}
				\min f(x_{1},x_{2}) & =x_{1}+x_{2}\\
					h(x) & =x_{1}^2+x_{2}^2-1=0\end{align*}
			\end{minipage}
			
		\item %
			\begin{minipage}[c]{0.8\textwidth}%
			\begin{align*}
				\min f(x_{1},x_{2},x_{3}) & =x_{1} x_{2} + x_{2} x_{3} + x_{1} x_{3}\\
					h(x) & =x_{1}+x_{2}+x_{3}-3=0\end{align*}
			\end{minipage}
\end{enumerate}

\end{document}