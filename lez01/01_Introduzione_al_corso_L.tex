\documentclass{beamer}
\def\presentationtype{0}
\input{../template/savoia_benincasa}

\begin{document}

\generatitolo

\section{\insegnamento}

\begin{frame}{\insegnamento: di cosa tratta}
  Il corso tratta alcuni degli aspetti della
  \emph{Ricerca Operativa} trascurati nell'insegnamento
  del Corso di Laurea.

  La \emph{Ricerca Operativa}
  \begin{itemize}
    \item \`e un approccio scientifico che ha come oggetto lo studio (analisi)
     e la messa a punto di metodologie e strumenti quantitativi per la soluzione
     di problemi decisionali
    \item fornisce metodi e strumenti per aiutare manager, professionisti,
    in generale operatori umani a prendere delle decisioni in situazioni complesse
    (si dice che fornisce un supporto alle decisioni) attraverso modelli matematici ed algoritmi
  \end{itemize} 
\end{frame}

\begin{frame}{\insegnamento: dove si colloca}
  \begin{itemize}
    \item La Ricerca Operativa \`e una disciplina costituita dall'insieme dei modelli (matematici)
    e dei metodi quantitativi (algoritmi) utilizzabili per lo sviluppo e il supporto dei processi decisionali

    \item \`E una materia d'interfaccia fra le discipline matematiche di base e le discipline
    ingegneristiche ed economiche di tipo applicativo

    \item Si colloca a valle dell'Analisi Matematica e dell'Algebra Lineare e si caratterizza come
    matematica applicata

	\item Studia modelli e metodi di ottimizzazione matematica che vengono implementati
	come programmi di calcolo.
  \end{itemize}
\end{frame}

\begin{frame}{\insegnamento: come opera}
 Un consulente di ricerca operativa per fornire supporto alle decisioni deve:
  \begin{itemize}
    \item individuare il problema e le sue caratteristiche.
    Questa fase \`e detta di analisi del problema e conduce ad una descrizione
    formale del sistema
    \item costruire un modello logico-matematico che rappresenti il processo decisionale
    \item determinare una o pi\`u soluzioni al modello, utilizzando gli algoritmi noti pi\`u
    efficienti per la risoluzione di tale modello o attraverso lo sviluppo di nuovi algoritmi
    \item analizzare i risultati ottenuti
  \end{itemize}
\end{frame}

\begin{frame}{\insegnamento: perch\'e si studia}
  \begin{itemize}
    \item per interpretare in modo oggettivo fenomeni che richiedono delle scelte
    \item per poter determinare la decisone migliore
    \item per poter migliorare le prestazioni di un sistema
  \end{itemize}

\end{frame}

\section{Programma del corso}

\begin{frame}{Obiettivi del corso}

  \begin{itemize}
    \item Scopo del corso \`e fornire strumenti avanzati che si basano
	  su metodi di ottimizzazione e modelli di simulazione per
	  risolvere problemi decisionali
	  
    \item Lo studente imparer\`a a formulare problemi di gestione della
	  logistica e della produzione mediante modelli di programmazione
	  matematica e di simulazione
  \end{itemize}
\end{frame}

\begin{frame}{Contenuti del corso}
  \begin{enumerate}
    \item Problemi di ottimizzazione su rete%:
% 	  trasporto,
% 	  assegnamento,
% 	  tecniche reticolari per la gestione dei progetti,
% 	  cammini minimi,
% 	  albero ricoprente di costo minimo
    \item Modelli statistici per le decisioni di gestione della logistica e della produzione

    \item Problemi di gestione e controllo delle scorte

    \item Modelli di simulazione per la gestione delle scorte

    \item Metodi previsionali a breve termine

    \item  Uso di strumenti software per l'ottimizzazione
  \end{enumerate}
\end{frame}

\section{Requisiti di conoscenza}

\begin{frame}{Prerequisiti}
  \begin{itemize}
  \item Conoscenza della notazione matematica usata nei corsi di ``Analisi Matematica''
  \item Conoscenza di alcuni elementi di ``Algebra Lineare''
  \item Conoscenza di alcuni elementi di ``Analisi Matematica''
  \item Nozioni elementari di ``Ricerca Operativa'', in parte richiamate durante il corso
  \item Competenze di ``Analisi Numerica'' e di ``Algoritmi e Strutture Dati'' ritenute utili per sperimentare i metodi al calcolatore
  \end{itemize}
\end{frame}

% \begin{frame}{Requisiti per superare l'esame}
%   \begin{itemize}
%   \item Nozioni elementari di Algebra Lineare
%   \item Nozioni elementari di Analisi Matematica
%   \item Nozioni elementari di Ricerca Operativa
%   \end{itemize}
% \end{frame}

% \begin{frame}{Domande d'esame}
%   \begin{itemize}
%   \item Nozioni elementari di Algebra Lineare
%   \item Nozioni elementari di Analisi Matematica
%   \item Nozioni elementari di Ricerca Operativa
%   \end{itemize}
% \end{frame}

\section{Come e dove studiare}

\begin{frame}{Lezioni del corso}
  \begin{itemize}
    \item Lezioni teoriche costituite da lucidi
    \item Esercizi e aprrofondimenti costituiti da ``Sessioni di Studio''
  \end{itemize}
\end{frame}


\begin{frame}{Sessioni di studio}
    Le sezioni di studio consistono  in
    \begin{itemize}
     \item letture e approfondimenti, con eventuale produzione di una relazione
     \item svolgimento di esercizi
     \item attivit\`a pratiche mediante l'uso di software per la risoluzione di problemi
    \end{itemize}
    
    Tutte le relazioni e \alert{gli esercizi} richiedono la produzione di un elaborato scritto e
    \alert{dovranno essere presentati}, in formato elettronico oppure cartaceo, \alert{in sede d'esame}
 
    Lo \alert{svolgimento degli esercizi \`e fondamentale} per Il superamento dell'esame stesso
\end{frame}

\begin{frame}{Testo di riferimento e bibliografia}

    Il riferimento principale sono le slide del corso
    
    \vspace*{1cm}

    Il libro di testo e riferimento fondamentale per il corso \`e
%    Il libro di testo per le applicazioni della programmazione lineare continua, intera e intera mista ai problemi di gestione della produzione 
    \begin{thebibliography}{9}
      \setbeamertemplate{bibliography item}[book]
      \bibitem{PezzellaFaggioli} F. Pezzella, E. Faggioli, ``Ricerca operativa. Problemi di gestione della produzione'', Pitagora Editrice, Bologna
    \end{thebibliography}
\end{frame}

\begin{frame}{Software di supporto allo studio}
    \begin{description}
     \item[LINDO] \href{http://www.lindo.com/}{www.lindo.com}, versione 14 o successive
     \item[GLPK] \href{http://www.gnu.org/software/glpk/}{www.gnu.org/software/glpk}
    \end{description}
\end{frame}

\section{Modalit\`a d'esame}

\begin{frame}{Esame e modalit\`a di valutazione}
    \begin{itemize}
     \item  In sede di esame lo studente dovr\`a presentare tutti gli elaborati richiesti durante
     lo svolgimento del corso (esercizi svolti, relazioni, eventuali codici software elaborati)
     \item  La prova d'esame sar\`a finalizzata alla verifica della comprensione degli argomenti
     teorici sviluppati nel corso e della capacit\`a di applicazione degli stessi in casi pratici
     \item  Sar\`a verificata la comprensione dei concetti di teoria
     \item  Verr\`a richiesto di risolvere per iscritto alcuni esercizi e verranno formulate domande
     circa gli aspetti teorici connessi a tali esercizi
    \end{itemize}
\end{frame}

\begin{frame}{Interazione col docente}
  \begin{itemize}
    \item posta elettronica: \textcolor{blue}{\href{mailto:gionata.massi@uniecampus.it}{gionata.massi@uniecampus.it}}
    \item ricevimento web, su appuntamento
    \item seminari
  \end{itemize}
\end{frame}

\end{document}